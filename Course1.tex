Machine Learning Foundations: A Case Study Approach
Upcoming session: Jul 4 — Aug 22.
Commitment
6 weeks of study, 5-8 hours/week
About the Course
Do you have data and wonder what it can tell you?  Do you need a deeper understanding of the core ways in which machine learning can improve your business?  Do you want to be able to converse with specialists about anything from regression and classification to deep learning and recommender systems?

In this course, you will get hands-on experience with machine learning from a series of practical case-studies.  At the end of the first course you will have studied how to predict house prices based on house-level features, analyze sentiment from user reviews, retrieve documents of interest, recommend products, and search for images.  Through hands-on practice with these use cases, you will be able to apply machine learning methods in a wide range of domains.

This first course treats the machine learning method as a black box.  Using this abstraction, you will focus on understanding tasks of interest, matching these tasks to machine learning tools, and assessing the quality of the output. In subsequent courses, you will delve into the components of this black box by examining models and algorithms.  Together, these pieces form the machine learning pipeline, which you will use in developing intelligent applications.

Learning Outcomes:  By the end of this course, you will be able to:
   -Identify potential applications of machine learning in practice.  
   -Describe the core differences in analyses enabled by regression, classification, and clustering.
   -Select the appropriate machine learning task for a potential application.  
   -Apply regression, classification, clustering, retrieval, recommender systems, and deep learning.
   -Represent your data as features to serve as input to machine learning models. 
   -Assess the model quality in terms of relevant error metrics for each task.
   -Utilize a dataset to fit a model to analyze new data.
   -Build an end-to-end application that uses machine learning at its core.  
   -Implement these techniques in Python.
WEEK 1
Welcome
Machine learning is everywhere, but is often operating behind the scenes.
This introduction to the specialization provides you with insights into the power of machine learning, and the multitude of intelligent applications you personally will be able to develop and deploy upon completion.

We also discuss who we are, how we got here, and our view of the future of intelligent applications.
\item Slides presented in this module
\item Welcome to this course and specialization
\item Who we are
\item Machine learning is changing the world
\item Why a case study approach?
\item Specialization overview
\item How we got into ML
\item Who is this specialization for?
\item What you'll be able to do
\item The capstone and an example intelligent application
\item The future of intelligent applications
\item Getting Started with Python & iPython Notebooks
\item Open your notebook workspace to follow along
\item Starting an IPython Notebook
\item Creating variables in Python
\item Conditional statements and loops in Python
\item Creating functions and lambdas in Python
\item Getting started with SFrame and GraphLab Create
\item Starting GraphLab Create & loading an SFrame
\item Canvas for data visualization
\item Interacting with columns of an SFrame
\item Using .apply() for data transformation
WEEK 2
Regression: Predicting House Prices
This week you will build your first intelligent application that makes predictions from data.
We will explore this idea within the context of our first case study, predicting house prices, where you will create models that predict a continuous value (price) from input features (square footage, number of bedrooms and bathrooms,...).

This is just one of the many places where regression can be applied.Other applications range from predicting health outcomes in medicine, stock prices in finance, and power usage in high-performance computing, to analyzing which regulators are important for gene expression.

You will also examine how to analyze the performance of your predictive model and implement regression in practice using an iPython notebook.
\item Slides presented in this module
\item Predicting house prices: A case study in regression
\item What is the goal and how might you naively address it?
\item Linear Regression: A Model-Based Approach
\item Adding higher order effects
\item Evaluating overfitting via training/test split
\item Training/test curves
\item Adding other features
\item Other regression examples
\item Regression ML block diagram
Quiz · Regression
\item Open the iPython Notebook used in this lesson to follow along
\item Loading & exploring house sale data
\item Splitting the data into training and test sets
\item Learning a simple regression model to predict house prices from house size
\item Evaluating error (RMSE) of the simple model
\item Visualizing predictions of simple model with Matplotlib
\item Inspecting the model coefficients learned
\item Exploring other features of the data
\item Learning a model to predict house prices from more features
\item Applying learned models to predict price of an average house
\item Applying learned models to predict price of two fancy houses
\item Predicting house prices assignment
Quiz · Predicting house prices
WEEK 3
Classification: Analyzing Sentiment
How do you guess whether a person felt positively or negatively about an experience, just from a short review they wrote?
In our second case study, analyzing sentiment, you will create models that predict a class (positive/negative sentiment) from input features (text of the reviews, user profile information,...).This task is an example of classification, one of the most widely used areas of machine learning, with a broad array of applications, including ad targeting, spam detection, medical diagnosis and image classification.

You will analyze the accuracy of your classifier, implement an actual classifier in an iPython notebook, and take a first stab at a core piece of the intelligent application you will build and deploy in your capstone.
\item Slides presented in this module
\item Analyzing the sentiment of reviews: A case study in classification
\item What is an intelligent restaurant review system?
\item Examples of classification tasks
\item Linear classifiers
\item Decision boundaries
\item Training and evaluating a classifier
\item What's a good accuracy?
\item False positives, false negatives, and confusion matrices
\item Learning curves
\item Class probabilities
\item Classification ML block diagram
Quiz · Classification
\item Open the iPython Notebook used in this lesson to follow along
\item Loading & exploring product review data
\item Creating the word count vector
\item Exploring the most popular product
\item Defining which reviews have positive or negative sentiment
\item Training a sentiment classifier
\item Evaluating a classifier & the ROC curve
\item Applying model to find most positive & negative reviews for a product
\item Exploring the most positive & negative aspects of a product
\item Analyzing product sentiment assignment
Quiz · Analyzing product sentiment
WEEK 4
Clustering and Similarity: Retrieving Documents
A reader is interested in a specific news article and you want to find a similar articles to recommend. What is the right notion of similarity? How do I automatically search over documents to find the one that is most similar? How do I quantitatively represent the documents in the first place?
In this third case study, retrieving documents, you will examine various document representations and an algorithm to retrieve the most similar subset. You will also consider structured representations of the documents that automatically group articles by similarity (e.g., document topic).

You will actually build an intelligent document retrieval system for Wikipedia entries in an iPython notebook.
\item Slides presented in this module
\item Document retrieval: A case study in clustering and measuring similarity
\item What is the document retrieval task?
\item Word count representation for measuring similarity
\item Prioritizing important words with tf-idf
\item Calculating tf-idf vectors
\item Retrieving similar documents using nearest neighbor search
\item Clustering documents task overview
\item Clustering documents: An unsupervised learning task
\item k-means: A clustering algorithm
\item Other examples of clustering
\item Clustering and similarity ML block diagram
Quiz · Clustering and Similarity
\item Open the iPython Notebook used in this lesson to follow along
\item Loading & exploring Wikipedia data
\item Exploring word counts
\item Computing & exploring TF-IDFs
\item Computing distances between Wikipedia articles
\item Building & exploring a nearest neighbors model for Wikipedia articles
\item Examples of document retrieval in action
\item Retrieving Wikipedia articles assignment
Quiz · Retrieving Wikipedia articles
WEEK 5
Recommending Products
Ever wonder how Amazon forms its personalized product recommendations? How Netflix suggests movies to watch? How Pandora selects the next song to stream? How Facebook or LinkedIn finds people you might connect with? Underlying all of these technologies for personalized content is something called collaborative filtering.
You will learn how to build such a recommender system using a variety of techniques, and explore their tradeoffs.

One method we examine is matrix factorization, which learns features of users and products to form recommendations. In an iPython notebook, you will use these techniques to build a real song recommender system.
\item Slides presented in this module
\item Recommender systems overview
\item Where we see recommender systems in action
\item Building a recommender system via classification
\item Collaborative filtering: People who bought this also bought...
\item Effect of popular items
\item Normalizing co-occurrence matrices and leveraging purchase histories
\item The matrix completion task
\item Recommendations from known user/item features
\item Predictions in matrix form
\item Discovering hidden structure by matrix factorization
\item Bringing it all together: Featurized matrix factorization
\item A performance metric for recommender systems
\item Optimal recommenders
\item Precision-recall curves
\item Recommender systems ML block diagram
Quiz · Recommender Systems
\item Open the iPython Notebook used in this lesson to follow along
\item Loading and exploring song data
\item Creating & evaluating a popularity-based song recommender
\item Creating & evaluating a personalized song recommender
\item Using precision-recall to compare recommender models
\item Recommending songs assignment
Quiz · Recommending songs
WEEK 6
Deep Learning: Searching for Images
You’ve probably heard that Deep Learning is making news across the world as one of the most promising techniques in machine learning. Every industry is dedicating resources to unlock the deep learning potential, including for tasks such as image tagging, object recognition, speech recognition, and text analysis.
In our final case study, searching for images, you will learn how layers of neural networks provide very descriptive (non-linear) features that provide impressive performance in image classification and retrieval tasks. You will then construct deep features, a transfer learning technique that allows you to use deep learning very easily, even when you have little data to train the model.

Using iPhython notebooks, you will build an image classifier and an intelligent image retrieval system with deep learning.
\item Slides presented in this module
\item Searching for images: A case study in deep learning
\item What is a visual product recommender?
\item Learning very non-linear features with neural networks
\item Application of deep learning to computer vision
\item Deep learning performance
\item Demo of deep learning model on ImageNet data
\item Other examples of deep learning in computer vision
\item Challenges of deep learning
\item Deep Features
\item Deep learning ML block diagram
Quiz · Deep Learning
\item Open the iPython Notebook used in this lesson to follow along
\item Loading image data
\item Training & evaluating a classifier using raw image pixels
\item Training & evaluating a classifier using deep features
\item Open the iPython Notebook used in this lesson to follow along
\item Loading image data
\item Creating a nearest neighbors model for image retrieval
\item Querying the nearest neighbors model to retrieve images
\item Querying for the most similar images for car image
\item Displaying other example image retrievals with a Python lambda
\item Deep features for image classification & retrieval assignment
Quiz · Deep features for image retrieval
Closing Remarks
In the conclusion of the course, we will describe the final stage in turning our machine learning tools into a service: deployment.
We will also discuss some open challenges that the field of machine learning still faces, and where we think machine learning is heading. We conclude with an overview of what's in store for you in the rest of the specialization, and the amazing intelligent applications that are ahead for us as we evolve machine learning.

\item Slides presented in this module
\item You've made it!
\item Deploying an ML service
\item What happens after deployment?
\item Open challenges in ML
\item Where is ML going?
\item What's ahead in the specialization
\item Thank you!

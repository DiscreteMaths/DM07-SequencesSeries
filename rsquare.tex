
\documentclass[]{report}

\voffset=-1.5cm
\oddsidemargin=0.0cm
\textwidth = 480pt

%\usepackage{framed}
%\usepackage{subfiles}
%\usepackage{enumerate}
%\usepackage{graphics}
%\usepackage{newlfont}
%\usepackage{eurosym}
%\usepackage{amsmath,amsthm,amsfonts}
%\usepackage{amsmath}
%\usepackage{color}
%\usepackage{amssymb}
%\usepackage{multicol}
%\usepackage[dvipsnames]{xcolor}
%\usepackage{graphicx}
\begin{document}
	
\section*{The Coefficient of Determination}
\begin{itemize}
	\item 

The coefficient of determination $R^2$ is used in the context of statistical models whose main
purpose is the prediction of future outcomes on the basis of other related information. \item It is the
proportion of variability in a data set that is accounted for by the statistical model. \item It provides
a measure of how well future outcomes are likely to be predicted by the model.
$R^2$ is a statistic that will give some information about the goodness of t of a model.
\item  In
regression, the $R^2$ coefficient of determination is a statistical measure of how well the regression
line approximates the real data points. An $R^2$ of 1.0 indicates that the regression line perfectly
ts the data.
\item In the case of simple linear regression, the coefficient of determination is equivalent to the
squared value of the Pearson correlation coefficient. (Consider this to be co-incidental, rather
than a definition)
\end{itemize}
\subsection*{The Adjusted Coefficient of Determination}
\begin{itemize}
	\item Adjusted $R^2$ (often written as and pronounced "R bar squared") is a modication of $R^2$ that
	adjusts for the number of predictor terms in a model. Adjusted $R^2$ is used to compensate
	for the addition of variables to the model. As more independent variables are added to the
	regression model, unadjusted $R^2$ will generally increase but there will never be a decrease.
\item This
	will occur even when the additional variables do little to help explain the dependent variable.
\item To compensate for this, adjusted $R^2$ is corrected for the number of independent variables in
	the model, increases only if the new term improves the model more than would be expected
	by chance. If too many predictor variables are being used, this will be reflected in a reduced adjusted $R^2$. 
	\item The adjusted $R^2$ can be negative (unlikely, but not impossible), and will always be less than or equal to $R^2$.
\item The result is an adjusted $R^2$ than can go up or down depending on whether the addition of
	another variable adds or does not add to the explanatory power of the model. Adjusted $R^2$ will
	always be lower than unadjusted.
\item Adjusted R square is generally considered to be a more accurate goodness-of-t measure
	than R square. It has become standard practice to report the adjusted $R^2$, especially when
	there are multiple models presented with varying numbers of independent variables. 
\end{itemize}
\newpage
\section*{The Coefficient of Determination}
\begin{itemize}
	
	\item  The coefficient of determination R2 is used in the context of statistical models whose main purpose is
	the prediction of future outcomes on the basis of other related information.
	\item  It is the proportion of variability in a data set that is accounted for by the statistical model.
	\item  It provides a measure of how well future outcomes are likely to be predicted by the model. R2 is a
	statistic that will give some information about the goodness of
	t of a model.
	\item  In regression, the R2 coefficient of determination is a statistical measure of how well the regression line
	approximates the real data points. An R2 of 1.0 indicates that the regression line perfectly ts the data.
	\item  In the case of simple linear regression, the coefficient of determination is equivalent to the squared value
	of the Pearson correlation coefficient. (Consider this to be co-incidental, rather than a definition)
	The Adjusted Coefficient of Determination
	\item  Adjusted R2 (often written as and pronounced "R bar squared") is a modi
	cation of R2 that adjusts for the number of predictor terms in a model. Adjusted R2 is used to compensate
	for the addition of variables to the model. As more independent variables are added to the regression
	model, unadjusted R2 will generally increase but there will never be a decrease.
	\item  This will occur even when the additional variables do little to help explain the dependent variable.
	\item  To compensate for this, adjusted R2 is corrected for the number of independent variables in the model,
	increases only if the new term improves the model more than would be expected by chance. If too many
	predictor variables are being used, this will be re
	ected in a reduced adjusted R2.
	\item  The adjusted R2 can be negative (unlikely, but not impossible), and will always be less than or equal to
	R2.
	\item  The result is an adjusted R2 than can go up or down depending on whether the addition of another
	variable adds or does not add to the explanatory power of the model. Adjusted R2 will always be lower
	than unadjusted.
	\item  Adjusted R square is generally considered to be a more accurate goodness-of-
	t measure than R square. It has become standard practice to report the adjusted R2, especially when
	there are multiple models presented with varying numbers of independent variables.
\end{itemize}
%============================================%
\end{document}
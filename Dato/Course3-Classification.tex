Machine Learning: Classification
Current session: Jun 27 — Aug 22.
About the Course
Case Studies: Analyzing Sentiment & Loan Default Prediction

In our case study on analyzing sentiment, you will create models that predict a class (positive/negative sentiment) from input features (text of the reviews, user profile information,...).  In our second case study for this course, loan default prediction, you will tackle financial data, and predict when a loan is likely to be risky or safe for the bank. These tasks are an examples of classification, one of the most widely used areas of machine learning, with a broad array of applications, including ad targeting, spam detection, medical diagnosis and image classification. 

In this course, you will create classifiers that provide state-of-the-art performance on a variety of tasks.  You will become familiar with  the most successful techniques, which are most widely used in practice, including logistic regression, decision trees and boosting.  In addition, you will be able to design and implement the underlying algorithms that can learn these models at scale, using stochastic gradient ascent.  You will implement these technique on real-world, large-scale machine learning tasks.  You will also address significant tasks you will face in real-world applications of ML, including handling missing data and measuring precision and recall to evaluate a classifier.  This course is hands-on, action-packed, and full of visualizations and illustrations of how these techniques will behave on real data.  We've also included optional content in every module, covering advanced topics for those who want to go even deeper! 

Learning Objectives: By the end of this course, you will be able to:
\begin{itemize}
	\item \item Describe the input and output of a classification model.
	\item Tackle both binary and multiclass classification problems.
	\item Implement a logistic regression model for large\item scale classification.  
	\item Create a non\item linear model using decision trees.
	\item Improve the performance of any model using boosting.
	\item Scale your methods with stochastic gradient ascent.
	\item Describe the underlying decision boundaries.  
	\item Build a classification model to predict sentiment in a product review dataset.  
	\item Analyze financial data to predict loan defaults.
	\item Use techniques for handling missing data.
	\item Evaluate your models using precision\item recall metrics.
	\item Implement these techniques in Python (or in the language of your choice, though Python is highly recommended).
\end{itemize}
\newpage
\subsection{WEEK 1 - Welcome!}
Classification is one of the most widely used techniques in machine learning, with a broad array of applications, including sentiment analysis, ad targeting, spam detection, risk assessment, medical diagnosis and image classification. The core goal of classification is to predict a category or class y from some inputs x. Through this course, you will become familiar with the fundamental models and algorithms used in classification, as well as a number of core machine learning concepts. Rather than covering all aspects of classification, you will focus on a few core techniques, which are widely used in the real-world to get state-of-the-art performance. By following our hands-on approach, you will implement your own algorithms on multiple real-world tasks, and deeply grasp the core techniques needed to be successful with these approaches in practice. This introduction to the course provides you with an overview of the topics we will cover and the background knowledge and resources we assume you have.
\begin{itemize}
\item Slides presented in this module
\item Welcome to the classification course, a part of the Machine Learning Specialization
\item What is this course about?
\item Impact of classification
\item Course overview
\item Outline of first half of course
\item Outline of second half of course
\item Assumed background
\item Let's get started!
\item Reading: Software tools you'll need
\item Installing correct version of GraphLab Create
\end{itemize}
\newpage
\subsection{Linear Classifiers \& Logistic Regression}
Linear classifiers are amongst the most practical classification methods. For example, in our sentiment analysis case-study, a linear classifier associates a coefficient with the counts of each word in the sentence. In this module, you will become proficient in this type of representation. You will focus on a particularly useful type of linear classifier called logistic regression, which, in addition to allowing you to predict a class, provides a probability associated with the prediction. These probabilities are extremely useful, since they provide a degree of confidence in the predictions. In this module, you will also be able to construct features from categorical inputs, and to tackle classification problems with more than two class (multiclass problems). You will examine the results of these techniques on a real-world product sentiment analysis task.
\begin{itemize}
\item Slides presented in this module
\item Linear classifiers: A motivating example
\item Intuition behind linear classifiers
\item Decision boundaries
\item Linear classifier model
\item Effect of coefficient values on decision boundary
\item Using features of the inputs
\item Predicting class probabilities
\item Review of basics of probabilities
\item Review of basics of conditional probabilities
\item Using probabilities in classification
\item Predicting class probabilities with (generalized) linear models
\item The sigmoid (or logistic) link function
\item Logistic regression model
\item Effect of coefficient values on predicted probabilities
\item Overview of learning logistic regression models
\item Encoding categorical inputs
\item Multiclass classification with 1 versus all
\item Recap of logistic regression classifier
Quiz · Linear Classifiers \& Logistic Regression
\item Predicting sentiment from product reviews
Quiz · Predicting sentiment from product reviews
\end{itemize}
%=======================================%
\subsection{WEEK 2 - Learning Linear Classifiers}
Once familiar with linear classifiers and logistic regression, you can now dive in and write your first learning algorithm for classification. In particular, you will use gradient ascent to learn the coefficients of your classifier from data. You first will need to define the quality metric for these tasks using an approach called maximum likelihood estimation (MLE). You will also become familiar with a simple technique for selecting the step size for gradient ascent. An optional, advanced part of this module will cover the derivation of the gradient for logistic regression. You will implement your own learning algorithm for logistic regression from scratch, and use it to learn a sentiment analysis classifier.
\begin{itemize}
\item Slides presented in this module
\item Goal: Learning parameters of logistic regression
\item Intuition behind maximum likelihood estimation
\item Data likelihood
\item Finding best linear classifier with gradient ascent
\item Review of gradient ascent
\item Learning algorithm for logistic regression
\item Example of computing derivative for logistic regression
\item Interpreting derivative for logistic regression
\item Summary of gradient ascent for logistic regression
\item Choosing step size
\item Careful with step sizes that are too large
\item Rule of thumb for choosing step size
\item (VERY OPTIONAL) Deriving gradient of logistic regression: Log trick
\item (VERY OPTIONAL) Expressing the log-likelihood
\item (VERY OPTIONAL) Deriving probability y=-1 given x
\item (VERY OPTIONAL) Rewriting the log likelihood into a simpler form
\item (VERY OPTIONAL) Deriving gradient of log likelihood
\item Recap of learning logistic regression classifiers
Quiz · Learning Linear Classifiers
\item Implementing logistic regression from scratch
Quiz · Implementing logistic regression from scratch
\end{itemize}

\subsection{Overfitting \& Regularization in Logistic Regression}
As we saw in the regression course, overfitting is perhaps the most significant challenge you will face as you apply machine learning approaches in practice. This challenge can be particularly significant for logistic regression, as you will discover in this module, since we not only risk getting an overly complex decision boundary, but your classifier can also become overly confident about the probabilities it predicts. In this module, you will investigate overfitting in classification in significant detail, and obtain broad practical insights from some interesting visualizations of the classifiers' outputs. You will then add a regularization term to your optimization to mitigate overfitting. You will investigate both L2 regularization to penalize large coefficient values, and L1 regularization to obtain additional sparsity in the coefficients. Finally, you will modify your gradient ascent algorithm to learn regularized logistic regression classifiers. You will implement your own regularized logistic regression classifier from scratch, and investigate the impact of the L2 penalty on real-world sentiment analysis data.
\begin{itemize}
\item Slides presented in this module
\item Evaluating a classifier
\item Review of overfitting in regression
\item Overfitting in classification
\item Visualizing overfitting with high-degree polynomial features
\item Overfitting in classifiers leads to overconfident predictions
\item Visualizing overconfident predictions
\item (OPTIONAL) Another perspecting on overfitting in logistic regression
\item Penalizing large coefficients to mitigate overfitting
\item L2 regularized logistic regression
\item Visualizing effect of L2 regularization in logistic regression
\item Learning L2 regularized logistic regression with gradient ascent
\item Sparse logistic regression with L1 regularization
\item Recap of overfitting \& regularization in logistic regression
Quiz · Overfitting \& Regularization in Logistic Regression
\item Logistic Regression with L2 regularization
Quiz · Logistic Regression with L2 regularization
\end{itemize}
\newpage
\subsection{WEEK 3 -Decision Trees}
Along with linear classifiers, decision trees are amongst the most widely used classification techniques in the real world. This method is extremely intuitive, simple to implement and provides interpretable predictions. In this module, you will become familiar with the core decision trees representation. You will then design a simple, recursive greedy algorithm to learn decision trees from data. Finally, you will extend this approach to deal with continuous inputs, a fundamental requirement for practical problems. In this module, you will investigate a brand new case-study in the financial sector: predicting the risk associated with a bank loan. You will implement your own decision tree learning algorithm on real loan data.
\begin{itemize}
\item Slides presented in this module
\item Predicting loan defaults with decision trees
\item Intuition behind decision trees
\item Task of learning decision trees from data
\item Recursive greedy algorithm
\item Learning a decision stump
\item Selecting best feature to split on
\item When to stop recursing
\item Making predictions with decision trees
\item Multiclass classification with decision trees
\item Threshold splits for continuous inputs
\item (OPTIONAL) Picking the best threshold to split on
\item Visualizing decision boundaries
\item Recap of decision trees
Quiz · Decision Trees
\item Identifying safe loans with decision trees
Quiz · Identifying safe loans with decision trees
\item Implementing binary decision trees
Quiz · Implementing binary decision trees
\end{itemize}
\newpage
\subsection{WEEK 4 - Preventing Overfitting in Decision Trees}
Out of all machine learning techniques, decision trees are amongst the most prone to overfitting. No practical implementation is possible without including approaches that mitigate this challenge. In this module, through various visualizations and investigations, you will investigate why decision trees suffer from significant overfitting problems. Using the principle of Occam's razor, you will mitigate overfitting by learning simpler trees. At first, you will design algorithms that stop the learning process before the decision trees become overly complex. In an optional segment, you will design a very practical approach that learns an overly-complex tree, and then simplifies it with pruning. Your implementation will investigate the effect of these techniques on mitigating overfitting on our real-world loan data set.
\begin{itemize}
\item Slides presented in this module
\item A review of overfitting
\item Overfitting in decision trees
\item Principle of Occam's razor: Learning simpler decision trees
\item Early stopping in learning decision trees
\item (OPTIONAL) Motivating pruning
\item (OPTIONAL) Pruning decision trees to avoid overfitting
\item (OPTIONAL) Tree pruning algorithm
\item Recap of overfitting and regularization in decision trees
Quiz · Preventing Overfitting in Decision Trees
\item Decision Trees in Practice
Quiz · Decision Trees in Practice
\end{itemize}

\subsection{Handling Missing Data}
Real-world machine learning problems are fraught with missing data. That is, very often, some of the inputs are not observed for all data points. This challenge is very significant, happens in most cases, and needs to be addressed carefully to obtain great performance. And, this issue is rarely discussed in machine learning courses. In this module, you will tackle the missing data challenge head on. You will start with the two most basic techniques to convert a dataset with missing data into a clean dataset, namely skipping missing values and inputing missing values. In an advanced section, you will also design a modification of the decision tree learning algorithm that builds decisions about missing data right into the model. You will also explore these techniques in your real-data implementation.
\item Slides presented in this module
\item Challenge of missing data
\item Strategy 1: Purification by skipping missing data
\item Strategy 2: Purification by imputing missing data
\item Modifying decision trees to handle missing data
\item Feature split selection with missing data
\item Recap of handling missing data
Quiz · Handling Missing Data
WEEK 5
Boosting
One of the most exciting theoretical questions that have been asked about machine learning is whether simple classifiers can be combined into a highly accurate ensemble. This question lead to the developing of boosting, one of the most important and practical techniques in machine learning today. This simple approach can boost the accuracy of any classifier, and is widely used in practice, e.g., it's used by more than half of the teams who win the Kaggle machine learning competitions. In this module, you will first define the ensemble classifier, where multiple models vote on the best prediction. You will then explore a boosting algorithm called AdaBoost, which provides a great approach for boosting classifiers. Through visualizations, you will become familiar with many of the practical aspects of this techniques. You will create your very own implementation of AdaBoost, from scratch, and use it to boost the performance of your loan risk predictor on real data.
\item Slides presented in this module
\item The boosting question
\item Ensemble classifiers
\item Boosting
\item AdaBoost overview
\item Weighted error
\item Computing coefficient of each ensemble component
\item Reweighing data to focus on mistakes
\item Normalizing weights
\item Example of AdaBoost in action
\item Learning boosted decision stumps with AdaBoost
\item Exploring Ensemble Methods
Quiz · Exploring Ensemble Methods
\item The Boosting Theorem
\item Overfitting in boosting
\item Ensemble methods, impact of boosting & quick recap
Quiz · Boosting
\item Boosting a decision stump
Quiz · Boosting a decision stump
WEEK 6
Precision-Recall
In many real-world settings, accuracy or error are not the best quality metrics for classification. You will explore a case-study that significantly highlights this issue: using sentiment analysis to display positive reviews on a restaurant website. Instead of accuracy, you will define two metrics: precision and recall, which are widely used in real-world applications to measure the quality of classifiers. You will explore how the probabilities output by your classifier can be used to trade-off precision with recall, and dive into this spectrum, using precision-recall curves. In your hands-on implementation, you will compute these metrics with your learned classifier on real-world sentiment analysis data.
\item Slides presented in this module
\item Case-study where accuracy is not best metric for classification
\item What is good performance for a classifier?
\item Precision: Fraction of positive predictions that are actually positive
\item Recall: Fraction of positive data predicted to be positive
\item Precision-recall extremes
\item Trading off precision and recall
\item Precision-recall curve
\item Recap of precision-recall
Quiz · Precision-Recall
\item Exploring precision and recall
Quiz · Exploring precision and recall
WEEK 7
Scaling to Huge Datasets & Online Learning
With the advent of the internet, the growth of social media, and the embedding of sensors in the world, the magnitudes of data that our machine learning algorithms must handle have grown tremendously over the last decade. This effect is sometimes called "Big Data". Thus, our learning algorithms must scale to bigger and bigger datasets. In this module, you will develop a small modification of gradient ascent called stochastic gradient, which provides significant speedups in the running time of our algorithms. This simple change can drastically improve scaling, but makes the algorithm less stable and harder to use in practice. In this module, you will investigate the practical techniques needed to make stochastic gradient viable, and to thus to obtain learning algorithms that scale to huge datasets. You will also address a new kind of machine learning problem, online learning, where the data streams in over time, and we must learn the coefficients as the data arrives. This task can also be solved with stochastic gradient. You will implement your very own stochastic gradient ascent algorithm for logistic regression from scratch, and evaluate it on sentiment analysis data.
\item Slides presented in this module
\item Gradient ascent won't scale to today's huge datasets
\item Timeline of scalable machine learning & stochastic gradient
\item Why gradient ascent won't scale
\item Stochastic gradient: Learning one data point at a time
\item Comparing gradient to stochastic gradient
\item Why would stochastic gradient ever work?
\item Convergence paths
\item Shuffle data before running stochastic gradient
\item Choosing step size
\item Don't trust last coefficients
\item (OPTIONAL) Learning from batches of data
\item (OPTIONAL) Measuring convergence
\item (OPTIONAL) Adding regularization
\item The online learning task
\item Using stochastic gradient for online learning
\item Scaling to huge datasets through parallelization & module recap
Quiz · Scaling to Huge Datasets & Online Learning
\item Training Logistic Regression via Stochastic Gradient Ascent
Quiz · Training Logistic Regression via Stochastic Gradient Ascent
Hide Details

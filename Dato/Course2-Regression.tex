Machine Learning: Regression
Current session: Jun 27 — Aug 15.
Commitment
6 weeks of study, 5-8 hours/week
About the Course
Case Study - Predicting Housing Prices

%======================================================================================%
In our first case study, predicting house prices, you will create models that predict a continuous value (price) from input features (square footage, number of bedrooms and bathrooms,...).  This is just one of the many places where regression can be applied.  Other applications range from predicting health outcomes in medicine, stock prices in finance, and power usage in high-performance computing, to analyzing which regulators are important for gene expression.

In this course, you will explore regularized linear regression models for the task of prediction and feature selection.  You will be able to handle very large sets of features and select between models of various complexity.  You will also analyze the impact of aspects of your data -- such as outliers -- on your selected models and predictions.  To fit these models, you will implement optimization algorithms that scale to large datasets.
%======================================================================================%
Learning Outcomes:  By the end of this course, you will be able to:
\begin{itemize}
   \item Describe the input and output of a regression model.
   \item Compare and contrast bias and variance when modeling data.
   \item Estimate model parameters using optimization algorithms.
   \item Tune parameters with cross validation.
   \item Analyze the performance of the model.
   \item Describe the notion of sparsity and how LASSO leads to sparse solutions.
   \item Deploy methods to select between models.
   \item Exploit the model to form predictions. 
   \item Build a regression model to predict prices using a housing dataset.
   \item Implement these techniques in Python.
\end{itemize}
%======================================================================================%
WEEK 1
Welcome
Regression is one of the most important and broadly used machine learning and statistics tools out there. It allows you to make predictions from data by learning the relationship between features of your data and some observed, continuous-valued response. Regression is used in a massive number of applications ranging from predicting stock prices to understanding gene regulatory networks.
This introduction to the course provides you with an overview of the topics we will cover and the background knowledge and resources we assume you have.

\item Slides presented in this module
\item Welcome!
\item What is the course about?
\item Outlining the first half of the course
\item Outlining the second half of the course
\item Assumed background
\item Reading: Software tools you'll need
Simple Linear Regression
Our course starts from the most basic regression model: Just fitting a line to data. This simple model for forming predictions from a single, univariate feature of the data is appropriately called "simple linear regression".
In this module, we describe the high-level regression task and then specialize these concepts to the simple linear regression case. You will learn how to formulate a simple regression model and fit the model to data using both a closed-form solution as well as an iterative optimization algorithm called gradient descent. Based on this fitted function, you will interpret the estimated model parameters and form predictions. You will also analyze the sensitivity of your fit to outlying observations.
%======================================================================================%
You will examine all of these concepts in the context of a case study of predicting house prices from the square feet of the house.
\begin{itemize}
\item Slides presented in this module
\item A case study in predicting house prices
\item Regression fundamentals: data & model
\item Regression fundamentals: the task
\item Regression ML block diagram
\item The simple linear regression model
\item The cost of using a given line
\item Using the fitted line
\item Interpreting the fitted line
\item Defining our least squares optimization objective
\item Finding maxima or minima analytically
\item Maximizing a 1d function: a worked example
\item Finding the max via hill climbing
\item Finding the min via hill descent
\item Choosing stepsize and convergence criteria
\item Gradients: derivatives in multiple dimensions
\item Gradient descent: multidimensional hill descent
\item Computing the gradient of RSS
\item Approach 1: closed-form solution
\item Optional reading: worked-out example for closed-form solution
\item Approach 2: gradient descent
\item Optional reading: worked-out example for gradient descent
\item Comparing the approaches
\item Download notebooks to follow along
\item Influence of high leverage points: exploring the data
\item Influence of high leverage points: removing Center City
\item Influence of high leverage points: removing high-end towns
\item Asymmetric cost functions
\item A brief recap
\end{itemize}
%============================================================================================%
Quiz · Simple Linear Regression
\item Reading: Fitting a simple linear regression model on housing data
Quiz · Fitting a simple linear regression model on housing data

%======================================================================================%
\section{WEEK 2:Multiple Regression}
The next step in moving beyond simple linear regression is to consider "multiple regression" where multiple features of the data are used to form predictions.
More specifically, in this module, you will learn how to build models of more complex relationship between a single variable (e.g., 'square feet') and the observed response (like 'house sales price'). This includes things like fitting a polynomial to your data, or capturing seasonal changes in the response value. You will also learn how to incorporate multiple input variables (e.g., 'square feet', '# bedrooms', '# bathrooms'). You will then be able to describe how all of these models can still be cast within the linear regression framework, but now using multiple "features". Within this multiple regression framework, you will fit models to data, interpret estimated coefficients, and form predictions.
%======================================================================================%
Here, you will also implement a gradient descent algorithm for fitting a multiple regression model.
\begin{itemize}
\item Slides presented in this module
\item Multiple regression intro
\item Polynomial regression
\item Modeling seasonality
\item Where we see seasonality
\item Regression with general features of 1 input
\item Motivating the use of multiple inputs
\item Defining notation
\item Regression with features of multiple inputs
\item Interpreting the multiple regression fit
\item Optional reading: review of matrix algebra
\item Rewriting the single observation model in vector notation
\item Rewriting the model for all observations in matrix notation
\item Computing the cost of a D-dimensional curve
\item Computing the gradient of RSS
\item Approach 1: closed-form solution
\item Discussing the closed-form solution
\item Approach 2: gradient descent
\item Feature-by-feature update
\item Algorithmic summary of gradient descent approach
\item A brief recap
\end{itemize}
Quiz · Multiple Regression
\item Reading: Exploring different multiple regression models for house price prediction
Quiz · Exploring different multiple regression models for house price prediction
\item Numpy tutorial
\item Reading: Implementing gradient descent for multiple regression
Quiz · Implementing gradient descent for multiple regression

%========================================================================%
\section*{WEEK 3: Assessing Performance}
Having learned about linear regression models and algorithms for estimating the parameters of such models, you are now ready to assess how well your considered method should perform in predicting new data. You are also ready to select amongst possible models to choose the best performing.
This module is all about these important topics of model selection and assessment. You will examine both theoretical and practical aspects of such analyses. You will first explore the concept of measuring the "loss" of your predictions, and use this to define training, test, and generalization error. For these measures of error, you will analyze how they vary with model complexity and how they might be utilized to form a valid assessment of predictive performance. This leads directly to an important conversation about the bias-variance tradeoff, which is fundamental to machine learning. Finally, you will devise a method to first select amongst models and then assess the performance of the selected model.

The concepts described in this module are key to all machine learning problems, well-beyond the regression setting addressed in this course.
\begin{itemize}
\item Slides presented in this module
\item Assessing performance intro
\item What do we mean by "loss"?
\item Training error: assessing loss on the training set
\item Generalization error: what we really want
\item Test error: what we can actually compute
\item Defining overfitting
\item Training/test split
\item Irreducible error and bias
\item Variance and the bias-variance tradeoff
\item Error vs. amount of data
\item Formally defining the 3 sources of error
\item Formally deriving why 3 sources of error
\item Training/validation/test split for model selection, fitting, and assessment
\item A brief recap
\end{itemize}

%===================================================================%
Quiz · Assessing Performance
\item Reading: Exploring the bias-variance tradeoff
Quiz · Exploring the bias-variance tradeoff
%===================================================================%
\section*{WEEK 4: Ridge Regression}
You have examined how the performance of a model varies with increasing model complexity, and can describe the potential pitfall of complex models becoming overfit to the training data. In this module, you will explore a very simple, but extremely effective technique for automatically coping with this issue. This method is called "ridge regression". You start out with a complex model, but now fit the model in a manner that not only incorporates a measure of fit to the training data, but also a term that biases the solution away from overfitted functions. To this end, you will explore symptoms of overfitted functions and use this to define a quantitative measure to use in your revised optimization objective. You will derive both a closed-form and gradient descent algorithm for fitting the ridge regression objective; these forms are small modifications from the original algorithms you derived for multiple regression. To select the strength of the bias away from overfitting, you will explore a general-purpose method called "cross validation".
You will implement both cross-validation and gradient descent to fit a ridge regression model and select the regularization constant.
\begin{itemize}
\item Slides presented in this module
\item Symptoms of overfitting in polynomial regression
\item Download the notebook and follow along
\item Overfitting demo
\item Overfitting for more general multiple regression models
\item Balancing fit and magnitude of coefficients
\item The resulting ridge objective and its extreme solutions
\item How ridge regression balances bias and variance
\item Download the notebook and follow along
\item Ridge regression demo
\item The ridge coefficient path
\item Computing the gradient of the ridge objective
\item Approach 1: closed-form solution
\item Discussing the closed-form solution
\item Approach 2: gradient descent
\item Selecting tuning parameters via cross validation
\item K-fold cross validation
\item How to handle the intercept
\item A brief recap
\end{itemize}
%---------------------------------------------------------------%
Quiz · Ridge Regression
\item Reading: Observing effects of L2 penalty in polynomial regression
Quiz · Observing effects of L2 penalty in polynomial regression
\item Reading: Implementing ridge regression via gradient descent
Quiz · Implementing ridge regression via gradient descent
WEEK 5
Feature Selection & Lasso
A fundamental machine learning task is to select amongst a set of features to include in a model. In this module, you will explore this idea in the context of multiple regression, and describe how such feature selection is important for both interpretability and efficiency of forming predictions.
To start, you will examine methods that search over an enumeration of models including different subsets of features. You will analyze both exhaustive search and greedy algorithms. Then, instead of an explicit enumeration, we turn to Lasso regression, which implicitly performs feature selection in a manner akin to ridge regression: A complex model is fit based on a measure of fit to the training data plus a measure of overfitting different than that used in ridge. This lasso method has had impact in numerous applied domains, and the ideas behind the method have fundamentally changed machine learning and statistics. You will also implement a coordinate descent algorithm for fitting a Lasso model.

Coordinate descent is another, general, optimization technique, which is useful in many areas of machine learning.
\begin{itemize}
\item Slides presented in this module
\item The feature selection task
\item All subsets
\item Complexity of all subsets
\item Greedy algorithms
\item Complexity of the greedy forward stepwise algorithm
\item Can we use regularization for feature selection?
\item Thresholding ridge coefficients?
\item The lasso objective and its coefficient path
\item Visualizing the ridge cost
\item Visualizing the ridge solution
\item Visualizing the lasso cost and solution
\item Download the notebook and follow along
\item Lasso demo
\item What makes the lasso objective different
\item Coordinate descent
\item Normalizing features
\item Coordinate descent for least squares regression (normalized features)
\item Coordinate descent for lasso (normalized features)
\item Assessing convergence and other lasso solvers
\item Coordinate descent for lasso (unnormalized features)
\item Deriving the lasso coordinate descent update
\item Choosing the penalty strength and other practical issues with lasso
\item A brief recap
\end{itemize}
%======================================================================================%
Quiz · Feature Selection and Lasso
\item Reading: Using LASSO to select features
Quiz · Using LASSO to select features
\item Reading: Implementing LASSO using coordinate descent
Quiz · Implementing LASSO using coordinate descent

\section{WEEK 6 :Nearest Neighbors & Kernel Regression}
Up to this point, we have focused on methods that fit parametric functions---like polynomials and hyperplanes---to the entire dataset. In this module, we instead turn our attention to a class of "nonparametric" methods. These methods allow the complexity of the model to increase as more data are observed, and result in fits that adapt locally to the observations.
We start by considering the simple and intuitive example of nonparametric methods, nearest neighbor regression: The prediction for a query point is based on the outputs of the most related observations in the training set. This approach is extremely simple, but can provide excellent predictions, especially for large datasets. You will deploy algorithms to search for the nearest neighbors and form predictions based on the discovered neighbors. Building on this idea, we turn to kernel regression. Instead of forming predictions based on a small set of neighboring observations, kernel regression uses all observations in the dataset, but the impact of these observations on the predicted value is weighted by their similarity to the query point. You will analyze the theoretical performance of these methods in the limit of infinite training data, and explore the scenarios in which these methods work well versus struggle. You will also implement these techniques and observe their practical behavior.
\begin{itemize}
\item Slides presented in this module
\item Limitations of parametric regression
\item 1-Nearest neighbor regression approach
\item Distance metrics
\item 1-Nearest neighbor algorithm
\item k-Nearest neighbors regression
\item k-Nearest neighbors in practice
\item Weighted k-nearest neighbors
\item From weighted k-NN to kernel regression
\item Global fits of parametric models vs. local fits of kernel regression
\item Performance of NN as amount of data grows
\item Issues with high-dimensions, data scarcity, and computational complexity
\item k-NN for classification
\item A brief recap
\end{itemize}
Quiz · Nearest Neighbors & Kernel Regression
\item Reading: Predicting house prices using k-nearest neighbors regression
Quiz · Predicting house prices using k-nearest neighbors regression

%=======================================================================================%
\section{Closing Remarks}
In the conclusion of the course, we will recap what we have covered. This represents both techniques specific to regression, as well as foundational machine learning concepts that will appear throughout the specialization. We also briefly discuss some important regression techniques we did not cover in this course.
We conclude with an overview of what's in store for you in the rest of the specialization.

\begin{itemize}
\item Slides presented in this module
\item Simple and multiple regression
\item Assessing performance and ridge regression
\item Feature selection, lasso, and nearest neighbor regression
\item What we covered and what we didn't cover
\item What's ahead in the ML specialization
\end{itemize}
\end{document}
%======================================================================================%

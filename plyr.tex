% - http://www.cookbook-r.com/Manipulating_data/Summarizing_data/
% - http://www.inside-r.org/packages/cran/plyr/docs/ddply
% - http://www.computerworld.com/s/article/9243391/4_data_wrangling_tasks_in_R_for_advanced_beginners?taxonomyId=9&pageNumber=1
% - http://seananderson.ca/courses/12-plyr/plyr_2012.pdf

%----------------------------------------------------------------------------------------------------%

\subsection{plyr basics}
plyr builds on the built-in apply functions by giving you control over the input and
output formats and keeping the syntax consistent across all variations. It also adds
some niceties like error processing, parallel processing, and progress bars.

The basic format is 2 letters followed by ply(). The first letter refers to the format
in and the second to the format out.

The 3 main letters are:
\begin{itemize}
\item[d] data frame
\item[a]array (includes matrices)
\item[l] = list
\end{itemize}

%-------------------------------------------------------------------------------%

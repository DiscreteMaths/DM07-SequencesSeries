Binary Classification is the task of classifying the members of a given set of objects into two groups on the basis
if them having a particular set of characteristics.

ensitivity and specificity are statistical measures of the performance of a binary classification test, also known in statistics as classification function. Sensitivity (also called the true positive rate, or the recall rate in some fields) measures the proportion of actual positives which are correctly identified as such (e.g., the percentage of sick people who are correctly identified as having the condition), and is complementary to the false negative rate. Specificity (sometimes called the true negative rate) measures the proportion of negatives which are correctly identified as such (e.g., the percentage of healthy people who are correctly identified as not having the condition), and is complementary to the false positive rate.
A perfect predictor would be described as 100% sensitive (e.g., all sick are identified as sick) and 100% specific (e.g., all healthy are identified as healthy); however, theoretically any predictor will possess a minimum error bound known as the Bayes error rate.
For any test, there is usually a trade-off between the measures. For instance, in an airport security setting in which one is testing for potential threats to safety, scanners may be set to trigger on low-risk items like belt buckles and keys (low specificity), in order to reduce the risk of missing objects that do pose a threat to the aircraft and those aboard (high sensitivity). This trade-off can be represented graphically as a receiver operating characteristic curve.

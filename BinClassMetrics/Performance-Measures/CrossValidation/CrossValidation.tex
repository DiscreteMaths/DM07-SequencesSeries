\subsection{Cross Validation}
%-------------------------------------------------------------------------------------%

The confusion
table is a table in which the rows are the observed categories of the dependent and
the columns are the predicted categories. When prediction is perfect all cases will lie on the
diagonal. The percentage of cases on the diagonal is the percentage of correct classifications. 
The cross validated set of data is a more honest presentation of the power of the
discriminant function than that provided by the original classifications and often produces
a poorer outcome. The cross validation is often termed a ‘jack-knife’ classification, in that
it successively classifies \textbf{all cases but one} to develop a discriminant function and then
categorizes the case that was left out. This process is repeated with each case left out in
turn.This is known as leave-1-out cross validation. 

This cross validation produces a more reliable function. The argument behind it is that
one should not use the case you are trying to predict as part of the categorization process.


%-----------------------------------------------------------------------------------%
\subsection{Error Rates}

We can evaluate error rates by means of a training sample (to construct the discrimination rule) and a test sample.


An optimistic error rate is obtained by reclassifying the training data. (In the \textbf{\textit{training data}} sets, how many cases were misclassified). This is known as the \textbf{apparent error rate}.


The apparent error rate is obtained by using in the training set to estimate
the error rates. It can be severely optimistically biased, particularly for complex classifiers, and in the presence of over-fitted models.


If an independent test sample is used for classifying, we arrive at the  \textbf{true error rate}.The true error rate (or conditional error rate) of a classifier is the expected
probability of misclassifying a randomly selected pattern.
It is the error rate of an infinitely large test set drawn from the same distribution as the training data.



%---------------------------------------------------------------------------------------%

\subsection{Misclassification Cost}

As in all statistical procedures it is helpful to use diagnostic procedures to asses the efficacy of the discriminant analysis. We use \textbf{cross-validation} to assess the classification probability.
Typically you are going to have some prior rule as to what is an \textbf{acceptable misclassification rate}.

Those rules might involve things like, ``what is the cost of misclassification?" Consider a medical study where you might be able to diagnose cancer.

There are really two alternative costs. The cost of misclassifying someone as having cancer when they don't.
This could cause a certain amount of emotional grief. Additionally there would be the substantial cost of unnecessary treatment.

There is also the alternative cost of misclassifying someone as not having cancer when in fact they do have it.

A good classification procedure should
 \begin{itemize}
 \item result in few misclassifications
 \item take \textbf{\textit{prior probabilities of occurrence}} into account
 \item consider the cost of misclassification
 \end{itemize}
 
For example, suppose there tend to be more financially sound firms than bankrupt
firm. If we really believe that the prior probability of a financially
distressed and ultimately bankrupted firm is very small, then one should
classify a randomly selected firm as non-bankrupt unless the data
overwhelmingly favor bankruptcy.



There are two costs associated with discriminant analysis classification: The true misclassification cost per class, and the expected misclassification cost (ECM) per observation.

Suppose there we have a binary classification system, with two classes: class 1 and class 2.
Suppose that classifying a class 1 object as belonging to class 2 represents a more serious error than classifying a class 2 object as belonging to class 1. There would an assignable cost to each error.
$c(i|j)$ is the cost of classifying an observation into class $j$ if its true class is $i$.
The costs of misclassification can be defined by a cost matrix.

\begin{tabular}{|c|c|c|}
  \hline
  % after \\: \hline or \cline{col1-col2} \cline{col3-col4} ...
  & Predicted & Predicted \\
   & Class 1 & Class 2 \\  \hline
  Class 1 & 0 & $c(2|1)$  \\
  Class 2 & $c(1|2)$ & 0 \\
  \hline
\end{tabular}




\subsection{Expected cost of misclassification (ECM)}
Let $p_1$ and $p_2$ be the prior probability of class 1 and class 2 respectively.
Necessarily $p_1$ + $p_2$ = 1.

The conditional probability of classifying an object as class 1 when it is in fact from
class 2 is denoted $p(1|2)$.
Similarly the conditional probability of classifying an object as class 2 when it is in
fact from class 1 is denoted $p(2|1)$.

\[ECM = c(2|1)p(2|1)p_1 + c(1|2)p(1|2)p_2\]
(In other words: the sum of the cost of misclassification times the (joint) probability of that misclassification.

A reasonable classification rule should have ECM as small as possible.







\end{document}

\documentclass[12pt]{article}

%opening
\usepackage{framed}
\usepackage{amsmath}
\usepackage{amssymb}
\usepackage{graphicx}
\begin{document}


\section*{Binary Classification}

\subsection*{Binary Classification}
\subsubsection*{Defining true/false positives}
In general, Positive = identified and negative = rejected. Therefore:

\begin{itemize}
\item[TN] True negative = correctly rejected
\item[FP] False positive = incorrectly identified
\item[FN] False negative = incorrectly rejected
\item[TP] True positive = correctly identified
\end{itemize}
\subsubsection*{Medical testing example}
\begin{itemize}
\item True positive = Sick people correctly diagnosed as sick

\item False positive= Healthy people incorrectly identified as sick

\item True negative = Healthy people correctly identified as healthy

\item False negative = Sick people incorrectly identified as healthy.
\end{itemize}
\newpage
\subsection*{Definitions}
\textbf{Accuracy Rate}\\
The accuracy rate calculates the proportion ofobservations being allocated to the \textbf{correct} group by the predictive model. It is calculated as follows:
\[ \frac{
\mbox{Number of Correct Classifications }}{\mbox{Total Number of Classifications }} \]

\[ = \frac{TP + TN}{TP+FP+TN+FN}\]


\noindent \textbf{Misclassification Rate}\\
The misclassification rate calculates the proportion ofobservations being allocated to the \textbf{incorrect} group by the predictive model. It is calculated as follows:
\[ \frac{
\mbox{Number of Incorrect Classifications }}{\mbox{Total Number of Classifications }} \]

\[ = \frac{FP + FN}{TP+FP+TN+FN}\]


\end{document}

\documentclass[]{article}
\usepackage{framed}
\usepackage{amsmath}
\usepackage{amssymb}
\usepackage{multicol}
%opening

\begin{document}
\subsection*{Sequences, Series and Proof
by Induction}

%---------------------------------------------- %
\begin{verbatim}
Summary
Sequences; Proof by induction; Series and aha sigma notation.
References: Epp Sections 4.1. 4.2, 4.3` 44, 81, 82 nr MSLB Secnizm 3.1.
\end{verbatim}
\subsection{Sequences}

A sequence is simply a list as, for example
\begin{itemize}
\item[(a)] $2,5,8,11,14, ldots,$
\item[(b)] $5,0.5,0.05,0.005,0.0005,...$
\item[(c)] $0,1,1,2,3,5,8,13,21,...$
\end{itemize}
%---------------------------------------------- %
\begin{verbatim}
Formally. a sequence is a function from the set 2+ imo   The first term in
nhs sequence is called the initial term and is the image of 1, the second term is the
image cf 2, che chird is the image of 3, and so nm. 

We usually denote aha terms of
nhe sequence by a letter with a subscript. thus
uk 1 u2,u;, ....

\end{verbatim}
For example, in the sequence (a) given above, the initial term is $u_l = 2$, then $u_2 = 5$,
$u_3 = 8$, and so on.


Sequences are important because they arise naturally in a wide variety of practical situations, whenever a process is repeated and the result recorded.

%---------------------------------------------- %
\begin{verbatim}
 \Vhc¤ the
process is random as, for example, when the air uemperaturc is recorded at a weather
station or a die is rolled, there is nc way of predicting for certain what the next
term of a sequence will be, no matter how many earlier terms we have knowledge
of, There are processes, however, nhar give rise Lo sequences where the terms fall
into a. pattern as, for example; when the vakue of a sum of money invested at. a fixed
rate of compound interest is calculated at regular intervals. It is this latter type of
sequence, where we can continue nhe sequence when we know the pzmem and the
Fast few terms, that concerns us on this course. In this section, 0ur objective is mu
had a way of expressing the relationship between the terms of this kind of sequence.
\end{verbatim}
%---------------------------------------------- %
\begin{verbatim}
To be able to continue che inzeuded sequence, you must be given sufiiciem
data E0 be sure of the rclanicmship between its terms; as for example can be seen
by considering the sequence 1,2,4,.A .. (There: are at least uw logical ways of
continuing this sequence.)
\end{verbatim}

We have enough terms $u_f$ the sequence (a) 2,5,8, 11, 14, .. f, to convince us that
each term is found by adding 3 to the preceding berm. 

%---------------------------------------------- %
\begin{verbatim}
S0 c-he terms are calculaned
successively by the rules:
ul = 2;
uz Z ui + 3 : 5.
243 : ug + 3 : 8, ..,.
VVe can express this relationship between the terms in general by 14,,+; : 1%+3, for
all Tl E 2+. This is called the recurrence relation for this sequence. A sequence
for which the recurrence relation is of the form un+1 = un + d, where dis a cnnshanll
is known as an arithmetic progression (A.P.).
\end{verbatim}
%---------------------------------------------- %
\begin{verbatim}
In the sequence (b] 5,0.5, [}.05,0.005,0r00O5, ..., we obnain each term by multi-
plying the preceding Lerm by 0.1. This time, the terms are calculated successively
by the rules:
ur = 5;
ug = (0.1)u1 Z 0.5,
ug = [0.1] ug : 0.05, ....
\end{verbatim}
%---------------------------------------------- %
\begin{verbatim}
The recurrence relation for this sequence is un.}; : (O.l)u,.,, for all n E ZU'. A
sequence fur which the recurrence relation is ofthe form 14,1+; : run, where r is a
constant, is called a geometric progression (G.P.).
\end{verbatim}
%---------------------------------------------- %
\begin{verbatim}
The sequence (c}, given at Lhe beginning of She subsection, is known as the
Fibonacci sequence. The terms are called Fibonacci numbers and we shall
denoce them by F¤}F;,Fg,   (note that it is customary to start this sequence at
term O instead of term I). The sequence has so many interesting properties that
it has fascinated mathematicians for centuries. Recently a number 0f applicanions
ha.“v'e been found 1:0 computer science.
\end{verbatim}

Careful consideration of the Fibonacci sequence tells us that each term is the
sum of the previous two. So, starting from the initial terms $F_0 = 0, F_1 = 1$, the
terms are calculated successively by the rules:
%---------------------------------------------- %
\begin{verbatim}
F2 = F0+F1l;'O+1: 1),
Fa = F1+F2(:]+1:2),
F4 = Fz+Fs(=1+2=3).
F5 = F5-!-F4(=2·i·3=5), ....
\end{verbatim}
%---------------------------------------------- %
\begin{verbatim}
The recurrence relation is FM.; = Farr; + Fm where n 2 G, Notice that this time
we need knowledge uf two initial zermsr Fg and F1, in order no use nhe recurrence
relation to calculate successive terms.
\end{verbatim}
\subsection*{Proof by induction}


A technique that is often useful in proving results “‘for all positive integers n" is
called the Principle of Induction. It is based on the following fundamental
property of the integers.

%---------------------------------------------- %
\begin{verbatim}
Suppose that S is a subset of 2+ and than we have the following information
about 5:
(i) 1 E S;
(ii} whenever the integers 1, 2, . . ,,k 6 S, nhen k +1 6 S also.

\end{verbatim}
%---------------------------------------------- %
\begin{verbatim}
Then we may conclude that S = ZL?
To sec why this is true) we note Flrst that L E S by fl), and since 1 E S, than
2 6 S by (ii). But since 1, 2 6 S: then 3 E S by (ii) again; similarly, since 1, 2,3 6 S,
than 4 E S by (ii) . .. and so cn. Thus the two conditions together show that 2* Q S.
But. we are wld that S Q 2+ and hence S : 2+.
\end{verbatim}
%---------------------------------------------- %
\begin{verbatim}
New suppose that we wish to prove that a certain result is true "f0r all rz E 2+7
Lat S be the subset of 2+ for which the result holds. We can prove that. S : Zi
by showing that conditions   and {ii) above are satisfled by S. 
\end{verbatim}
%---------------------------------------------- %
We can do this if
we can establish the following THREE steps.
\textbf{Base case}
\begin{verbatim}
 Give a verification that the result is true when n = l so that 1 E S.
Induction hypothesis We suppose that the result is true for all the integers
1,2 .... ,k (for some integer lc 2 1).
\end{verbatim}
%---------------------------------------------- %
\textbf{Induction step}
\begin{verbatim}
 Using the hypothesis that, the result is true when n : 1, 2, .... k,
we prove thai the result also holds when n : k + 1.
\end{verbatim}
%---------------------------------------------- %
\begin{verbatim}
Example 2.1 Consider the sequence 2,5,8,11, 14, .... We saw above that the re-
currence relation for this sequence is M,-.+1 = ur. + 3. Sc starting from the initial
term ul = 2: we can calculate succesively:
ug : ul-l-3:11;-%-3xI
ua Z uz-?-3:u;+3+3:u1+3x2
uq : 213+3:ul +3+3+3:12;+3X 3,
amd it would be reasonable to guess that a formula that would give us Lhe value of
uu directly in terms of n mighc be
un :1l; +3(n—-}.}:`Z+3(n- I) :311- 1:
for all n E 2+. 
\end{verbatim}
%---------------------------------------------- %
We can use the Principle of Induction to prove that this guess is
correct.

\begin{verbatim}

Base case The formula is correct when n : l, since 3 (1) - l : 2 : ui.
Induction hypothesis Suppose that un : 3n — I is true for in = 1,2,3, . . ,,k.
Thus in particular we know that uk : 3}: —- l.
Induction step We prove that ur, : 3n — 1 is also true when n : k + 1, To do
this, we must calculate che value <>fuk+1 from uk (using the recurrence relation and
the induction hypothesis) and check that the result agrees with the formula, i.e. we
check chan we get uw; : 3(/c +1)— 1.
Putting n : k in the recurrence relation, gives
11;;+1 : uk + 34 (2,1]
\end{verbatim}
%---------------------------------------------- %

\begin{verbatim}
Using the induction hypothesis to substitute for uk in [2.}.), gives
1.%+; =(3k—l)+3=3k+2:3(k+l}—1.
Thus the formula holds when wz : k-+1. Hence it holds for all n 2 1, by inducticnl
\·Vc can use induction tu prove results “f`0r all n 2 ng", for any integer ng; the
base: case is than n : no and che zest of the proof follows as above. Notice that
your base case is always the least value of rz for which Lhe scacemenu is true. In nhs
following example, this least value of n is n = 0.
\end{verbatim}
%---------------------------------------------- %
\newpage

%15 CHAPTER 2. SEQUENCES, SERIES AND PROOF BY INDUCTION

\subsection*{Example 2.2}
\begin{verbatim}

 A sequence is determined by the recurrence relation u,., = 4u,,-i —
314,,-2 and the initial terms up ¤ O, ui = 2. We shall prove that un : 3" — 1.
Notice that we could not calculate ug and subsequent terms of this sequence
unless we had been given the values of two initial terms. Thus for the base case.
\end{verbatim}
%---------------------------------------------- %
\begin{verbatim}
we must verify the formula is correct for BOTH Hq and ug. Also, note that the
recurrence relation connects u,, with two previous terms, not just with uh,.
Base cases VVhen n : 0, the formula u,, : 3" -1 gives no ; SU -1 : O; and when
rr: l, it gives nl : 31 -1: 2. Hence it holds for ri : O and rz z l.
Induction hypothesis Suppose the formulaun : 3“——1lioldsfor n = O, 1, 2, . . .,k—
I (note that for algebraic convenience, we go just to k - l this time).
Induction step We prove the formula also holds for n : k. From the recurrence
relation, we have
uk:4uk_1-321;,-2. (2.2]
\end{verbatim}
%---------------------------------------------- %
\begin{verbatim}
By the induction hypothesis, the result is true when ri : Ic -1 and n : lr-2. Hence
ure; : 2k-* -1 uuui ui-2 : 3*-2 -1. Substituting into (2.2) gives
ur : 4 ($*-1 ~ 1) ~ 3 (2** -1)
Z ,,<3k-1) _4_ 3k-i +3
e (4-i)3k`;-1:3*-1.
\end{verbatim}
%---------------------------------------------- %
\begin{verbatim}
Thus the formula also holds when n : k and hence holds for all n E   by induction]
We shall find further applications of proof by induction later.
$
\end{verbatim}
%---------------------------------------------- %
\subsection*{Series and the Sigma Notation}
\begin{verbatim}

A finite series is what we get when we add together a finite number of terms of
za. sequence. A handy notation for writing series uses the Greek letter sigma E as
follows:
 
u-;+u;;+...+un :21:,
1*:1
\end{verbatim}
%---------------------------------------------- %
\begin{verbatim}
We read the right hand side as “the sum of ur from r = l to P 2 ri". The integers
1 and rz are known respectively as the lower and upper limits of summation;
the variable r is called the index of summation.
\end{verbatim}
%---------------------------------------------- %
\subsection*{Example 2.3}
\begin{verbatim}
 Consider the following sums.
(ei) 1+2+3+...+u
Here, we can put u,. : r; then i· : 1 gives the first term in the sum and r : n
gives the last term. S0 we can write
 
l+2+3+...+n=Zi*.
rei
(ti 1¤+22+a2+...+¤’
Here, we can put u, : rz; then r : I gives the first term in the sum and
r : n gives the last term. So we can write
 
12+2;)+32+...-é·nE = Eli-?
rei
\end{verbatim}
%---------------------------------------------- %
\begin{verbatim}
17
(c)1+2+4+S·e...2" :2°+2' +22+...+2r
Here, we can put u, : S1'; then r z O gives the first term in the sum and
v- = n gives the last term. So we can write
-  
1+2+4+S+...2":Z2'.
rec
 
(d) 2 (3r — 1)
\end{verbatim}
%---------------------------------------------- %
\begin{verbatim}
In Example 241, we showed that the formula rr, : Sr — 1, generates the
sequence 2,5,8,. . .,(3n — 1), where the First term ur : 2 is given by ·r = 1,
and the last term u,, = 3n - 1 is given by r : n, So we can write
 
 \[ \sum_{i=1}^{i=n} \]
E(3r+1):2+5+8+..,+(3n—l)-

Tar
\end{verbatim}
%---------------------------------------------- %

Induction is a useful method for verifying a formula for the sum of a finite number
of terms of a sequence because it is very easy to obtain a recurrence relation betwean
the sum ofthe first n +1 terms and the sum of the First n terms. 



\begin{verbatim}

To see this let
ul, 112,,.. be a sequence and for any positive integer n let 5,, = ul — ug + ...-1- 21,,.
\end{verbatim}
%---------------------------------------------- %
\begin{verbatim}
Then
5}.+.1 = (ur + 112+-..+ un) + um.; = 5,, +u+,+r-
Tbis idea is illustrated in the proofs of parts (b) and (c) of the following theorem.
Theorem 2.4 Let- n be a positive integer. Then
 
(E!.) E 1 : 11.
.-:1
\end{verbatim}
\[\frac{(n+1)}{2}\]
%---------------------------------------------- %
\begin{verbatim} 
(b) Z r:n   \frac{(n+1)}{2}.
r:1
(c) fj rz: n(n+1)(2n+1)/5.
mr
"‘ »~+i_, __ .
(d) Z af = LET, for anyx G 5% with m #1.
peo
Proof. (a) In this sum we have ur : 1, for r : 1,2, 4. .n. So we are adding
, 1+1+... + 1, giving rz altogether.
\end{verbatim}
%---------------------------------------------- %
\begin{verbatim}
V (b) Let 5,, denote the sum of the first n integers, so that 5,, : 1+ 2 + . . . + n. We
prove by induction that Sn 2 n (n + 1) /2, for all ri 2 1.
Base case The formula gives S; = 1 (1+1)/2 = 1, so the formula holds when
n : 1.
\end{verbatim}
%---------------------------------------------- %
\textbf{Induction Hypothesis}
\begin{verbatim}

Suppose that Sn = ri (n + 1) /2, for n : 1, 2, . . Wk; then
in particular we know that Sk = k (k + 1) /2,
Induction step We prove that that 5,, : n (n -5- 1) /2 is also true when rt = ic +1;
that is, we find 5;,.,.; from S), and check that the result agrees with the the formula.
\end{verbatim}
%---------------------------------------------- %
\begin{verbatim}

New Sk:1+2+3~$-...+k and.51+;:1+2+3+..,+k+(k+1),s0
5,,+, = St + (k + 1) . (23)
Using the induction hypothesis to substitute for Sk in (2.3) gives
51.:+1 = k(k+1)/2+ (k-l-1)
: (lc +1) (lc/2+1)
: (ir-1-1) (k+2)/2.
$
\end{verbatim}
%---------------------------------------------- %

%18 CHAPTER 2. SEQUENCES, SERIES AND PROOF BY INDUCTION

\begin{verbatim}

But putting T2 = fr + 1 in the formula gives .5}+1 : (lc »§·1](/it + 2) /2. Thus the
formula also holds for n :. k + 1 and hence it holds for all ri 2 1, by induction.
(0) Let Tn denote the sum of the squares of the first 21 integers, S0 that T,. =
12 + 22 + .i .+ nz. We shall prove by induction that Tn is given by the formula;
T,,=n(n+ l)(2n+1)/6. for ell ri 21.
\end{verbatim}
%---------------------------------------------- %
\textbf{Base Case}
\begin{verbatim}
When n :: 1, K :12. The formula gives T,:1(1+1)(2 +1)/6 :1.



\end{verbatim}
%---------------------------------------------- %
Hence the formula holds when n = 1.

\textbf{Induction hypothesis}
\begin{verbatim}

 Suppose Tn : n (n +1)(2n + 1} /5 is uruefor n = 1, 2,. ,/cg
then, in particular, we know that Ti, :     + 1} (2k + 1) /6.
Induction stop 1:Ve prove that the fvrmula. also holds for n : lr + 1; that is, we
calculate Tk 1., from Tk and check that the result agrees with the formula. From the
recurrence relation we have
Tk-il =YL+(l¢+l)2.
\end{verbatim}
%---------------------------------------------- %
Using the induction hypothesis to substitute for Tk gives
\begin{verbatim}

Tk+i Z k(k+1)(2k+1)/6-&—(k+1)2
= (k+ 1)[k(2k+1)/6-1-(k-i-1)]
: (i+1){2k2-+rk+6}/6
: (k+i)(k+2)(2k+3)/6.
\end{verbatim}
%---------------------------------------------- %

\begin{verbatim}
But putting n, : k+1 in the formula gives TH; : (lc +1](}c + 2] (2}: + 3) /6. Thus
the formula also holds for n : k + 1 and hence it holds for all rz 2 1, by induction.
[cl) Let S : 1-+ z + .7:2 +1 ..+ :c“, where n: st 1. Multiplying through by z, gives
2:8 : m + 2:2 + ara + . ..+ x"‘“. Subtracting, we have a:S -— S = :v"+1 — l. Thus
S(r —~ 1) : NH ~— 1, and dividing both sides by m —~ 1 gives the required result]
Note Part (cl) can also be proved by induction.

\end{verbatim}
%---------------------------------------------- %
The sigma notation is not gust a convenient shorthand for writing sums. Most
importantly, it gives us a way of working out the sum of a complicated expression
by turning it into simpler sums. We can do this by applying combinations of the
following three simple rules.
Expressing a sum as a difference of known sums.
\begin{verbatim}
 For example
zu zu in
Z r :. Z r - E r
mir rei rzl
: 20 >< 21/2- 10 ><11/2:155.
Taking out ci common factor For example
s(1)+s(3)+5(s2)+...+s(s"·i):a(i+2+2F+...+a"·1).
Thus
¤-i n-1
Einar; : afar.
eso rea
\end{verbatim}
%---------------------------------------------- %
The common factor 5 can be taken outside the sigma sign because it can be taken
outside the bracket in the "long hand" version of the sum.
\begin{verbatim}

Splitting o sum into two {or mare) components. For example
(1+12)-l—(2+22)+   +(n+n2)
Z (1+2+2+...e¤)+(12+22+32+...+n2)
1

\end{verbatim}
%---------------------------------------------- %

\begin{verbatim}
Thus
E{7"‘l·T’2) Z 27*+273
1*:1 1*:1 r:i
We may formalise these rules in the following theorem.
Theorem 2.5 (21) Z u, :. E ur — Z u,.
v’:m 1*:1 :*:1
\end{verbatim}
%---------------------------------------------- %
\begin{verbatim}  
fb) Z cu, : c Z 11, , where c is a constant.
(C) Z (¤T+w¢l= E ¤1+ E w1·
Proof.
(ai) follows immedianely,
(bl
H N
Em =c(um+um+1+um+g#...+un) =cZ1-rr.
(C)
 
2 (ue + wr) = (um + wi-»1)+(Um+1 + w1n+1)’l" »..-#(1:.-. +wn)
= (vm + um~l·1 +   uy.)   (wm + wm+1 + ·.-+w¤)
= Z ue + Z wr.
I
\end{verbatim}
%---------------------------------------------- %
By taking out factors and splitting up the sums, we may reduce a complicated
sum to simpler sums for which we already know a formula.
\begin{verbatim}

Example 2.6 We now find the formula for the sum in Example 2.3(d).
xw - 1) Z Ear - Z1, by Theorem 2.5(b).
r:l :-:1 :-:1
   
: 3 Z r — Z 1, by Theorem 2.5(a),
r:1 r:1
Hence, by Theorem 2.4 (a) and (ln),
 
§;(3r-· 1) : 3n(n—|-1)/2-n
1-:1
: n[3{n+1)/2-1}
\end{verbatim}
%%- 20 CHAPTER 24 SEQUENCES, SERIES AND PROOF BY INDUCTION
\section{Exercise 3}
%---------------------------------------------- %
\begin{verbatim}

Q1 For each of the following sequences, [i) calculate the next term ofthe sequence
and (ii) find a recurrence relation that gives #4,,+; in terms of un.
(sp 4.2.1,%,%,...: [z]
(lu) 2.7,111122 ,l,. [3]
Q2 Determine the value of u,._, for n : 1, 2,3,4, for the sequences determined by
each of the following recurrence relations.
(al wei = 5%   2.111 : 0; [2]
(bl ¤,,+;:un+;—un,u;=0 andug=1. [2]
\end{verbatim}
%---------------------------------------------- %
Q3 A sequence is determined by the recurrence relation
\begin{verbatim}
 une] cv Bun + 2 and initial
term u; = 2. 

Prove by induction that un : 3" — 1.. for all n E 2+.  
Q4 Let n be E1 positive integer and z be a real number with x gé l. State, without
proving, the formulae for
ei it rr me
rei
 \end{verbatim}
 %---------------------------------------------- %
 \begin{verbatim}
(bl Z rs Eli
rei
(cl Z ¤¤’· [11
rea
Q5 Use the formulae you stated in Q4 to evaluate
au _
(el Eli; [21
im . .
(bl El? + il- [21
\end{verbatim}
%---------------------------------------------- %
\begin{verbatim}
Q6 Let sn :1+3+5-]-...+(2n—1)forn&Z+.
(a) Express sn using Z notation. [1]
(b) Calculate sl, s; and sg. [1]
(c) Find a recurrence relation which expresses sus.; in terms of sn. [2]
(d) Use induction to prove that su : nz for all ri 2 1. [5]
\end{verbatim}
%---------------------------------------------- %
\end{document}

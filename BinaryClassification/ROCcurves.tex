\documentclass[]{report}

\voffset=-1.5cm
\oddsidemargin=0.0cm
\textwidth = 480pt

\usepackage{framed}
\usepackage{subfiles}
\usepackage{enumerate}
\usepackage{graphics}
\usepackage{newlfont}
\usepackage{eurosym}
\usepackage{amsmath,amsthm,amsfonts}
\usepackage{amsmath}
\usepackage{color}
\usepackage{amssymb}
\usepackage{multicol}
\usepackage[dvipsnames]{xcolor}
\usepackage{graphicx}
\begin{document}

\subsection{ROC Curves}
\begin{itemize}
\item  A receiver operating characteristic (ROC), or ROC curve, is a graphical plot that illustrates
the performance of a binary classification system as its discrimination threshold is varied.
\item  The ROC curve is created by plotting the true positive rate against the false positive rate at
various threshold settings. (The true-positive rate is also known as sensitivity in biomedicine,
or recall in machine learning. The false-positive rate is also known as the fall-out and can be
calculated as 1 - specificity).
\item  The ROC curve was first developed by electrical engineers and radar engineers during World
War II for detecting enemy objects in battlefields. ROC analysis since then has been used in
medicine, radiology, biometrics, and other areas for many decades and is increasingly used
in machine learning and data mining research.
\item  The ROC is also known as a relative operating characteristic curve, because it is a comparison
of two operating characteristics (TPR and FPR) as the criterion changes.
\end{itemize}
%====================================================%
\subsection{Properties of ROC Curves}
An ROC curve demonstrates several things:
1. It shows the tradeoff between sensitivity and specificity (any increase in sensitivity will be
accompanied by a decrease in specificity).
2. The closer the curve follows the upper-left border of the ROC space, the more accurate the
test.
3. The closer the curve comes to the 45-degree diagonal of the ROC space, the less accurate the
test.
4. The slope of the tangent line at a cutpoint gives the likelihood ratio (LR) for that value of
the test.
5. The Area Under the Curve is a measure of accuracy.
Figure 1: Receiver Operating Characteristic (ROC) curve

%====================================================%
\begin{itemize}
\item  In a Receiver Operating Characteristic (ROC) curve the true positive rate (Sensitivity) is
plotted in function of the false positive rate (100-Specificity) for different cut-off points.
\item  Each point on the ROC curve represents a sensitivity/specificity pair corresponding to a
particular decision threshold.

\item  A test with perfect discrimination (no overlap in the two distributions) has a ROC curve that
passes through the upper left corner (100% sensitivity, 100% specificity).
\item  Therefore the closer the ROC curve is to the upper left corner, the higher the overall accuracy
of the test.

\end{itemize}
\end{document}


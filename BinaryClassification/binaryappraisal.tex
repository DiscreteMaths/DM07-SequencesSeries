
\documentclass[a4paper,12pt]{article}
%%%%%%%%%%%%%%%%%%%%%%%%%%%%%%%%%%%%%%%%%%%%%%%%%%%%%%%%%%%%%%%%%%%%%%%%%%%%%%%%%%%%%%%%%%%%%%%%%%%%%%%%%%%%%%%%%%%%%%%%%%%%%%%%%%%%%%%%%%%%%%%%%%%%%%%%%%%%%%%%%%%%%%%%%%%%%%%%%%%%%%%%%%%%%%%%%%%%%%%%%%%%%%%%%%%%%%%%%%%%%%%%%%%%%%%%%%%%%%%%%%%%%%%%%%%%
\usepackage{eurosym}
\usepackage{vmargin}
\usepackage{amsmath}
\usepackage{graphics}
\usepackage{epsfig}
\usepackage{framed}
\usepackage{subfigure}
\usepackage{fancyhdr}

\setcounter{MaxMatrixCols}{10}
%TCIDATA{OutputFilter=LATEX.DLL}
%TCIDATA{Version=5.00.0.2570}
%TCIDATA{<META NAME="SaveForMode"CONTENT="1">}
%TCIDATA{LastRevised=Wednesday, February 23, 201113:24:34}
%TCIDATA{<META NAME="GraphicsSave" CONTENT="32">}
%TCIDATA{Language=American English}

\pagestyle{fancy}
\setmarginsrb{20mm}{0mm}{20mm}{25mm}{12mm}{11mm}{0mm}{11mm}
\lhead{MA4128} \rhead{Kevin O'Brien} \chead{Week 8} %\input{tcilatex}

%http://www.electronics.dit.ie/staff/ysemenova/Opto2/CO_IntroLab.pdf
\begin{document}


%%%%%%%%%%%%%%%%%%%%%%%%%%%%%%%%%%%%%%%%%%%%%%%%%%%%%%%%%%%%%%%%%%%%%%%%%%%%%%%%%%%%%%%%%%%%%%%%%%%%
\section{Performance of Classification Procedure}
	
	These classifications are used to calculate accuracy, precision (also called positive predictive value), recall (also called sensitivity), specificity and negative predictive value:
	
	\begin{itemize}
		\item  \textbf{Accuracy} is the fraction of observations with correct predicted classification
		\[ \mbox{Accuracy}=\frac{TP+TN}{TP+FP+FN+TN}\]
		
		
		\item \textbf{Precision} is the proportion of predicted positives that are correct
		\[
		\mbox{Precision} = \mbox{Positive Predictive Value} =\frac{TP}{TP+FP} \, \]
		
		\item \textbf{Negative Predictive Value} is the  fraction of predicted negatives that are correct
		\[\mbox{Negative Predictive Value} = \frac{TN}{TN+FN}\]
		
		\item \textbf{Recall} is the fraction of observations that are actually 1 with a correct predicted classification
		\[ 
		\mbox{Recall} = \mbox{Sensitivity} = \frac{TP}{TP+FN} \,  \]
		
		\item \textbf{Specificity} is the fraction of observations that are actually 0 with a correct predicted classification
		\[ \mbox{Specificity} = \frac{TN}{TN+FP} \]
		
	\end{itemize}



%%%%%%%%%%%%%%%%%%%%%%%%%%%%%%%%%%%%%%%%%%%%%%%%%%%%%%%%%%%%%%%%%%%%%%%%%%%%%%%%%%%%%%%%%%%%%%%%%%%%
\section{Performance of Classification Procedure}
	
	These classifications are used to calculate accuracy, precision (also called positive predictive value), recall (also called sensitivity), specificity and negative predictive value:
	
	\begin{itemize}
		\item  \textbf{Accuracy} is the fraction of observations with correct predicted classification
		\[ \mbox{Accuracy}=\frac{TP+TN}{TP+FP+FN+TN}\]
		
		
		\item \textbf{Precision} is the proportion of predicted positives that are correct
		\[
		\mbox{Precision} = \mbox{Positive Predictive Value} =\frac{TP}{TP+FP} \, \]
		
		\item \textbf{Negative Predictive Value} is the  fraction of predicted negatives that are correct
		\[\mbox{Negative Predictive Value} = \frac{TN}{TN+FN}\]
		
		\item \textbf{Recall} is the fraction of observations that are actually 1 with a correct predicted classification
		\[ 
		\mbox{Recall} = \mbox{Sensitivity} = \frac{TP}{TP+FN} \,  \]
		
		\item \textbf{Specificity} is the fraction of observations that are actually 0 with a correct predicted classification
		\[ \mbox{Specificity} = \frac{TN}{TN+FP} \]
		
	\end{itemize}

\end{document}

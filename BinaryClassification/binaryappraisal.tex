
\documentclass[a4paper,12pt]{article}
%%%%%%%%%%%%%%%%%%%%%%%%%%%%%%%%%%%%%%%%%%%%%%%%%%%%%%%%%%%%%%%%%%%%%%%%%%%%%%%%%%%%%%%%%%%%%%%%%%%%%%%%%%%%%%%%%%%%%%%%%%%%%%%%%%%%%%%%%%%%%%%%%%%%%%%%%%%%%%%%%%%%%%%%%%%%%%%%%%%%%%%%%%%%%%%%%%%%%%%%%%%%%%%%%%%%%%%%%%%%%%%%%%%%%%%%%%%%%%%%%%%%%%%%%%%%
\usepackage{eurosym}
\usepackage{vmargin}
\usepackage{amsmath}
\usepackage{graphics}
\usepackage{epsfig}
\usepackage{framed}
\usepackage{subfigure}
\usepackage{fancyhdr}

\setcounter{MaxMatrixCols}{10}
%TCIDATA{OutputFilter=LATEX.DLL}
%TCIDATA{Version=5.00.0.2570}
%TCIDATA{<META NAME="SaveForMode"CONTENT="1">}
%TCIDATA{LastRevised=Wednesday, February 23, 201113:24:34}
%TCIDATA{<META NAME="GraphicsSave" CONTENT="32">}
%TCIDATA{Language=American English}

\pagestyle{fancy}
\setmarginsrb{20mm}{0mm}{20mm}{25mm}{12mm}{11mm}{0mm}{11mm}
\lhead{MA4128} \rhead{Kevin O'Brien} \chead{Week 8} %\input{tcilatex}

%http://www.electronics.dit.ie/staff/ysemenova/Opto2/CO_IntroLab.pdf
\begin{document}


Binary Classification
What Is Classification
Classification is the problem of identifying to which of a set of categories (sub-populations) a new
observation belongs, on the basis of a training set of data containing observations (or instances)
whose category membership is known. Binary Classification is the task of classifying the members of
a given set of objects into two groups on the basis if them having a particular set of characteristics.
• To train (create) a classifier, the fitting function estimates the parameters of a Gaussian
distribution for each class.
• To predict the classes of new data, the trained classifier finds the class with the smallest
misclassification cost.
Binary Classification Prediction Procedure
Binary Variabble: Positive or Negative
Four Possible Outcomes from Classification Procedure:
• TN / True Negative: Case was actually negative and was also predicted negative
(CORRECT).
• TP / True Positive: Case was actually positive and was also predicted positive (CORRECT).
• FN / False Negative: Case was actually positive but was predicted negative
(WRONG).
• FP / False Positive: Case was actually negative but was predicted positive (WRONG).
Remark : We will use this notation to specify the number of cases in each category also: i.e. TP
= 5000 means 5000 True Positives.
Confusion Matrix
• The Confusion Matrix is a table in which the rows are the observed categories of the dependent
and the columns are the predicted categories.
• A confusion matrix reports the number of false positives, false negatives, true positives, and
true negatives. This allows more detailed analysis than mere proportion of correct guesses
(accuracy).
• Accuracy is not a reliable metric for the real performance of a classification system, because
it will yield misleading results if the data set is unbalanced (that is, when the number of
samples in different classes vary greatly).
• For example, if there were 95 cats and only 5 dogs in the data set, the classifier could
easily be biased into classifying all the samples as cats. The overall accuracy would be
95%, but in practice the classifier would have a 100% recognition rate for the cat class
but a 0% recognition rate for the dog class.
1
Predicted Predicted
Negative Positive
Actual State: Negative TN FP
Actual State: Positive FN TP
False Positive and False Negative Error
• A false positive error, commonly called a “false alarm“, is a result that indicates a given
condition has been fulfilled, when it actually has not been fulfilled. A false positive error is
a Type I error.
• A false negative error is where a test result indicates that a condition failed, while it actually
was successful. A false negative error is a Type II error.
Medical Testing example Defining true/false positives
In general, Positive = identified and negative = rejected. Therefore:
TN True negative = Healthy people correctly identified as healthy (correctly rejected)
FP False positive = Healthy people incorrectly identified as sick (incorrectly identified)
FN False negative = Sick people incorrectly identified as healthy. incorrectly rejected
TP True positive = Sick people correctly diagnosed as sick (correctly identified)
Types I and II Error (For Later)
• A Type I error is the incorrect rejection of a true null hypothesis.
• A Type II error is the failure to reject a false null hypothesis.
• A Type I error is a false positive. Usually a type I error leads one to conclude that a thing
or relationship exists when really it doesn’t.
• A type II error is a false negative.
Null hypothesis (H0) is true Null hypothesis (H0) is false
Reject Type I error Correct Outcome
null hypothesis False positive True positive
Fail to reject Correct Outcome Type II error
null hypothesis True negative False negative
Accuracy Rate
The accuracy rate calculates the proportion ofobservations being allocated to the correct group
by the predictive model. It is calculated as follows:
Accuracy = Number of Correct Classifications
Total Number of Classifications
Accuracy = T P + T N
T P + F P + T N + F N
2
Misclassification Rate
The misclassification rate calculates the proportion ofobservations being allocated to the incorrect
group by the predictive model. It is calculated as follows:
Number of Incorrect Classifications
Total Number of Classifications
=
F P + F N
T P + F P + T N + F N
Sensitivity and Specificity
Sensitivity and specificity are measures of the performance of a binary classification test.
• Sensitivity (also called the true positive rate, or the recall rate) measures the proportion
of actual positives which are correctly identified as such (e.g. the percentage of sick people
who are correctly identified as having the condition).
Sensitivity (Recall) = T P
T P + F N
• Examples: Sensitivity (TPR), also known as recall, is the proportion of people that tested
positive (TP) of all the people that actually are positive (TP+FN).
– It can be seen as the probability that the test is positive given that the patient is sick.
– With higher sensitivity, fewer actual cases of disease go undetected (or, in the case of
the factory quality control, the fewer faulty products go to the market).
– (Remark: We will use the terms Sensitivity and Recall interchangeably. Sensitivity is
more commonly used in a medical context, while recall is more commonly used in data
science.)
• Specificity measures the proportion of negatives which are correctly identified as such (e.g.
the percentage of healthy people who are correctly identified as not having the condition,
sometimes called the true negative rate).
Specificity = T N
T P + F N
– (Remark: Not commonly used in Data Sciences, and NOT a synonym for Precision)
• Examples: Specificity (TNR) is the proportion of people that tested negative (TN) of all
the people that actually are negative (TN+FP). As with sensitivity, it can be looked at as
the probability that the test result is negative given that the patient is not sick.
• With higher specificity, fewer healthy people are labeled as sick (or, in the factory case, the
less money the factory loses by discarding good products instead of selling them).
• The relationship between sensitivity and specificity, as well as the performance of the classi-
fier, can be visualized and studied using the ROC curve (Which we shall see shortly).
Precision
In a binary classification procedure, the precision for a class is the number of true positives (i.e. the
number of items correctly labeled as belonging to the positive class) divided by the total number
of cases labeled as belonging to the positive class (i.e. the sum of true positives and false positives,
which are items incorrectly labeled as belonging to the class).
Precision = T P
T P + F P
(1)
3
Recall
Recall is defined as the number of true positives divided by the total number of cases that actually
belong to the positive class (i.e. the sum of true positives and false negatives, which are items
which were not labeled as belonging to the positive class but should have been).
Recall = T P
T P + F N
(2)
Accuracy, Recall and Precision: An Example
Suppose we are designing a medical diagnosis system, and we have enlisted 10000 volunteers to
help us test it. Suppose there are 135 positive cases of an illness among the10,000 cases. You want
to predict which ones are positive, and you pick 265 to have a better chance of catching many of
the 135 positive cases. You record the IDs of your predictions, and when you get the actual results
you sum up how many times you were right or wrong.
Now count how many of the 10,000 cases fall in each category:
Predicted Negative Predicted Positive
Negative Cases TN: 9,700 FP: 165
Positive Cases FN: 35 TP: 100
1. What percent of your predictions were correct?
• The accuracy was (9,760+60) out of 10,000 = 98.00%
2. What percent of the positive cases did you catch?
• The recall was 100 out of 135 = 74.07%
3. What percent of positive predictions were correct?
• The precision was 100 out of 265 = 37.74%
4. What percent of negative predictions were correct?
• The specifity was 9700 out of 9735 = 99.64%
Class Imbalance
• A data set said to be highly skewed if sample from one class is in higher number than other.
• In an imbalanced data set the class having more number of instances is called as major class
while the one having relatively less number of instances are called as minor class .
• Applications such as medical diagnosis prediction of rare but important disease is very important
than regular treatment.
• Similar situations are observed in other areas, such as detecting fraud in banking operations,
detecting network intrusions, managing risk and predicting failures of technical equipment.
• In such situation most of the binary classification procedure are biased towards the major
classes and hence show very poor classification rates on minor classes.
• It is also possible that classifier predicts everything as major class and ignores the minor class
completely.
• The Accuracy measure is an example of an metric that is affected by this bias.
• As the F-Score is not computed using the True Negatives, it is less biased.
4
The F Score
The F-score or F-measure is a measure of a classification procedure’s accuracy. It considers both
the precision and the recall to compute the score.
F = 2 ·
precision · recall
precision + recall
• The F-score or F-measure is a single measure of a classification procedure’s usefulness.
• The F-score considers both the Precision and the Recall of the procedure to compute the
score.
• The higher the F-score, the better the predictive power of the classification procedure.
• A score of 1 means the classification procedure is perfect. The lowest possible F-score is 0.
0 ≤ F ≤ 1
From Before
• Precision is the number of correct positive results divided by the number of predicted
positive results.
Precision = T P
T P + F P
• Recall is the number of correct positive results divided by the number of actual positive
results.
Recall = T P
T P + F N
The F-score is the Harmonic mean of Precision and Recall.
F =
2
1
Recall +
1
Precision
Alternatively
F = 2 ×

Precision × Recall
Precision + Recall
Example
Number of cases: 100,000
Predicted Negative Predicted Positive
Actual State: Negative TN = 97750 FP = 150
Actual State: Positive FN= 330 TP =1770
• Accuracy = 0.9952
• Recall = 0.8428
• Precision = 0.9218
F = 2 ×
Precision × Recall
Precision + Recall
F = 2 ×

0.9218 × 0.8428
0.9218 + 0.8428
= 2 ×

0.7770
1.7646
= 2 × 0.4402
F = 0.8804
5
ROC Curves
• A receiver operating characteristic (ROC), or ROC curve, is a graphical plot that illustrates
the performance of a binary classification system as its discrimination threshold is varied.
• The ROC curve is created by plotting the true positive rate against the false positive rate at
various threshold settings. (The true-positive rate is also known as sensitivity in biomedicine,
or recall in machine learning. The false-positive rate is also known as the fall-out and can be
calculated as 1 - specificity).
• The ROC curve was first developed by electrical engineers and radar engineers during World
War II for detecting enemy objects in battlefields. ROC analysis since then has been used in
medicine, radiology, biometrics, and other areas for many decades and is increasingly used
in machine learning and data mining research.
• The ROC is also known as a relative operating characteristic curve, because it is a comparison
of two operating characteristics (TPR and FPR) as the criterion changes.
Properties of ROC Curves
An ROC curve demonstrates several things:
1. It shows the tradeoff between sensitivity and specificity (any increase in sensitivity will be
accompanied by a decrease in specificity).
2. The closer the curve follows the upper-left border of the ROC space, the more accurate the
test.
3. The closer the curve comes to the 45-degree diagonal of the ROC space, the less accurate the
test.
4. The slope of the tangent line at a cutpoint gives the likelihood ratio (LR) for that value of
the test.
5. The Area Under the Curve is a measure of accuracy.
Figure 1: Receiver Operating Characteristic (ROC) curve
• In a Receiver Operating Characteristic (ROC) curve the true positive rate (Sensitivity) is
plotted in function of the false positive rate (100-Specificity) for different cut-off points.
• Each point on the ROC curve represents a sensitivity/specificity pair corresponding to a
particular decision threshold.
6
• A test with perfect discrimination (no overlap in the two distributions) has a ROC curve that
passes through the upper left corner (100% sensitivity, 100% specificity).
• Therefore the closer the ROC curve is to the upper left corner, the higher the overall accuracy
of the test.
7

%%%%%%%%%%%%%%%%%%%%%%%%%%%%%%%%%%%%%%%%%%%%%%%%%%%%%%%%%%%%%%%%%%%%%%%%%%%%%%%%%%%%%%%%%%%%%%%%%%%%
\section{Performance of Classification Procedure}
	
	These classifications are used to calculate accuracy, precision (also called positive predictive value), recall (also called sensitivity), specificity and negative predictive value:
	
	\begin{itemize}
		\item  \textbf{Accuracy} is the fraction of observations with correct predicted classification
		\[ \mbox{Accuracy}=\frac{TP+TN}{TP+FP+FN+TN}\]
		
		
		\item \textbf{Precision} is the proportion of predicted positives that are correct
		\[
		\mbox{Precision} = \mbox{Positive Predictive Value} =\frac{TP}{TP+FP} \, \]
		
		\item \textbf{Negative Predictive Value} is the  fraction of predicted negatives that are correct
		\[\mbox{Negative Predictive Value} = \frac{TN}{TN+FN}\]
		
		\item \textbf{Recall} is the fraction of observations that are actually 1 with a correct predicted classification
		\[ 
		\mbox{Recall} = \mbox{Sensitivity} = \frac{TP}{TP+FN} \,  \]
		
		\item \textbf{Specificity} is the fraction of observations that are actually 0 with a correct predicted classification
		\[ \mbox{Specificity} = \frac{TN}{TN+FP} \]
		
	\end{itemize}



%%%%%%%%%%%%%%%%%%%%%%%%%%%%%%%%%%%%%%%%%%%%%%%%%%%%%%%%%%%%%%%%%%%%%%%%%%%%%%%%%%%%%%%%%%%%%%%%%%%%
\section{Performance of Classification Procedure}
	
	These classifications are used to calculate accuracy, precision (also called positive predictive value), recall (also called sensitivity), specificity and negative predictive value:
	
	\begin{itemize}
		\item  \textbf{Accuracy} is the fraction of observations with correct predicted classification
		\[ \mbox{Accuracy}=\frac{TP+TN}{TP+FP+FN+TN}\]
		
		
		\item \textbf{Precision} is the proportion of predicted positives that are correct
		\[
		\mbox{Precision} = \mbox{Positive Predictive Value} =\frac{TP}{TP+FP} \, \]
		
		\item \textbf{Negative Predictive Value} is the  fraction of predicted negatives that are correct
		\[\mbox{Negative Predictive Value} = \frac{TN}{TN+FN}\]
		
		\item \textbf{Recall} is the fraction of observations that are actually 1 with a correct predicted classification
		\[ 
		\mbox{Recall} = \mbox{Sensitivity} = \frac{TP}{TP+FN} \,  \]
		
		\item \textbf{Specificity} is the fraction of observations that are actually 0 with a correct predicted classification
		\[ \mbox{Specificity} = \frac{TN}{TN+FP} \]
		
	\end{itemize}

\end{document}

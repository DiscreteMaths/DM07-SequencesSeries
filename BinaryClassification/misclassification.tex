
\subsection{Misclassification Cost}
\begin{itemize}
	\item As in all statistical procedures it is helpful to use diagnostic procedures to assess the efficacy of the analysis. We use \textbf{cross-validation} to assess the classification probability.
	Typically you are going to have some prior rule as to what is an \textbf{acceptable misclassification rate}.
	
	\item	Those rules might involve things like, \textit{``what is the cost of misclassification?"} Consider a medical study where you might be able to diagnose cancer.
	
	\item There are really two alternative costs. 
	\begin{itemize}
		\item[$\ast$] The cost of misclassifying someone as having cancer when they don't.
	This could cause a certain amount of emotional grief. Additionally there would be the substantial cost of unnecessary treatment.
	
	\item[$\ast$] There is also the alternative cost of misclassifying someone as not having cancer when in fact they do have it.
	\end{itemize}
	\item A good classification procedure should
	\begin{itemize}
		\item[$\ast$] result in few misclassifications
		\item[$\ast$] take \textbf{\textit{prior probabilities of occurrence}} into account
		\item[$\ast$] consider the cost of misclassification
	\end{itemize}
	
	\item 	For example, suppose there tend to be more financially sound firms than bankrupt
	firm. If we really believe that the prior probability of a financially
	distressed and ultimately bankrupted firm is very small, then one should
	classify a randomly selected firm as non-bankrupt unless the data
	overwhelmingly favor bankruptcy.
	
	
	
	\item 	There are two costs associated with discriminant analysis classification: The true misclassification cost per class, and the expected misclassification cost (ECM) per observation.
	
	\item 	\textbf{Example} Suppose there we have a binary classification system, with two classes: class 1 and class 2.
	Suppose that classifying a class 1 object as belonging to class 2 represents a more serious error than classifying a class 2 object as belonging to class 1. There would an assignable cost to each error.
	\textbf{c(i$|$j)} is the cost of classifying an observation into class $j$ if its true class is $i$.
	The costs of misclassification can be defined by a cost matrix.
	
	\begin{center}
	\begin{tabular}{|c|c|c|}
		\hline
		% after \\: \hline or \cline{col1-col2} \cline{col3-col4} ...
		& Predicted & Predicted \\
		& Class 1 & Class 2 \\  \hline
		Class 1 & 0 & $c(2|1)$  \\ \hline
		Class 2 & $c(1|2)$ & 0 \\
		\hline
	\end{tabular}
	\end{center}
	
\end{itemize}

\noindent \textbf{Expected cost of misclassification (ECM)}
\begin{itemize}
	\item Let $p_1$ and $p_2$ be the prior probability of class 1 and class 2 respectively.
	Necessarily $p_1$ + $p_2$ = 1.
	
\item	The conditional probability of classifying an object as class 1 when it is in fact from
	class 2 is denoted $p(1|2)$.
\item 	Similarly the conditional probability of classifying an object as class 2 when it is in
	fact from class 1 is denoted $p(2|1)$.
	
	\[ECM = c(2|1)p(2|1)p_1 + c(1|2)p(1|2)p_2\]
\textit{(In other words: the sum of the cost of misclassification times the (joint) probability of that misclassification.)}
	
\item 	A reasonable classification rule should have ECM as small as possible.
\end{itemize}


\subsection{Misclassification Rate}
The misclassification rate calculates the proportion ofobservations being allocated to the \textbf{incorrect} group by the predictive model. It is calculated as follows:
\[ \frac{
\mbox{Number of Incorrect Classifications }}{\mbox{Total Number of Classifications }} \]

\[ = \frac{FP + FN}{TP+FP+TN+FN}\]


\subsection{Misclassification Cost}
\begin{itemize}
	\item As in all statistical procedures it is helpful to use diagnostic procedures to assess the efficacy of the analysis. We use \textbf{cross-validation} to assess the classification probability.
	Typically you are going to have some prior rule as to what is an \textbf{acceptable misclassification rate}.
	
	\item	Those rules might involve things like, \textit{``what is the cost of misclassification?"} Consider a medical study where you might be able to diagnose cancer.
	
	\item There are really two alternative costs. 
	\begin{itemize}
		\item[$\ast$] The cost of misclassifying someone as having cancer when they don't.
	This could cause a certain amount of emotional grief. Additionally there would be the substantial cost of unnecessary treatment.
	
	\item[$\ast$] There is also the alternative cost of misclassifying someone as not having cancer when in fact they do have it.
	\end{itemize}
	\item A good classification procedure should
	\begin{itemize}
		\item[$\ast$] result in few misclassifications
		\item[$\ast$] take \textbf{\textit{prior probabilities of occurrence}} into account
		\item[$\ast$] consider the cost of misclassification
	\end{itemize}
	
	\item 	For example, suppose there tend to be more financially sound firms than bankrupt
	firm. If we really believe that the prior probability of a financially
	distressed and ultimately bankrupted firm is very small, then one should
	classify a randomly selected firm as non-bankrupt unless the data
	overwhelmingly favor bankruptcy.
	
	
	
	\item 	There are two costs associated with discriminant analysis classification: The true misclassification cost per class, and the expected misclassification cost (ECM) per observation.
	
	\item 	\textbf{Example} Suppose there we have a binary classification system, with two classes: class 1 and class 2.
	Suppose that classifying a class 1 object as belonging to class 2 represents a more serious error than classifying a class 2 object as belonging to class 1. There would an assignable cost to each error.
	\textbf{c(i$|$j)} is the cost of classifying an observation into class $j$ if its true class is $i$.
	The costs of misclassification can be defined by a cost matrix.
	
	\begin{center}
	\begin{tabular}{|c|c|c|}
		\hline
		% after \\: \hline or \cline{col1-col2} \cline{col3-col4} ...
		& Predicted & Predicted \\
		& Class 1 & Class 2 \\  \hline
		Class 1 & 0 & $c(2|1)$  \\ \hline
		Class 2 & $c(1|2)$ & 0 \\
		\hline
	\end{tabular}
	\end{center}
	
\end{itemize}

\noindent \textbf{Expected cost of misclassification (ECM)}
\begin{itemize}
	\item Let $p_1$ and $p_2$ be the prior probability of class 1 and class 2 respectively.
	Necessarily $p_1$ + $p_2$ = 1.
	
\item	The conditional probability of classifying an object as class 1 when it is in fact from
	class 2 is denoted $p(1|2)$.
\item 	Similarly the conditional probability of classifying an object as class 2 when it is in
	fact from class 1 is denoted $p(2|1)$.
	
	\[ECM = c(2|1)p(2|1)p_1 + c(1|2)p(1|2)p_2\]
\textit{(In other words: the sum of the cost of misclassification times the (joint) probability of that misclassification.)}
	
\item 	A reasonable classification rule should have ECM as small as possible.
\end{itemize}






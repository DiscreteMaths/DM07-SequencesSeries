\documentclass[]{report}

\voffset=-1.5cm
\oddsidemargin=0.0cm
\textwidth = 480pt

\usepackage{framed}
\usepackage{subfiles}
\usepackage{enumerate}
\usepackage{graphics}
\usepackage{newlfont}
\usepackage{eurosym}
\usepackage{amsmath,amsthm,amsfonts}
\usepackage{amsmath}
\usepackage{color}
\usepackage{amssymb}
\usepackage{multicol}
\usepackage[dvipsnames]{xcolor}
\usepackage{graphicx}
\begin{document}

%---------------------------%
\section{Statistical Data Mining}


Statistical data mining, also known as knowledge or data discovery, is a computerized method of collecting and analyzing information. The data-mining tool takes data and categorizes the information to discover patterns or correlations that can be used in important applications, such as medicine, computer programming, business promotion, and robotic design. 


Statistical data mining techniques use complex mathematics and complicated statistical processes to create an analysis.


Data mining involves five major steps. 

\begin{enumerate}
\item The first data mining application collects statistical data and places the information in a warehouse-type program. 

\item Next, the data in the warehouse is organized and creates a management system. 

\item The next step creates a way to access the managed data. 

\item Then, the fourth step develops software to analyze the data, also known as data mining regression,. 

\item The final step facilitates using or interpreting the statistical data in a practical way.
\end{enumerate}

Generally, data mining techniques integrate analytical and transaction data systems. Analytical software sorts through both types of data systems using open-ended user questions. Open-ended questions allow countless answers so programmers are not influencing the results of the sorting. Programmers create lists of questions to assist in categorizing the information using an overall focus.


Sorting is then based on developing classes and clusters of data, associations found in the data, and attempts to define patterns and trends based on the associations. For example, Google collects information on users' purchasing habits to assist in placing online advertising. Open-ended questions used to sort this buyer data focus on buying preferences or viewing habits of Internet users.


Computer scientists and programmers focus on the analysis of the statistical data that is collected. Creation of decision trees, artificial neural networks, nearest neighbor method, rule induction, data visualization, and genetic algorithms all use the statistically-mined data. 


These classification systems assist in interpreting the associations discovered by the analytical data programs. Statistical data mining involves small projects that can be done on a small scale on a home computer, but most data mining association sets are so large and the data mining regression so complicated that they require a supercomputer or a network of high-speed computers.


Statistical data mining collects three general types of data, including operational data, non-operational data, and meta data. In a clothing store, operational data is basic data used to run the business, such as accounting, sales, and inventory control. 


Non-operational data, which is indirectly related to the business, includes estimates of future sales and general information about the national clothing market. 


Meta data concerns the data itself. A program using meta data might sort store customers into classifications based on gender or geographic location of the clothing buyers or the customers favorite color, if that data was collected.


%---------------------------%
\subsection{Applications of Data Mining}

A data mining application can be extremely sophisticated and the statistical data mining tool may have widespread practical applications. The study of disease outbreaks is one example. A 2000 data mining project analyzed the disease outbreak of cryptosporidium in Ontario, Canada to determine the causes of the increase in disease cases. The results of the data mining assisted in linking the bacteria outbreak to local water conditions and the lack of proper municipal water treatment. A field called "biosurveillance" uses epidemiological data mining to identify outbreaks of a single disease.


Computer programmers and designers also employ the study of probability and statistical data analysis to develop machines and computer programs. The Google Internet search engine was designed using statistical data mining. Google continues to collect and use data mining to create program updates and applications.

\newpage
%=====================================================%


% http://www.norusis.com/pdf/SPC_v13.pdf

\subsection*{Supervised Learning}
\begin{itemize}
\item \textbf{Supervised learning} is tasked with learning a function from labeled training data in order to predict the value of any valid input. 

\item Common examples of supervised learning include classifying e-mail messages as spam, labeling Web pages according to their genre, and recognizing handwriting. 
\item Many algorithms are used to create supervised learners, the most common being neural networks, Support Vector Machines (SVMs), and Naive Bayes classifiers.
\end{itemize}

%=====================================================%
\subsection*{Supervised and Unsupervised Learning}
\textbf{Supervised learning} is tasked with learning a function from labeled training data in order to predict the value of any valid input. 


\subsection*{Unsupervised Learning}
\begin{itemize}
	\item
\textbf{Unsupervised learning} is tasked with making sense of data without any examples of what is correct or incorrect. It is most commonly used for clustering similar input into logical groups. 
\item Unsupervised learning  can be used to reduce the number of dimensions in a data set in order to focus on only the most useful attributes, or to detect trends. 

\item Common approaches to unsupervised learning include k-Means, hierarchical clustering, and self-organizing maps.
\end{itemize}


%%%%%%%%%%%%%%%%%%%%%%%%%%%%%%%%%%%%%%%%%%%%%%%%%%%%%%%%%%%%%%%%%%%%%%%%%%%%%%%%%%%%%%%%%%%%%%%%%%%%
\section{Performance of Classification Procedure}
	
	These classifications are used to calculate accuracy, precision (also called positive predictive value), recall (also called sensitivity), specificity and negative predictive value:
	
	\begin{itemize}
		\item  \textbf{Accuracy} is the fraction of observations with correct predicted classification
		\[ \mbox{Accuracy}=\frac{TP+TN}{TP+FP+FN+TN}\]
		
		
		\item \textbf{Precision} is the proportion of predicted positives that are correct
		\[
		\mbox{Precision} = \mbox{Positive Predictive Value} =\frac{TP}{TP+FP} \, \]
		
		\item \textbf{Negative Predictive Value} is the  fraction of predicted negatives that are correct
		\[\mbox{Negative Predictive Value} = \frac{TN}{TN+FN}\]
		
		\item \textbf{Recall} is the fraction of observations that are actually 1 with a correct predicted classification
		\[ 
		\mbox{Recall} = \mbox{Sensitivity} = \frac{TP}{TP+FN} \,  \]
		
		\item \textbf{Specificity} is the fraction of observations that are actually 0 with a correct predicted classification
		\[ \mbox{Specificity} = \frac{TN}{TN+FP} \]
		
	\end{itemize}

%================================================================================== %
\newpage	
\section{Machine learning: the problem setting}
	
	\noindent In general, a learning problem considers a set of $n$ samples of data and try to predict properties of unknown data. If
	each sample is more than a single number, and for instance a multi-dimensional entry (aka multivariate data), is it said
	to have several variables, also known as attributes or \textbf{\textit{features}}.\\
	\bigskip
	We can separate learning problems in a few large categories:
	
	\begin{itemize}
		\item \textbf{Supervised learning}, in which the data comes with additional attributes that we want to predict.\\ 
		%(Click here to go to the Scikit-Learn supervised learning page).
		\bigskip
		This problem can be either:
		\begin{description}
			\item[Classification:] samples belong to two or more classes and we want to learn from already labeled data how
			to predict the class of unlabeled data. \\ An example of classification problem would be the digit recognition
			example, in which the aim is to assign each input vector to one of a finite number of discrete categories.
			\item[Regression:] if the desired output consists of one or more continuous variables, then the task is called
			regression. \\ An example of a regression problem would be the prediction of the weight of a pony as a
			function of its age and height.
		\end{description}
		
		\newpage
		\item  \textbf{Unsupervised learning}, in which the training data consists of a set of input vectors $x$ without any corresponding
		target values. \\ \bigskip The goal in such problems may be
		\begin{itemize}
			\item to discover groups of similar examples within the data, where
			it is called \textbf{\textit{clustering}}, \item to determine the distribution of data within the input space, known as \textbf{\textit{density estimation}},
			\item to project the data from a high-dimensional space down to two or thee dimensions for the purpose of
			visualization 
		\end{itemize}
		% (Click here to go to the Scikit-Learn unsupervised learning page).
	\end{itemize}
	
	%======================================================================== %
\newpage

\section{Training and validation}
%http://www.jmp.com/support/help/Validation_2.shtml
Using Validation and Test Data

%When you have sufficient data, you can subdivide your data into three parts called the training, validation, and test data. During the selection process, models are fit on the training data, and the prediction error for the models so obtained is found by using the validation data. This prediction error on the validation data can be used to decide when to terminate the selection process or to decide what effects to include as the selection process proceeds. Finally, once a selected model has been obtained, the test set can be used to assess how the selected model generalizes on data that played no role in selecting the model.

In some cases you might want to use only training and test data. For example, you might decide to use an information criterion to decide what effects to include and when to terminate the selection process. In this case no validation data are required, but test data can still be useful in assessing the predictive performance of the selected model. In other cases you might decide to use validation data during the selection process but forgo assessing the selected model on test data. 
%Hastie, Tibshirani, and Friedman (2001) note that it is difficult to give a general rule on how many observations you should assign to each role. They note that a typical split might be 50\% for training and 25% each for validation and testing.



%---------------------------%
\section{Supervised learning}


Supervised learning is the machine learning task of inferring a function from supervised training data. The training data consist of a set of training examples. In supervised learning, each example is a pair consisting of an input object (typically a vector) and a desired output value (also called the supervisory signal). A supervised learning algorithm analyzes the training data and produces an inferred function, which is called a classifier (if the output is discrete, see classification) or a regression function (if the output is continuous, see regression). The inferred function should predict the correct output value for any valid input object. This requires the learning algorithm to generalize from the training data to unseen situations 

in a "reasonable" way (see inductive bias). 



%%%%%%%%%%%%%%%%%%%%%%%%%%%%%%%%%%%%%%%%%%%%%%%%%%%%%%%%%%%%%%%%%%%%%%%%%%%%%%%%%%%%%%%%%%%%%%%%%%%%
%%%%%%%%%%%%%%%%%%%%%%%%%%%%%%%%%%%%%%%%%%%%%%%%%%%%


\newpage
\section{Week 6 General Theory Topics}


\subsection{Steps in Building a Predictive Model}
\begin{enumerate}
\item Find the right data
\item Define your error rate
\item Split data into:
\begin{itemize}
\item \textbf{Training Set}
\item \textbf{Testing Set}
\item \textbf{Validation Set} (optional)
\end{itemize}
\item On the training set select predictor variables (features)
\item On the training set generate your predictive model
\item On the training set cross-validate

%\item If no validation - apply 1x to test set
%\item If validation - apply to test set and refine
%\item If validation - apply 1x to validation
\end{enumerate}
%-------------------------------------------------------%
\subsection{Descriptive vs Predictive Models}

\begin{itemize}
	\item A \textbf{descriptive model} is only concerned with modeling the structure in the observed data. It makes sense to train and evaluate it on the same dataset.
	
	\item The \textbf{predictive model} is attempting a much more difficult problem, approximating the true discrimination function from a sample of data. We want to use algorithms that do not pick out and model all of the noise in our sample. We do want to chose algorithms that generalize beyond the observed data. It makes sense that we could only evaluate the ability of the model to generalize from a data sample on data that it had not see before during training.
	
	\item \textbf{IMPORTANT} The best descriptive model is accurate on the observed data. The best predictive model is accurate on unobserved data.
\end{itemize}




\subsection{Cross-Validation and Testing}
%====================================================%
\begin{itemize}
	\item In order to build the best possible mode, we will split our training data into two parts: a training set and a test set. 
	
	\item 	The general idea is as follows. The model parameters (the regression coefficients) are learned using the training set as above. 
	\item The error is evaluated on the test set, and the meta-parameters are adjusted so that this cross-validation error is minimized. 

\end{itemize}	
\subsection{Cross Validation}
%-------------------------------------------------------------------------------------%
\begin{itemize}	
%	\item 	
%	
%	\item	The cross validated set of data is a more honest presentation of the power of the
%	discriminant function than that provided by the original classifications and often produces
%	a poorer outcome. 
	\item The cross validation is often termed a ‘jack-knife’ classification, in that
	it successively classifies \textbf{all cases but one} to develop a predictive model and then
	categorizes the case that was left out. This process is repeated with each case left out in
	turn.This is known as leave-1-out cross validation. 
	
	\item 	This cross validation produces a more reliable function. The argument behind it is that
	one should not use the case you are trying to predict as part of the categorization process.
\end{itemize}



%-----------------------------------------------------------------------------------%
\subsection{Error Rates}

\begin{itemize}
	\item We can evaluate error rates by means of a training sample (to construct build a model) and a test sample.
	
	
	\item 	An optimistic error rate is obtained by reclassifying the training data. (In the \textbf{\textit{training data}} sets, how many cases were misclassified). This is known as the \textbf{apparent error rate}.
	
	
	\item 	The apparent error rate is obtained by using in the training set to estimate
	the error rates. It can be severely optimistically biased, particularly for complex classifiers, and in the presence of over-fitted models.
	
	
	\item	If an independent test sample is used for classifying, we arrive at the  \textbf{true error rate}.The true error rate (or conditional error rate) of a classifier is the expected
	probability of misclassifying a randomly selected pattern.
	It is the error rate of an infinitely large test set drawn from the same distribution as the training data.
\end{itemize}




%---------------------------------------------------------------------------------------%




\subsection{Cross Validation}
\begin{itemize}
\item In a prediction problem, a model is usually given a dataset of known data 
on which training is run (\textit{training dataset}), and a dataset of unknown data (or \textit{first seen data/ testing dataset}) against which testing the model is performed.
\item Cross-validation is mainly used in settings where the goal is prediction, and one wants to estimate how accurately a predictive model will perform in practice, with unseen data.
\item The goal of cross validation is to define a dataset to ``test" the model in the training phase, in order to limit problems like overfitting, give an insight on how the model will generalize to an independent data set (i.e., an unknown dataset, for instance from a real problem), etc.
\item Cross-validation is important in guarding against testing hypotheses suggested by the data (called ``\textbf{\textit{Type III errors}}"), especially where further samples 
are hazardous, costly or impossible to collect 
\end{itemize}
\subsection*{K-fold Cross Validation}
\begin{itemize}
\item In k-fold cross-validation, the original data set is randomly partitioned into $k$ equally sized subsamples (e.g. 10 samples).
 
\item Of the $k$ subsamples, a single subsample is retained as the testing data for testing the model, and the remaining k - 1 subsamples are used as training data. 
\item The cross-validation process is then repeated k times (the folds), with each of the $k$ subsamples used exactly once as the test data. \item The $k$ results from the folds can then be averaged (or otherwise combined) to produce a single estimation.
\item The advantage of this method over repeated random sub-sampling is that all observations are used for both training and testing, and each observation is used for testing exactly once. 
\end{itemize}
%\subsection*{Choosing K - Bias and Variance}
%In general, when using k-fold cross validation, it seems to be the case that:
%\begin{itemize}
%\item A larger k will produce an estimate with smaller bias but potentially higher variance (on top of being computationally expensive)
%\item A smaller k will lead to a smaller variance but may lead to a a biased estimate.
%\end{itemize}
\newpage

\subsection*{Leave-One-Out Cross-Validation}
\begin{itemize}
\item As the name suggests, \textbf{leave-one-out cross-validation}  \textbf{(LOOCV)} involves using a single observation from the original sample as the validation data, and the remaining observations as the training data. 
\item This is repeated such that each observation in the sample is used once as the validation data. 
\item This is the same as a K-fold cross-validation with K being equal to the number of observations in the original sampling, i.e. \textbf{K=n}.
\end{itemize}



%-------------------------------------------------------%


\subsection{Binary Classification}
\noindent \textbf{Defining True/False Positives}
In general, Positive = identified and negative = rejected. Therefore:

\begin{itemize}
	\item True positive = correctly identified
	
	\item False positive = incorrectly identified
	
	\item True negative = correctly rejected
	
	\item False negative = incorrectly rejected
\end{itemize}
\subsubsection*{Medical Testing Example:}
\begin{itemize}
	\item True positive = Sick people correctly diagnosed as sick
	
	\item False positive= Healthy people incorrectly identified as sick
	
	\item True negative = Healthy people correctly identified as healthy
	
	\item False negative = Sick people incorrectly identified as healthy.
\end{itemize}
%-------------------------------------------------- %
\newpage
\subsection{Definitions (From Week 1)}
\textbf{Confusion Matrix} \\
The confusion
table is a table in which the rows are the observed categories of the dependent and
the columns are the predicted categories. When prediction is perfect all cases will lie on the
diagonal. The percentage of cases on the diagonal is the percentage of correct classifications. 

\textbf{Accuracy Rate}\\
The accuracy rate calculates the proportion ofobservations being allocated to the \textbf{correct} group by the predictive model. It is calculated as follows:
\[ \frac{
\mbox{Number of Correct Classifications }}{\mbox{Total Number of Classifications }} \]

\[ = \frac{TP + TN}{TP+FP+TN+FN}\]

\medskip

\noindent \textbf{Misclassification Rate}\\
The misclassification rate calculates the proportion ofobservations being allocated to the \textbf{incorrect} group by the predictive model. It is calculated as follows:
\[ \frac{
\mbox{Number of Incorrect Classifications }}{\mbox{Total Number of Classifications }} \]

\[ = \frac{FP + FN}{TP+FP+TN+FN}\]
\newpage

\subsection{Misclassification Cost}
\begin{itemize}
	\item As in all statistical procedures it is helpful to use diagnostic procedures to assess the efficacy of the analysis. We use \textbf{cross-validation} to assess the classification probability.
	Typically you are going to have some prior rule as to what is an \textbf{acceptable misclassification rate}.
	
	\item	Those rules might involve things like, \textit{``what is the cost of misclassification?"} Consider a medical study where you might be able to diagnose cancer.
	
	\item There are really two alternative costs. 
	\begin{itemize}
		\item[$\ast$] The cost of misclassifying someone as having cancer when they don't.
	This could cause a certain amount of emotional grief. Additionally there would be the substantial cost of unnecessary treatment.
	
	\item[$\ast$] There is also the alternative cost of misclassifying someone as not having cancer when in fact they do have it.
	\end{itemize}
	\item A good classification procedure should
	\begin{itemize}
		\item[$\ast$] result in few misclassifications
		\item[$\ast$] take \textbf{\textit{prior probabilities of occurrence}} into account
		\item[$\ast$] consider the cost of misclassification
	\end{itemize}
	
	\item 	For example, suppose there tend to be more financially sound firms than bankrupt
	firm. If we really believe that the prior probability of a financially
	distressed and ultimately bankrupted firm is very small, then one should
	classify a randomly selected firm as non-bankrupt unless the data
	overwhelmingly favor bankruptcy.
	
	
	
	\item 	There are two costs associated with discriminant analysis classification: The true misclassification cost per class, and the expected misclassification cost (ECM) per observation.
	
	\item 	\textbf{Example} Suppose there we have a binary classification system, with two classes: class 1 and class 2.
	Suppose that classifying a class 1 object as belonging to class 2 represents a more serious error than classifying a class 2 object as belonging to class 1. There would an assignable cost to each error.
	\textbf{c(i$|$j)} is the cost of classifying an observation into class $j$ if its true class is $i$.
	The costs of misclassification can be defined by a cost matrix.
	
	\begin{center}
	\begin{tabular}{|c|c|c|}
		\hline
		% after \\: \hline or \cline{col1-col2} \cline{col3-col4} ...
		& Predicted & Predicted \\
		& Class 1 & Class 2 \\  \hline
		Class 1 & 0 & $c(2|1)$  \\ \hline
		Class 2 & $c(1|2)$ & 0 \\
		\hline
	\end{tabular}
	\end{center}
	
\end{itemize}

\noindent \textbf{Expected cost of misclassification (ECM)}
\begin{itemize}
	\item Let $p_1$ and $p_2$ be the prior probability of class 1 and class 2 respectively.
	Necessarily $p_1$ + $p_2$ = 1.
	
\item	The conditional probability of classifying an object as class 1 when it is in fact from
	class 2 is denoted $p(1|2)$.
\item 	Similarly the conditional probability of classifying an object as class 2 when it is in
	fact from class 1 is denoted $p(2|1)$.
	
	\[ECM = c(2|1)p(2|1)p_1 + c(1|2)p(1|2)p_2\]
\textit{(In other words: the sum of the cost of misclassification times the (joint) probability of that misclassification.)}
	
\item 	A reasonable classification rule should have ECM as small as possible.
\end{itemize}







\end{document}

\end{document}

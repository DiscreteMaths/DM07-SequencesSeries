\documentclass[]{report}

\voffset=-1.5cm
\oddsidemargin=0.0cm
\textwidth = 480pt

\usepackage{framed}
\usepackage{subfiles}
\usepackage{enumerate}
\usepackage{graphics}
\usepackage{newlfont}
\usepackage{eurosym}
\usepackage{amsmath,amsthm,amsfonts}
\usepackage{amsmath}
\usepackage{color}
\usepackage{amssymb}
\usepackage{multicol}
\usepackage[dvipsnames]{xcolor}
\usepackage{graphicx}
\begin{document}

\subsection{Cross Validation}

Cross-validation, sometimes called rotation estimation,is a model validation technique for assessing how the results of a statistical analysis will generalize to an independent data set.
\begin{itemize}
\item It is mainly used in settings where the goal is prediction, and one wants to estimate how accurately a predictive model will perform in practice. It is worth highlighting that in a prediction problem, a model is usually given a dataset of known data on which training is run (training dataset), and a dataset of unknown data (or first seen data) against which the model is tested (testing dataset).
\item The goal of cross validation is to define a dataset to "test" the model in the training phase (i.e., the validation dataset), in order to limit problems like overfitting, give an insight on how the model will generalize to an independent data set (i.e., an unknown dataset, for instance from a real problem), etc.
\end{itemize}



%---------------------------%
\section{Training data sets}

A training set is a set of data used in various areas of information science to discover potentially predictive relationships. Training sets are used in artificial intelligence, machine learning, genetic programming, intelligent systems, and statistics. In all these fields, a training set has much the same role and is often used in conjunction with a test set.

\end{document}
%=====================%

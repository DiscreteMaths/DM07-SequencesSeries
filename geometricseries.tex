
\documentclass[12pt]{article}
%\usepackage[final]{pdfpages}

\usepackage{graphicx}
\graphicspath{{/Users/kevinhayes/Documents/teaching/images/}}

\usepackage{tikz}
\usetikzlibrary{arrows}

\newcommand{\bbr}{\Bbb{R}}
\newcommand{\zn}{\Bbb{Z}^n}

%\usepackage{epsfig}
%\usepackage{subfigure}
\usepackage{amscd}
\usepackage{amssymb}
\usepackage{amsbsy}
\usepackage{amsthm}
\usepackage{natbib}
\usepackage{amsbsy}
\usepackage{enumerate}
\usepackage{amsmath}
\usepackage{eurosym}
%\usepackage{beamerarticle}
\usepackage{txfonts}
\usepackage{fancyvrb}
\usepackage{fancyhdr}
\usepackage{natbib}
\bibliographystyle{chicago}

\usepackage{vmargin}
% left top textwidth textheight headheight
% headsep footheight footskip
\setmargins{2.0cm}{2.5cm}{16 cm}{22cm}{0.5cm}{0cm}{1cm}{1cm}
\renewcommand{\baselinestretch}{1.3}


\pagenumbering{arabic}

\begin{document}


\section{Geometric Series}

%% ------0.5cm}
\textbf{Important Equations}
\begin{itemize}
\item Summation of $n$ terms
{

\[ S_n = \frac{a(1-r)^n}{1-r} \]
}
\item Sum to infinity of a geometric series
{

\[ \mbox{ when } 0 < r < 1 : S_{\infty} = \frac{a}{1-r} \]
}
\end{itemize}


%----------------------------------%
{Geometric Series}{Example 1}


\[ \frac{1}{2} + \frac{1}{4} + \frac{1}{8} +  \frac{1}{16} +\ldots  \]

Write down $S_n$ and $S_{\infty}$ of the infinite geometric series.
\[ 0.7 + 0.07 + 0.007 + 0.0007 + \ldots  \]




Write down $S_n$ and $S_{\infty}$ of the infinite geometric series.
\[ 1 + \frac{3}{4} + \left( \frac{3}{4} \right)^2 + \left( \frac{3}{4} \right)^3 + \ldots  \]

\noindent\textbf{Summations}


Find $S_n$, the sum of $n$ terms, of the geometric series

\[  2 + \frac{2}{3} + \frac{2}{3^2} + \frac{2}{3^3} +  \ldots + \frac{2}{3^{n-1}} \]
\bigskip
If $S_n$ = 242/81, find the value of $n$.



%----------------------------------------%


\noindent\textbf{Summations}

%% ------0.5cm}
\[  2 + \frac{2}{3} + \frac{2}{3^2} + \frac{2}{3^3} +  \ldots + \frac{2}{3^{n-1}} \]
\[ \phantom{ 2 \times \left[ 1 + \frac{1}{3} + \frac{1}{3^2} + \frac{1}{3^3} +  \ldots + \frac{1}{3^{n-1}}   \right]  } \]


\textbf{Summation Theorem}

\[ \sum^{n}_{r=0} x^r = \frac{x^{n+1}-1}{x-1} \]
\[ \phantom{k \sum^{n}_{r=0} x^r = k \frac{x^{n+1}-1}{x-1}  } \]
\phantom{Here $k=2$ and $x = 1/3$ }




%----------------------------------------%


\noindent\textbf{Summations}

%% ------0.5cm}
\[  2 + \frac{2}{3} + \frac{2}{3^2} + \frac{2}{3^3} +  \ldots + \frac{2}{3^{n-1}} \]
\[  2 \times \left[ 1 + \frac{1}{3} + \frac{1}{3^2} + \frac{1}{3^3} +  \ldots + \frac{1}{3^{n-1}}   \right] \]

\textbf{Summation Theorem}

\[ \sum^{n}_{r=0} x^r = \frac{x^{n+1}-1}{x-1} \]
\[ k  \sum^{n}_{r=0} x^r  =  k \left( \frac{x^{n+1}-1}{x-1} \right) \]
Here $k=2$ and $x = 1/3$ 



%----------------------------------------%


\noindent\textbf{Summations}

%% ------0.5cm}

\[ k  \sum^{n}_{r=0} x^r  =  k \left( \frac{x^{n+1}-1}{x-1} \right) \]
Here $k=2$ and $x = 1/3$ 
\[  \phantom{ 2  \sum^{n}_{r=0} (1/3)^r  =  2 \left( \frac{(1/3)^{n+1}-1}{(1/3)-1} \right) } \]



%----------------------------------------%


\noindent\textbf{Summations}

%% ------0.5cm}

\[ k  \sum^{n}_{r=0} x^r  =  k \left( \frac{x^{n+1}-1}{x-1} \right) \]
Here $k=2$ and $x = 1/3$ 
\[  2  \sum^{n}_{r=0} (1/3)^r  =  2 \left( \frac{(1/3)^{n+1}-1}{(1/3)-1} \right)  = \frac{242}{81} \]



%----------------------------------------%


\noindent\textbf{Summations}

%% ------0.5cm}
\[    2 \left( \frac{(1/3)^{n+1}-1}{(1/3)-1} \right)  = \frac{-3}{4} \left[ (1/3)^{n+1}-1 \right]  = \frac{242}{81} \]

\[     \frac{-3}{4} \left[ (1/3)^{n+1}-1 \right]  = \frac{242}{81} \]

\[      \left[ (1/3)^{n+1}-1 \right]  =  \frac{-4}{3} \times \frac{242}{81} \]

\newpage


\pagebreak
\subsection*{Geometric Series}
A \emph{geometric sequence} $a_n=a_0 r^n$ is a sequence whose terms are multiplied constant ratio $r$ at each step. A \emph{geometric series} $s_n$ is the sum of a geometric sequence.
\begin{equation*}
  s_n=\sum_{k=0}^n a_0 r^n =  a_0 \sum_{k=0}^n r^n = a_0 + a_0r +a_0 r^2+a_0 r^3 + \cdots 
\end{equation*}
\noindent \textbf{Example:} Let $a_n=\left(\frac{1}{2}\right)^n$.
\begin{center}
\begin{tabular}{c|c|c|c|c|c|c}
$n$ & $0$ & $1$ & $2$ & $3$ & $4$ & $5$   \\ \hline
$a_n$ & $1$ & $\frac{1}{2}$ & $\frac{1}{4}$ & $\frac{1}{8}$ & $\frac{1}{16}$ & $\frac{1}{32}$  \\ \hline
$s_n$ & $1$ & $1.5$ & $1.75$ & $1.875$ & $1.9375$ & $1.96875$ 
\end{tabular}
\end{center}
As with arithmetic series, instead of having to sum up every term in a geometric series, a formula for the $n^{th}$ term in the series can be derived.
\begin{thm*}
  For a geometric series
  \begin{equation*}
    s_n = a_0 \sum_{k=0}^n r^n= a_0 \left( \frac{1-r^{n+1}}{1-r} \right)
  \end{equation*}
\begin{proof}
To prove this, we need a formula for the sum $S_n=\sum_{k=0}^n r^k$. We proceed proceed by writing out the sum in full, then the sum multiplied by $r$. These two sums are subtracted, resulting in most terms cancelling out.
  \begin{align*}
    S_n&=1+r+r^2+r^3+\cdots+r^{n-1}+r^n\\
    r S_n&=r+r^2+r^3+r^4+\cdots+r+r^{n+1}\\
    S_n-r S_n&= 1 + (r - r) + (r^2-^2)+\cdots+(r^n-r^n)-r^{n+1}
  \end{align*}
The bracketed terms cancel and so
\begin{equation*}
  S_n(1-r)=1 -r^{n+1}, \qquad \Rightarrow S_n=\frac{1-r^{n+1}}{1-r}
\end{equation*}
Multiplying by $a_0$ now gives the required result.
\end{proof}
\end{thm*}
\noindent  \textbf{Example:}
For $a_n=\left(\frac{1}{2}\right)^n$, the formula gives
\begin{equation*}
  s_5=(1)\frac{1-\frac{1}{2^{5+1}}}{1-\frac{1}{2}}=\frac{1-\frac{1}{64}}{\frac{1}{2}}=2\frac{63}{64}=1.96875
\end{equation*}
Which matches what was calculated for $s_5$ in the table above. Further terms can be calculated without summation. For example
\begin{equation*}
  s_{50}=(1)\frac{1-\frac{1}{2^{50+1}}}{1-\frac{1}{2}}=\frac{1-\frac{1}{2^{51}}}{\frac{1}{2}} \cong=1.99999999999999911182
\end{equation*}

%----------------------------------------%

\end{document}

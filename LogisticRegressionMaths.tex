\documentclass[12pt]{article}
\usepackage{amsmath}
\usepackage{amssymb}
\usepackage{framed}
\usepackage{graphicx}




\begin{document}
\author{Kevin O'Brien}
\title{Cluster Analysis - Distance Measures and Proximity Matrices}
\tableofcontents




%--------------------------------------------------------------------------------------%
\newpage
\section{Logistic Regression: Odds Ratios and Log-Odds}
Suppose that in a sample of 100 men, 90 drank wine in the previous week, while in a sample of 100 women only 20 drank wine in the same period. The odds of a man drinking wine are 90 to 10, or 9:1, while the odds of a woman drinking wine are only 20 to 80, or 1:4 = 0.25:1. The odds ratio is thus 9/0.25, or 36, showing that men are much more likely to drink wine than women. The detailed calculation is:

\[ { 0.9/0.1 \over 0.2/0.8}=\frac{\;0.9\times 0.8\;}{\;0.1\times 0.2\;} ={0.72 \over 0.02} = 36 \]

This example also shows how odds ratios are sometimes sensitive in stating relative positions: in this sample men are 90/20 = 4.5 times more likely to have drunk wine than women, but have 36 times the odds. 


The logarithm of the odds ratio, the difference of the logits of the probabilities, tempers this effect, and also makes the measure symmetric with respect to the ordering of groups. For example, using natural logarithms, an odds ratio of 36/1 maps to 3.584, and an odds ratio of 1/36 maps to -3.584.


\section{Logistic Regression: Logits}
%http://data.princeton.edu/wws509/notes/c3.pdf

The logit transformation is given by the following formula: 
\[ \eta_i = \mbox{logit}(\pi_i)  = \mbox{log}\left( \frac{\pi_i}{1- \pi_i} \right) \]

To inverse of the logit transformation is given by the following formula: 
\[ \pi_i = \mbox{logit}^{-1}(\eta_i)  =  \frac{e^{\eta_i}}{1 + e^{\eta_i}} \]

%---------------------------%
\subsection{Example 1}
Given that $\pi_i = 0.2$, compute $\eta_i$.

\[ \eta_i = \mbox{log}\left( \frac{0.2}{1-0.2} \right)= \mbox{log}\left( \frac{0.2}{0.8} \right)\] 

\[ \eta_i =  \mbox{log}(0.25) =-1.386 \]

%---------------------------%
\subsection{Example 2}
Given that $\eta_i = 2.3$, compute $\pi_i$.

\[ \pi_i  =  \frac{e^{2.3}}{1 + e^{2.3}} = \frac{9.974}{1 + 9.974} = 0.908 \]

%--------------------------------------------------------------------------------------%
\section{Logistic Regression}
In logistic regression, the logit may be computed in a manner similar to linear regression:
\[ \eta_i = \beta_0 + \beta_1x_1 + \beta_2x_2 + \ldots  \]

%---------------------------%
\subsection{Example 2}
Let us suppose that the probability of survival of a marine species of fauna is dependent on pollution, depth and water temperature. Suppose the logit for the logistic regression was computed as follows:
\[ \eta_i = 0.14 + 0.76x_1 - 0.093x_2 + 1.2x_3  \]
\begin{center}
\begin{tabular}{|c|c|c|}
  \hline
  % after \\: \hline or \cline{col1-col2} \cline{col3-col4} ...
Variables & case 1 & case 2 \\ \hline
Pollution($x_1$) & 6.0 & 1.9 \\
Depth ($x_2$)& 51 & 99 \\
Temp ($x_3$) & 3.0 & 2.9 \\
  \hline
\end{tabular}
\end{center}
Compute the probability of success for both case 1 and case 2.

\begin{itemize}
\item case 1$ \eta_1 = 0.14 + (0.76 \times 6)	- (0.093\times 51) + (1.2\times 3) = 3.557$
\item case 2$ \eta_2 = 0.14 + (0.76 \times 1.9)	- (0.093\times 99) + (1.2\times 2.9) = -4.143$
\end{itemize}

The probabilities for success are therefore:
\[ \pi_1  =  \frac{e^{3.557}}{1 + e^{3.557}} = \frac{35.057}{1 + 35.057} = 0.972 \]
\[ \pi_2  =  \frac{e^{-4.143}}{1 + e^{-4.143}} = \frac{0.0158}{1 + 0.0158} = 0.0156 \]
\end{document}

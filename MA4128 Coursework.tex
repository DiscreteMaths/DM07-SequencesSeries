
\newpage
\section{MANOVA}
\subsection{Example of MANOVA}
%https://statistics.laerd.com/spss-tutorials/one-way-manova-using-spss-statistics.php
The pupils at a high school come from three different primary schools. The headteacher wanted to know whether there were academic differences between the pupils from the three different primary schools. As such, she randomly selected 20 pupils from School A, 20 pupils from School B and 20 pupils from School C, and measured their academic performance as assessed by the marks they received for their end-of-year English and Maths exams. Therefore, the two dependent variables were "English score" and "Maths score", whilst the independent variable was "School", which consisted of three categories: "School A", "School B" and "School C".



\subsection{Wilk's Lambda}
% http://www.blackwellpublishing.com/specialarticles/jcn_9_381.pdf

Wilks' lambda is a test statistic used in multivariate analysis of variance
(MANOVA) to test whether there are differences between the means of
identified groups of subjects on a combination of dependent variables. For
example, in the paper above, the authors test whether the mean score of two
groups, graduates and diplomates, is the same across eight constructs
simultaneously. Thus, they are considering eight dependent variables and
comparing the mean of this combination for two groups.

Wilks' lambda performs, in the multivariate setting, with a combination of
dependent variables, the same role as the F-test performs in one-way analysis
of variance. 

Wilks' lambda is a direct measure of the proportion of variance in
the combination of dependent variables that is unaccounted for by the
independent variable (the grouping variable or factor). If a large proportion
of the variance is accounted for by the independent variable then it suggests
that there is an effect from the grouping variable and that the groups (in this
case the graduates and diplomates) have different mean values.

Wilks' lambda statistic can be transformed (mathematically adjusted) to a
statistic which has approximately an F distribution. This makes it easier to
calculate the P-value. Often authors will present the F-value and degrees of
freedom, as in the above paper, rather than giving the actual value of Wilks'
lambda.
There

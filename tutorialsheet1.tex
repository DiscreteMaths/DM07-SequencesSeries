\documentclass[]{report}

\voffset=-1.5cm
\oddsidemargin=0.0cm
\textwidth = 480pt

\usepackage{framed}
\usepackage{subfiles}
\usepackage{enumerate}
\usepackage{graphics}
\usepackage{newlfont}
\usepackage{eurosym}
\usepackage{amsmath,amsthm,amsfonts}
\usepackage{amsmath}
\usepackage{color}
\usepackage{amssymb}
\usepackage{multicol}
\usepackage[dvipsnames]{xcolor}
\usepackage{graphicx}
\begin{document}

\section*{Tutorial Sheet 1}
\begin{enumerate}
\item 
Compute the Euclidean Distance between the following points:
$X = \{1,5,4,3\}$ and $Y = \{2,1,8,7\}$

\begin{center}
	\begin{tabular}{|c|c|c|c|}
		\hline
		$x_j$	&	$y_j$	&   $x_j - y_j$	&	$(x_j - y_j)^2$	\\ \hline
		1	&	2	&	-1	&	1	\\
		5	&	1	&	4	&	16	\\
		4	&	8	&	-4	&	16	\\
		3	&	7	&	-4	&	16	\\ \hline
		&		&		&	49	\\ \hline
	\end{tabular}
\end{center}
The Euclidean Distance between the two points is $\sqrt{49}$ i.e. 7.
%--------------------------------------------------------------------------------------%





\item  \textbf{Example}\\
Compute the Manhattan Distance between the following points: 
$X = \{1,3,4,2\}$ and $Y = \{5,2,5,2\}$, based on four numeric variables.


\begin{center}
	\begin{tabular}{|c|c|c|c|c|}
		\hline
	&	Case $X$	&	Case $Y$	&   Difference	&	$| \mbox{Diff} |$	\\ \hline
Variable 1	&	1	&	5	&	-4	&	4	\\
Variable 2	&	3	&	2	&	1	&	1	\\
Variable 3	&	4	&	5	&	-1	&	1	\\
Variable 4	&	2	&	2	&	0	&	0	\\ \hline
		& & && 6 \\
		\hline
	\end{tabular}
\end{center}
\begin{itemize}
\item The Manhattan Distance between the two points is 6.
\end{itemize}
%====================%

\item 
Numeric Transformations, such as logarithmic transformation, are often used in statistical analysis as an approach for dealing with non-normal data.
\begin{itemize}
	%	\item[(i)] (1 Marks) Discuss the importance of numeric transformations, such as logarithmic transformation, in Statistics.
	%	\item[(ii)] Describe the process of transformations
	\item[(i.)] (1 Mark) Describe the purpose of Tukey's Ladder (referencing direction and relative strength).
	\item[(ii.)] (2 Marks) Give two examples of a transformation for various types of skewed data (i.e. an example for both types of skewness).
	\item[(iii.)] (1 Mark) Discuss the limitations of numeric transformations.
\end{itemize}
\bigskip

\item 
The typing speeds for one group of 12 Engineering students were recorded both at the beginning of year 1 of their studies. The results (in words per minute) are given below:

\begin{center}
	\begin{tabular}{|c|c|c|c|c|c|}
		\hline
		% Subject& A& B& C& D& E &F &G &H \\ \hline
		149  & 146 & 112 & 142 & 168& 153\\ \hline
		137 & 161 & 156& 165&  170&  159
		\\ \hline
	\end{tabular}
\end{center}
Use the Dixon Q-test to determine if the lowest value (118) is an outlier. You may assume a significance level of 5\%.
\begin{itemize}
	\item[(i.)](1 Mark)	State the Null and Alternative Hypothesis for this test.
	\item[(ii.)](2 Marks) Compute the test statistic
	\item[(iii.)](1 Mark) State the appropriate critical value.
	\item[(iv.)](1 Mark) What is your conclusion to this procedure.
\end{itemize}

%============================%
\item [\textbf{Outliers}]
\begin{itemize}
	\item[(i.)] (3 Marks) Provide a brief description for three tests from the family of Grubb's  Outliers Tests. Include in your description a statement of the null and alternative hypothesis for each test
	\item[(ii.)] (2 Marks) Describe any required assumptions for tests, and the limitations of these tests.
\end{itemize}
%============================%
\item [\textbf{Model Selection}]
$x_1$, $x_2$,$x_3$ and $x_4$.

 Suppose we have 5 predictor variables.
Use \textbf{Forward Selection} and \textbf{Backward Selection} to choose the optimal set of predictor variables, based on the AIC measure.

{
	\large
	\begin{center}
		\begin{tabular}{||c|c||c|c||}
			\hline
			Variables & AIC & Variables & AIC \\ \hline \hline
			$\emptyset$	&	200	&	x1, x2, x3	&	74	\\ \hline
			\phantom{makemakespace}
			&	\phantom{makespace}
			&	x1, x2, x4	&	75	\\ \hline
			x1	&	150	&	x1, x2, x5	&	79	\\ \hline
			x2	&	145	&	x1, x3, x4	&	72	\\ \hline
			x3	&	135	&	x1, x3, x5	&	85	\\ \hline
			x4	&	136	&	x1, x4, x5	&	95	\\ \hline
			x5	&	139	&	x2, x3, x4	&	83	\\ \hline
			&		&	x2, x3, x5	&	82	\\ \hline
			x1, x2	&	97	&	x2, x4, x5	&	78	\\ \hline
			x1, x3	&	81	&	x3, x4, x5	&	85	\\ \hline
			x1, x4	&	94	&	\phantom{makemakespace}
			&	\phantom{makespace}
			\\ \hline
			x1, x5	&	88	&	x1, x2, x3, x4	&	93	\\ \hline
			x2, x3	&	87	&	x1, x2, x3, x5	&	120	\\ \hline
			x2, x4	&	108	&	x1, x2, x4, x5	&	104	\\ \hline
			x2, x5	&	87	&	x1, x3, x4, x5	&	101	\\ \hline
			x3, x4	&	105	&	x2, x3, x4, x5	&	89	\\ \hline
			x3, x5	&	82	&		&		\\ \hline
			x4, x5	&	86	&	x1, x2, x3, x4, x5	&	100	\\ \hline
		\end{tabular} 
	\end{center}
}


%==========================================================%
\newpage
\item Model Selection Question

\begin{itemize}
\item Suppose we have 5 predictor variables.
\item Use \textbf{Forward Selection} and \textbf{Backward Selection} to choose the optimal set of Predictor Variables, based on the AIC metric.
\end{itemize}
{
	\large
	\begin{center}
\begin{tabular}{|c|c|c|c|}
	\hline
$\emptyset$	&	200	&	x1,x2,x3	&	74	\\ \hline
\phantom{makespace}
 &	\phantom{makespace}
 	&	x1,x2,x4	&	75	\\ \hline
x1	&	150	&	x1,x2,x5	&	78	\\ \hline
x2	&	170	&	x1,x3,x4	&	72	\\ \hline
x3	&	135	&	x1,x3,x5	&	82	\\ \hline
x4	&	130	&	x1,x4,x5	&	70	\\ \hline
x5	&	140	&	x2,x3,x4	&	80	\\ \hline
&		&	x2,x3,x5	&	82	\\ \hline
x1,x2	&	90	&	x2,x4,x5	&	78	\\ \hline
x1,x3	&	81	&	x3,x4,x5	&	75	\\ \hline
x1,x4	&	84	&	\phantom{makespace}
	&	\phantom{makespace}
		\\ \hline
x1,x5	&	78	&	x1,x2,x3,x4	&	83	\\ \hline
x2,x3	&	87	&	x1,x2,x3,x5	&	130	\\ \hline
x2,x4	&	78	&	x1,x2,x4,x5	&	104	\\ \hline
x2,x5	&	87	&	x1,x3,x4,x5	&	101	\\ \hline
x3,x4	&	85	&	x2,x3,x4,x5	&	89	\\ \hline
x3,x5	&	88	&		&		\\ \hline
x4,x5	&	86	&	x1,x2,x3,x4,x5	&	100	\\ \hline
		\end{tabular} 
	\end{center}
}
\end{enumerate}

\end{document}

%---------------------------%
\section{Predictive analytics}


Predictive analytics is the branch of data mining concerned with the prediction of future probabilities and trends. The central element of predictive analytics is the predictor, a variable that can be measured for an individual or other entity to predict future behavior. For example, an insurance company is likely to take into account potential driving safety predictors such as age, gender, and driving record when issuing car insurance policies.


Multiple predictors are combined into a predictive model, which, when subjected to analysis, can be used to forecast future probabilities with an acceptable level of reliability. In predictive modeling, data is collected, a statistical model is formulated, predictions are made and the model is validated (or revised) as additional data becomes available. Predictive analytics are applied to many research areas, including meteorology, security, genetics, economics, and marketing.


In a data warehouse, dirty data is a database record that contains errors. Dirty data can be caused by a number of factors including duplicate records, incomplete or outdated data, and the improper parsing of record fields from disparate systems. The Data Warehousing Institute (TDWI) estimates that dirty data costs U.S. businesses more than 600 USD billion each year.
%---------------------------%
\subsection{Predictive Model Markup Language}


The Predictive Model Markup Language (PMML) is an XML-based markup language developed by the Data Mining Group (DMG) to provide a way for applications to define models related to predictive analytics and data mining and to share those models between PMML-compliant applications.


PMML provides applications a vendor-independent method of defining models so that proprietary issues and incompatibilities are no longer a barrier to the exchange of models between applications. It allows users to develop models within one vendor's application and use other vendors' applications to visualize, analyze, evaluate or otherwise use the models. Previously, this was very difficult, but with PMML, the exchange of models between compliant applications is straightforward.

Since PMML is an XML-based standard, the specification comes in the form of an XML schema.

%---------------------------%

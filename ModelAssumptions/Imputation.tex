
% http://www.ats.ucla.edu/stat/sas/library/multipleimputation.pdf
\documentclass[a4paper,12pt]{article}
%%%%%%%%%%%%%%%%%%%%%%%%%%%%%%%%%%%%%%%%%%%%%%%%%%%%%%%%%%%%%%%%%%%%%%%%%%%%%%%%%%%%%%%%%%%%%%%%%%%%%%%%%%%%%%%%%%%%%%%%%%%%%%%%%%%%%%%%%%%%%%%%%%%%%%%%%%%%%%%%%%%%%%%%%%%%%%%%%%%%%%%%%%%%%%%%%%%%%%%%%%%%%%%%%%%%%%%%%%%%%%%%%%%%%%%%%%%%%%%%%%%%%%%%%%%%
\usepackage{eurosym}
\usepackage{vmargin}
\usepackage{amsmath}
\usepackage{graphics}
\usepackage{epsfig}
\usepackage{subfigure}
\usepackage{fancyhdr}
%\usepackage{listings}
\usepackage{framed}
\usepackage{graphicx}

\setcounter{MaxMatrixCols}{10}
%TCIDATA{OutputFilter=LATEX.DLL}
%TCIDATA{Version=5.00.0.2570}
%TCIDATA{<META NAME="SaveForMode" CONTENT="1">}
%TCIDATA{LastRevised=Wednesday, February 23, 2011 13:24:34}
%TCIDATA{<META NAME="GraphicsSave" CONTENT="32">}
%TCIDATA{Language=American English}

\pagestyle{fancy}
\setmarginsrb{20mm}{0mm}{20mm}{25mm}{12mm}{11mm}{0mm}{11mm}
\lhead{MA4128} \rhead{Mr. Kevin O'Brien}
\chead{Advanced Data Modelling}
%\input{tcilatex}


% http://www.norusis.com/pdf/SPC_v13.pdf
\begin{document}
	
\section{Imputation of Missing Data}	

%----------------------------------------------------%
\subsection{Multiple Imputation}
\textbf{\textit{Multiple imputation}} provides a useful strategy for dealing
with data sets with missing values. Instead of filling in a
single value for each missing value, the multiple
imputation procedure replaces each missing value with a
set of plausible values that represent the uncertainty about
the right value to impute. These multiply imputed data sets
are then analyzed by using standard procedures for complete
data and then combining the results from all of these analyses.
No matter which complete-data analysis is used, the process
of combining results from different imputed data sets
is essentially the same. This results in valid statistical inferences
that properly reflect the uncertainty due to missing
values.

%----------------------------------------------------%
\subsubsection{Phases of Multiple Imputation}
Multiple imputation inference involves three distinct phases:
\begin{itemize}
	\item The missing data are filled in $m$ times to generate m
	complete data sets.
	\item The $m$ complete data sets are analyzed by using
	standard procedures.
	\item The results from the $m$ complete data sets are combined
	for the inference.
\end{itemize}
%---------------------------------------------------%

\end{document}
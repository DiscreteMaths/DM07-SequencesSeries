
\documentclass[12pt]{article}
%\usepackage[final]{pdfpages}

\usepackage{graphicx}
\graphicspath{{/Users/kevinhayes/Documents/teaching/images/}}

\usepackage{tikz}
\usetikzlibrary{arrows}

\newcommand{\bbr}{\Bbb{R}}
\newcommand{\zn}{\Bbb{Z}^n}

%\usepackage{epsfig}
%\usepackage{subfigure}
\usepackage{amscd}
\usepackage{amssymb}
\usepackage{amsbsy}
\usepackage{amsthm}
\usepackage{natbib}
\usepackage{amsbsy}
\usepackage{enumerate}
\usepackage{amsmath}
\usepackage{eurosym}
%\usepackage{beamerarticle}
\usepackage{txfonts}
\usepackage{fancyvrb}
\usepackage{fancyhdr}
\usepackage{natbib}
\bibliographystyle{chicago}

\usepackage{vmargin}
% left top textwidth textheight headheight
% headsep footheight footskip
\setmargins{2.0cm}{2.5cm}{16 cm}{22cm}{0.5cm}{0cm}{1cm}{1cm}
\renewcommand{\baselinestretch}{1.3}


\pagenumbering{arabic}

\begin{document}

%------------------------------------------------------------------------------------ %
\section{Aritmetic Progressions}

An arithmetic progression (AP) or arithmetic sequence is a sequence of numbers such that the difference between the consecutive terms is constant. For instance, the sequence $5, 7, 9, 11, 13, 15 \ldots$ is an arithmetic progression with common difference of 2.

%------------------------------------------------------------------------------------ %


If the initial term of an arithmetic progression is $a_{1}$ and the common difference of successive members is d, then the $n-$th term of the sequence ($a_{n}$) is given by:
\[ a_{n}=a_{1}+(n-1)d, \] 
and in general
\[ a_{n}=a_{m}+(n-m)d.\]


%------------------------------------------------------------------------------------ %

A finite portion of an arithmetic progression is called a \textbf{finite arithmetic progression} and sometimes just called an arithmetic progression. The sum of a finite arithmetic progression is called an arithmetic series.
\newline

%------------------------------------------------------------------------------------ %


The behavior of the arithmetic progression depends on the common difference d. If the common difference is:
\begin{itemize}
\item Positive, the members (terms) will grow towards positive infinity.
\item Negative, the members (terms) will grow towards negative infinity.
\end{itemize}



%----------------------------------------------------------------------------------- %
\textbf{Summation of an Arithmetic Progression}
The sum of the members of a finite arithmetic progression is called an arithmetic series. For example, consider the sum:
\[2+5+8+11+14\]
This sum can be found quickly by taking the number n of terms being added (here 5), multiplying by the sum of the first and last number in the progression (here 2 + 14 = 16), and dividing by 2:
\[{\frac  {n(a_{1}+a_{n})}{2}}\]


In the case above, this gives:
\[2+5+8+11+14={\frac  {5(2+14)}{2}}={\frac  {5\times 16}{2}}=40.\]
This formula works for any real numbers $a_{1}$ and $a_{n}$. For example:
\[\left(-{\frac  {3}{2}}\right)+\left(-{\frac  {1}{2}}\right)+{\frac  {1}{2}}={\frac  {3\left(-{\frac  {3}{2}}+{\frac  {1}{2}}\right)}{2}}=-{\frac  {3}{2}}.\]

%------------------------------------------------------------------------------------ %

\noindent\textbf{Arithmetic Progressions}

%% ------1.5cm}
Find the sum of the arithmetic progression
{

\[ 11 + 13 + 15 + \ldots + 49 + 51 \]
}


%----------------------------------%


\noindent\textbf{Arithmetic Progressions}

%% ------0.5cm}
First recall two useful equations for working with arithmetic progressions.\\
\bigskip


For the arithmetic sequence $a,(a+d) ,(a+2d), \ldots$

\begin{itemize}
\item[(i)] $t_n$ is the $n-th$ term of series.\[ t_n= a+(n-1)d \]

\item[(ii)] $S_n$ is the sum of the first $n$ terms

\[ S_n  = \frac{n }{ 2} \left[ 2a+(n-1)d \right] \]
\end{itemize}


%----------------------------------%


\noindent\textbf{Arithmetic Progressions}

Find the sum of the arithmetic progression
{
\[ 11 + 13 + 15 + \dots + 49 + 51 \]
}
\begin{itemize}
\item Note that $a=11$ and $d=2$.
\item We need to find out what $n$ (the number of terms) is.
\item The last term is $51$.
\end{itemize}


%----------------------------------%


\noindent\textbf{Arithmetic Progressions}

%% ------2cm}
\begin{itemize}
\item The last term is $51$.
\item $t_n$ = 51 = [ 11 + (n-1)2 ] 
\item $t_n$ = 51 = 2n + 9
\item $n=21$
\end{itemize}
There are 21 terms in the series.


%----------------------------------%


\noindent\textbf{Arithmetic Progressions}

%% ------0.8cm}
\[ S_n  = {n \over 2} \left[ 2a+(n-1)d \right] \]

Recall $a=11$,$d=2$ and $n=21$


\[ \phantom{S_n  = {21 \over 2} \left[ (2.11) +[(21-1)2] \right]} \]
\[ \phantom{S_n  = 10.5 \left[ 22 + 40 \right]  = 10.5 \times 62}\]
\[ \phantom{S_n  = 651} \] 


%----------------------------------%


\noindent\textbf{Arithmetic Progressions}

%% ------0.8cm}
\[ S_n  = {n \over 2} \left[ 2a+(n-1)d \right] \]

Recall $a=11$,$d=2$ and $n=21$


\[ S_n  = {21 \over 2} \left[ (2.11) +[(21-1)2] \right] \]
\[ S_n  = 10.5 \left[ 22 + 40 \right]  = 10.5 \times 62\]
\[ S_n  = 651 \] 






% HOB book Page 393



%%- \vspace{-1.5cm}
Find the sum of the arithmetic progression
{

\[ 11 + 13 + 15 + \ldots + 49 + 51 \]
}



%%- \vspace{-0.5cm}
First recall two useful equations for working with arithmetic progressions.\\
\bigskip


For the arithmetic sequence $a,(a+d) ,(a+2d), \ldots$

\begin{itemize}
\item[(i)] $t_n$ is the $n-th$ term of series.\[ t_n= a+(n-1)d \]

\item[(ii)] $S_n$ is the sum of the first $n$ terms

\[ S_n  = {n \over 2} \left[ 2a+(n-1)d \right] \]
\end{itemize}



Find the sum of the arithmetic progression
{
\[ 11 + 13 + 15 + \dots + 49 + 51 \]
}
\begin{itemize}
\item Note that $a=11$ and $d=2$.
\item We need to find out what $n$ (the number of terms) is.
\item The last term is $51$.
\end{itemize}



%%- \vspace{-2cm}
\begin{itemize}
\item The last term is $51$.
\item $t_n$ = 51 = [ 11 + (n-1)2 ] 
\item $t_n$ = 51 = 2n + 9
\item $n=21$
\end{itemize}
There are 21 terms in the series.



%%- \vspace{-0.8cm}
\[ S_n  = {n \over 2} \left[ 2a+(n-1)d \right] \]

Recall $a=11$,$d=2$ and $n=21$


\[ \phantom{S_n  = {21 \over 2} \left[ (2.11) +[(21-1)2] \right]} \]
\[ \phantom{S_n  = 10.5 \left[ 22 + 40 \right]  = 10.5 \times 62}\]
\[ \phantom{S_n  = 651} \] 



%%- \vspace{-0.8cm}
\[ S_n  = {n \over 2} \left[ 2a+(n-1)d \right] \]

Recall $a=11$,$d=2$ and $n=21$


\[ S_n  = {21 \over 2} \left[ (2.11) +[(21-1)2] \right] \]
\[ S_n  = 10.5 \left[ 22 + 40 \right]  = 10.5 \times 62\]
\[ S_n  = 651 \] 




\section{Summations}

%%- \vspace{-2cm}
Find $S_n$, the sum of $n$ terms, of the geometric series

\[  2 + \frac{2}{3} + \frac{2}{3^2} + \frac{2}{3^3} +  \ldots + \frac{2}{3^{n-1}} \]
\bigskip
If $S_n$ = 242/81, find the value of $n$.





%%- \vspace{-0.5cm}
\[  2 + \frac{2}{3} + \frac{2}{3^2} + \frac{2}{3^3} +  \ldots + \frac{2}{3^{n-1}} \]
\[ \phantom{ 2 \times \left[ 1 + \frac{1}{3} + \frac{1}{3^2} + \frac{1}{3^3} +  \ldots + \frac{1}{3^{n-1}}   \right]  } \]


\textbf{Summation Theorem}

\[ \sum^{n}_{r=0} x^r = \frac{x^{n+1}-1}{x-1} \]
\[ \phantom{k \sum^{n}_{r=0} x^r = k \frac{x^{n+1}-1}{x-1}  } \]
\phantom{Here $k=2$ and $x = 1/3$ }






%%- \vspace{-0.5cm}
\[  2 + \frac{2}{3} + \frac{2}{3^2} + \frac{2}{3^3} +  \ldots + \frac{2}{3^{n-1}} \]
\[  2 \times \left[ 1 + \frac{1}{3} + \frac{1}{3^2} + \frac{1}{3^3} +  \ldots + \frac{1}{3^{n-1}}   \right] \]

\textbf{Summation Theorem}

\[ \sum^{n}_{r=0} x^r = \frac{x^{n+1}-1}{x-1} \]
\[ k  \sum^{n}_{r=0} x^r  =  k \left( \frac{x^{n+1}-1}{x-1} \right) \]
Here $k=2$ and $x = 1/3$ 





%%- \vspace{-0.5cm}

\[ k  \sum^{n}_{r=0} x^r  =  k \left( \frac{x^{n+1}-1}{x-1} \right) \]
Here $k=2$ and $x = 1/3$ 
\[  \phantom{ 2  \sum^{n}_{r=0} (1/3)^r  =  2 \left( \frac{(1/3)^{n+1}-1}{(1/3)-1} \right) } \]





%%- \vspace{-0.5cm}

\[ k  \sum^{n}_{r=0} x^r  =  k \left( \frac{x^{n+1}-1}{x-1} \right) \]
Here $k=2$ and $x = 1/3$ 
\[  2  \sum^{n}_{r=0} (1/3)^r  =  2 \left( \frac{(1/3)^{n+1}-1}{(1/3)-1} \right)  = \frac{242}{81} \]





%%- \vspace{-0.5cm}
\[    2 \left( \frac{(1/3)^{n+1}-1}{(1/3)-1} \right)  = \frac{-3}{4} \left[ (1/3)^{n+1}-1 \right]  = \frac{242}{81} \]

\[     \frac{-3}{4} \left[ (1/3)^{n+1}-1 \right]  = \frac{242}{81} \]

\[      \left[ (1/3)^{n+1}-1 \right]  =  \frac{-4}{3} \times \frac{242}{81} \]





{Geometric Series}

%%- \vspace{-0.5cm}
\textbf{Important Equations}
\begin{itemize}
\item Summation of $n$ terms
{

\[ S_n = \frac{a(1-r)^n}{1-r} \]
}
\item Sum to infinity of a geometric series
{

\[ \mbox{ when } 0 < r < 1 : S_{\infty} = \frac{a}{1-r} \]
}
\end{itemize}


%----------------------------------%
{Geometric Series}{Example 1}


\[ \frac{1}{2} + \frac{1}{4} + \frac{1}{8} +  \frac{1}{16} +\ldots  \]




%----------------------------------%
{Geometric Series}{Example 1}




%----------------------------------%
{Geometric Series}{Example 2}

%%- \vspace{-2cm}
Write down $S_n$ and $S_{\infty}$ of the infinite geometric series.
\[ 0.7 + 0.07 + 0.007 + 0.0007 + \ldots  \]


%----------------------------------%
{Geometric Series}{Example 2}



%----------------------------------%
{Geometric Series}{Example 3}

%%- \vspace{-2cm}
Write down $S_n$ and $S_{\infty}$ of the infinite geometric series.
\[ 1 + \frac{3}{4} + \left( \frac{3}{4} \right)^2 + \left( \frac{3}{4} \right)^3 + \ldots  \]



%----------------------------------%
{Geometric Series}{Example 3}




%Page 43
%----------------------------------------%

%% - {Summations}

Find $S_n$, the sum of $n$ terms, of the geometric series

\[  2 + \frac{2}{3} + \frac{2}{3^2} + \frac{2}{3^3} +  \ldots + \frac{2}{3^{n-1}} \]

If $S_n$ = 242/81, find the value of $n$.


%----------------------------------------%

%% - {Summations}

Summation Theorem

\[ \sum^{n}_{r=0} x^r = \frac{x^{n+1}-1}{x-1} \]

Here $x = 1/3$

\[  2 + \frac{2}{3} + \frac{2}{3^2} + \frac{2}{3^3} +  \ldots + \frac{2}{3^{n-1}} \]


%----------------------------------------%

% HOB book Page 393


%----------------------------------%

%% - {Arithmetic Progressions}


Find the sum of the arithmetic progression
{

\[ 11 + 13 + 15 + \dots + 49 + 51 \]
}

%----------------------------------%

%% - {Arithmetic Progressions}

First recall two useful equations for working with Arithmetic Progressions


For the arithmetic sequence $a,(a+d) ,(a+2d), \ldots$

\begin{itemize}
\item[(i)] $t_n$ is the $n-th$ term of series = $a+(n-1)d$

\item[(ii)] $S_n$ is the sum of the first $n$ terms

\[ S_n  = {n \over 2} \left[ 2a+(n-1)d \right] \]
\end{itemize}

%----------------------------------%

%% - {Arithmetic Progressions}

Find the sum of the arithmetic progression
{
\[ 11 + 13 + 15 + \dots + 49 + 51 \]
}

\begin{itemize}
\item $a=11$ and $d=2$
\item We need to find out what $n$ (the number of terms) is.
\item The last term is $51$
\end{itemize}

%----------------------------------%

%% - {Arithmetic Progressions}


\begin{itemize}
\item The last term is $51$.
\item $t_n$ = 51 = [ 11 + (n-1)2 ] 
\item $t_n$ = 51 = 2n + 9
\item $n=21$
\end{itemize}
There are 21 terms in the series.

%----------------------------------%

%% - {Arithmetic Progressions}


\[ S_n  = {n \over 2} \left[ 2a+(n-1)d \right] \]

Recall $a=11$,$d=2$ and $n=21$


\[ \phantom{S_n  = {21 \over 2} \left[ (2.11) +[(21-1)2] \right]} \]
\[ \phantom{S_n  = 10.5 \left[ 22 + 40 \right]  = 10.5 \times 62}\]
\[ \phantom{S_n  = 651} \] 

%----------------------------------%

%% - {Arithmetic Progressions}


\[ S_n  = {n \over 2} \left[ 2a+(n-1)d \right] \]

Recall $a=11$,$d=2$ and $n=21$


\[ S_n  = {21 \over 2} \left[ (2.11) +[(21-1)2] \right] \]
\[ S_n  = 10.5 \left[ 22 + 40 \right]  = 10.5 \times 62\]
\[ S_n  = 651 \] 

%----------------------------------%
%----------------------------------%
\newpage
%% - {Arithmetic Progressions}






%----------------------------------%

%% - {Geometric Series}{Important Equations}

{

\[ S_n = \frac{a(1-r)^n}{1-r} \]
}
{

\[ \mbox{ when } 0 < r < 1 : S_{\infty} = \frac{a}{1-r} \]
}

%----------------------------------%
%% - {Geometric Series}{Example 1}


\[ \frac{1}{2} + \frac{1}{4} + \frac{1}{8} +  \frac{1}{16} +\ldots  \]





%----------------------------------%
%% - {Geometric Series}{Example 2}

Write down $S_n$ and $S_{\infty}$ of the infinite geometric series.
\[ 0.7 + 0.07 + 0.007 + 0.0007 + \ldots  \]


%----------------------------------%
%% - {Geometric Series}{Example 2}



%----------------------------------%
%% - {Geometric Series}{Example 3}

Write down $S_n$ and $S_{\infty}$ of the infinite geometric series.
\[ 1 + \frac{3}{4} + \left( \frac{3}{4} \right)^2 + \left( \frac{3}{4} \right)^3 + \ldots  \]



\end{document}

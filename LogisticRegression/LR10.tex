\documentclass[a4paper,12pt]{article}
%%%%%%%%%%%%%%%%%%%%%%%%%%%%%%%%%%%%%%%%%%%%%%%%%%%%%%%%%%%%%%%%%%%%%%%%%%%%%%%%%%%%%%%%%%%%%%%%%%%%%%%%%%%%%%%%%%%%%%%%%%%%%%%%%%%%%%%%%%%%%%%%%%%%%%%%%%%%%%%%%%%%%%%%%%%%%%%%%%%%%%%%%%%%%%%%%%%%%%%%%%%%%%%%%%%%%%%%%%%%%%%%%%%%%%%%%%%%%%%%%%%%%%%%%%%%
\usepackage{eurosym}
\usepackage{vmargin}
\usepackage{amsmath}
\usepackage{graphics}
\usepackage{epsfig}
\usepackage{framed}
\usepackage{subfigure}
\usepackage{fancyhdr}

\setcounter{MaxMatrixCols}{10}
%TCIDATA{OutputFilter=LATEX.DLL}
%TCIDATA{Version=5.00.0.2570}
%TCIDATA{<META NAME="SaveForMode"CONTENT="1">}
%TCIDATA{LastRevised=Wednesday, February 23, 201113:24:34}
%TCIDATA{<META NAME="GraphicsSave" CONTENT="32">}
%TCIDATA{Language=American English}

\pagestyle{fancy}
\setmarginsrb{20mm}{0mm}{20mm}{25mm}{12mm}{11mm}{0mm}{11mm}
\lhead{Advanced Data Modelling} \rhead{Logistic Regression} \chead{MA4128} %\input{tcilatex}

%http://www.electronics.dit.ie/staff/ysemenova/Opto2/CO_IntroLab.pdf
\begin{document}

\tableofcontents

\section{Assumptions of logistic regression}

	
	\begin{description}
		\item[Assumption 1:] Your dependent variable should be measured on a \textbf{dichotomous scale}. Examples of dichotomous variables include gender (two groups: "males" and "females"), presence of heart disease (two groups: "yes" and "no"), personality type (two groups: "introversion" or "extroversion"), body composition (two groups: "obese" or "not obese"), and so forth. \\
		\newline
		
		
		\item[Assumption 2:] You have one or more independent variables, which can be either continuous (i.e., an interval or ratio variable) or categorical (i.e., an ordinal or nominal variable). 
		
		
		
		\item[Assumption 3:] You should have independence of observations and the dependent variable should have mutually exclusive and exhaustive categories.
		
		\item[Assumption 4:] There needs to be a linear relationship between any continuous independent variables and the logit transformation of the dependent variable. 
	\end{description}
\begin{itemize}
	\item Logistic regression does not assume a linear relationship between the dependent and
	independent variables.
	\item The dependent variable must be a dichotomy (2 categories).
	\textit{(Remark: Dichotomous refers to two outcomes. Multichotomous refers to more than two outcomes)}.
	\item The independent variables need not be interval, nor normally distributed, nor linearly
	related, nor of equal variance within each group.
	\item The categories (groups) must be mutually exclusive and exhaustive; a case can only be
	in one group and every case must be a member of one of the groups.
	\item Larger samples are needed than for linear regression because maximum likelihood
	coefficients are large sample estimates. A minimum of 50 cases per predictor is
	recommended.
\end{itemize}	
	%----------------------------------------------------------------------------------------%
	
	\begin{framed}
		\subsection*{Types of Variables (Revision)}
		\begin{itemize}
			\item Examples of \textbf{continuous variables} include revision time (measured in hours), intelligence (measured using IQ score), exam performance (measured from 0 to 100), weight (measured in kg), and so forth. 
			
			\item Examples of \textbf{ordinal variables} include \textit{Likert} items (e.g., a 7-point scale from "strongly agree" through to "strongly disagree"), amongst other ways of ranking categories (e.g., a 3-point scale explaining how much a customer liked a product, ranging from "Not very much" to "Yes, a lot"). 
			\item Examples of \textbf{nominal variables} include gender (e.g., 2 groups: male and female), ethnicity (e.g., 3 groups: Caucasian, African American and Hispanic), profession (e.g., 5 groups: surgeon, doctor, nurse, dentist, therapist), and so forth.
		\end{itemize}
	\end{framed}
	
	
	
	
	\subsection*{South Africa Heart Disease Data Example}
	
	\begin{framed}
		A retrospective sample of males in a heart-disease high-risk region
		of the Western Cape, South Africa. There are roughly two controls per
		case of CHD. Many of the CHD positive men have undergone blood
		pressure reduction treatment and other programs to reduce their risk
		factors after their CHD event. In some cases the measurements were
		made after these treatments. These data are taken from a larger
		dataset, described in  Rousseauw et al, 1983, South African Medical
		Journal. 
		
	\end{framed}
	%Load the South Africa Heart Disease Data and create training and test sets with
	%the following code:
	%\begin{framed}
	%	\begin{verbatim}
	%	install.packages("ElemStatLearn")
	%	library(ElemStatLearn)
	%	data(SAheart)
	%	
	%	set.seed(8484)
	%	train = sample(1:dim(SAheart)[1],
	%	size=dim(SAheart)[1]/2,replace=F)
	%	trainSA = SAheart[train,]
	%	testSA = SAheart[-train,]
	%	
	%	\end{verbatim}
	%\end{framed}
	
	\noindent \textbf{Exercise}\\Fit a logistic regression model with
	\begin{itemize}
		\item \textit{Coronary Heart Disease} (\texttt{chd}) as the
		dependent variable
		
		\item \textit{age at onset, current alcohol consumption, obesity levels,
			cumulative tabacco, type-A behavior}, and \textit{low density lipoprotein cholesterol} as predictor variables. 
	\end{itemize} 
	{
		\large
		
		\begin{verbatim}
		> head(SAheart)
		sbp tobacco  ldl adiposity famhist typea obesity alcohol age chd
		1 160   12.00 5.73     23.11 Present    49   25.30   97.20  52   1
		2 144    0.01 4.41     28.61  Absent    55   28.87    2.06  63   1
		3 118    0.08 3.48     32.28 Present    52   29.14    3.81  46   0
		4 170    7.50 6.41     38.03 Present    51   31.99   24.26  58   1
		5 134   13.60 3.50     27.78 Present    60   25.99   57.34  49   1
		6 132    6.20 6.47     36.21 Present    62   30.77   14.14  45   0
		...
		...
		\end{verbatim}
		
	}
	\noindent Calculate the misclassification rate for your model using this model
	% function and a prediction on the "response" scale:
	
	%\noindent What is the misclassification rate on the training set? What is the
	%misclassification rate on the test set?
	%\begin{framed}
	%	\begin{verbatim}
	%	head(SAheart)
	%	
	%	lr1 <- glm(chd ~ age + alcohol + obesity + 
	%	tobacco + typea + ldl, data=trainSA, 
	%	family="binomial")
	%	
	%	lr1.train.predict <- predict(lr1, type="response")
	%	
	%	missclass.lr1.train <- missClass(trainSA$chd, 
	%	lr1.train.predict)
	%	
	%	lr1.test.predict <- predict(lr1, newdata=testSA, 
	%	type="response")
	%	
	%	missclass.lr1.test <- missClass(testSA$chd, 
	%	lr1.test.predict)
	%	\end{verbatim}
	%\end{framed}
	
	
	\newpage
	%-------------------------------------------------------%
	
	%------------------------------------------------------------------------------------%

	\section{Review of Logistic Regression}
	% http://www.nesug.org/proceedings/nesug06/an/da26.pdf
	% http://www.ccsr.ac.uk/publications/teaching/blr.pdf
	% http://www.southampton.ac.uk/ghp3/docs/unicef/presentation7.1a.pdf
	% ftp://public.dhe.ibm.com/software/analytics/spss/documentation/statistics/20.0/en/client/Manuals/IBM_SPSS_Regression.pdf
	% http://www.umass.edu/statdata/statdata/data/
		
	\subsection*{Logistic Regression: Logit Transformation}
	%http://data.princeton.edu/wws509/notes/c3.pdf
	
	The logit transformation is given by the following formula: 
	\[ \eta_i = \mbox{logit}(\pi_i)  = \mbox{log}\left( \frac{\pi_i}{1- \pi_i} \right) \]
	
	\noindent 
	The inverse of the logit transformation is given by the following formula: 
	\[ \pi_i = \mbox{logit}^{-1}(\eta_i)  =  \frac{e^{\eta_i}}{1 + e^{\eta_i}} \]
	
	%---------------------------%


	\subsection{Dummy variables}
	When an explanatory variable is categorical we can use \textbf{dummy variables} to contrast
	the different categories. For each variable we choose a baseline category and then
	contrast all remaining categories with the base line. If an explanatory variable
	has k categories, we need k-1 dummy variables to investigate all the differences in
	the categories with respect to the dependent variable.
	
	For example suppose the explanatory variable was \textbf{\textit{housing}} coded like this:
	\begin{itemize}
		\item[1:] Owner occupier
		\item[2:] renting from a private landlord
		\item[3:] renting from the local authority
	\end{itemize}
	
	We would therefore need to choose a baseline category and create two dummy
	variables. For example if we chose owner occupier as the baseline category we
	would code the dummy variables (House1 and House2) like this
	
	%Tenure: &House1 &House2\\
	%Owner occupier &0& 0\\
	%Rented private &1 &0\\
	%Rented local authority &0 &1\\
	
	\subsection{Log Likelihood}
	A ``likelihood" is a probability, specifically the probability that the observed values of the dependent may be predicted from the observed values of the independents. 
	
	Like any probability, the likelihood varies from 0 to 1. The log likelihood (LL) is its log and varies from 0 to minus infinity (it is negative because the log of any number less than 1 is negative). LL is calculated through iteration, using maximum likelihood estimation (MLE).
	
	
	\subsection{Maximum Likelihood Estimation}
	\begin{itemize}
		\item Maximum likelihood estimation, MLE, is the method used to calculate the logit coefficients. This contrasts to the use of ordinary least squares (OLS) estimation of coefficients in regression. OLS seeks to minimize the sum of squared distances of the data points to the regression line. 
		\item MLE seeks to maximize the log likelihood, LL, which reflects how likely it is (the odds) that the observed values of the dependent may be predicted from the observed values of the independents. (Equivalently MLE seeks to minimize the -2LL value.)
		
		\item MLE is an iterative algorithm which starts with an initial arbitrary ``guesstimate" of what the logit coefficients should be, the MLE algorithm determines the direction and size change in the logit coefficients which will increase LL. 
		\item After this initial function is estimated, the residuals are tested and a re-estimate is made with an improved function, and the process is repeated (usually about a half-dozen times) until convergence is reached (that is, until LL does not change significantly). There are several alternative convergence criteria.
	\end{itemize}
	
	

	
	\subsection*{Wald Test}
	\begin{itemize}
		\item 	The Wald test is a way of testing the signi�cance of particular explanatory variables in a statistical model. In logistic regression we have a binary outcome variable and one or more explanatory variables. For each explanatory variable in the model there will be an associated parameter.
	\item 	The Wald test, described by Polit (1996) and Agresti (1990), is one of a number of ways of testing whether the parameters associated with a group of
		explanatory variables are zero.
		
	\item 	If for a particular explanatory variable, or group of explanatory variables, the Wald test is significant, then we would conclude that the parameters associated with these variables are not zero, so that the variables should be included in the model. If the Wald test is not significant then these explanatory variables can be omitted from the model. When considering a single explanatory variable, Altman (1991) uses a t-test to check whether the parameter is significant. 
		
\item 		For a single parameter the Wald statistic is just the square of the t-statistic and so will give exactly equivalent results.
		An alternative and widely used approach to testing the significance of a number of explanatory variables is to use the likelihood ratio test. This is
		appropriate for a variety of types of statistical models. Agresti (1990) argues that the likelihood ratio test is better, particularly if the sample size is small or the parameters are large.
	\end{itemize}

	
	The Wald Test is a statistical test used to determine whether an effect exists or not,
	
	It tests whether an independent variable has a statistically significant relationship with a dependent variable.
	
	It is used in a great variety of different models including models for dichotomous variables and model for continuous
	variables.
	
	\begin{itemize}
		\item $\hat{\theta}$ Maximum likelihood estimate of the parameter of interest $\theta$
		\item $\theta_o$ Proposed value.
		This is an assumption of the fact that the differences between $\hat{\theta}$ and $\theta_o$ is normal.
		
		Univariate case
		\[ \frac{(\hat{\theta} - \theta_o )^2}{\mbox{var}(\hat{\theta})} \sim \chi^2 \]
		\[ \frac{(\hat{\theta} - \theta_o )^2}{\mbox{s.e.}(\hat{\theta})} \sim \mbox{Normal} \]
	\end{itemize}	
	The likelihood ratio test is also used to determine whether an effect exists.
	]
\newpage
\section{Logistic Regression}
Logistic regression determines the impact of multiple independent variables
presented simultaneously to predict membership of one or other of the two
dependent variable categories.

\subsection{The purpose of logistic regression}
\begin{itemize}
	\item The crucial limitation of linear regression is that it cannot deal with Dependent Variables�s that are \textbf{\textit{dichotomous}} and categorical. Many interesting variables in the business world are dichotomous: for
	example, consumers make a decision to buy or not buy (\textit{\textbf{Buy/Don't Buy}}), a product may pass or fail quality control (\textit{\textbf{Pass/Fail}}), there are good or poor credit risks (\textit{\textbf{Good/Poor}}), an employee may be promoted or not (\textit{\textbf{Promote/Don't Promote}}).
	
	\item 	A range of regression techniques have been developed for analysing data with categorical dependent
	variables, including logistic regression and discriminant analysis (Hence referred to as DA, which is the next section of course).
	
	\item 	Logistical regression is regularly used rather than discriminant analysis when there are only two categories
	for the dependent variable. Logistic regression is also easier to use with SPSS than DA when
	there is a mixture of numerical and categorical Independent Variables�s, because it includes procedures for
	generating the necessary dummy variables automatically, requires fewer assumptions, and
	is more statistically robust. DA strictly requires the continuous independent variables  (though dummy variables can be used as in multiple regression). Thus, in instances where
	the independent variables are categorical, or a mix of continuous and categorical, and the
	DV is categorical, logistic regression is necessary.
\end{itemize}


\subsection{Use of Binomial Probability Theory}
\begin{itemize}
	\item Since the dependent variable is dichotomous we cannot predict a numerical value for it
	using logistic regression, so the usual regression least squares deviations criteria for best fit
	approach of minimizing error around the line of best fit is inappropriate.
	
	\item 	Instead, logistic regression employs binomial probability theory in which there are only two values to
	predict: that probability (p) is 1 rather than 0, i.e. the event/person belongs to one group
	rather than the other.
	\item Logistic regression forms a best fitting equation or function using the
	maximum likelihood method (not part of course), which maximizes the probability of classifying the observed
	data into the appropriate category given the regression coefficients.
\end{itemize}


\section{Introduction to Logistic Regression}
\begin{itemize}
	\item Logistic regression or logit regression is a type of probabilistic statistical classification model.
	
	\item It is also used to predict a binary response from a binary predictor, used for predicting the outcome of a categorical dependent variable (i.e., a class label) based on one or more predictor variables (features). 
	
	\item That is, it is used in estimating empirical values of the parameters in a qualitative response model. The probabilities describing the possible outcomes of a single trial are modeled, as a function of the explanatory (predictor) variables, using a logistic function. 
	
	\item Logistic regression, also called a logit model, is used to model \textbf{dichotomous (i.e. Binary) outcome variables}. In the logit model the log odds of the outcome is modeled as a linear combination of the predictor variables.
	
	\item 
	Binary Logistic regression is used to determine the impact of multiple independent variables
	presented simultaneously to predict membership of one or other of the two
	dependent variable categories.
	\item Logistic regression determines the impact of multiple independent variables
	presented simultaneously to predict membership of one or other of the two
	dependent variable categories.
	\item However, if your dependent variable was not measured on a dichotomous scale, but a continuous scale instead, you will need to carry out \textbf{multiple regression}, whereas if your dependent variable was measured on an ordinal scale, \textbf{ordinal regression} would be a more appropriate starting point.
\end{itemize}



\section*{Introduction to Logistic Regression}

The term �\textbf{\textit{generalized linear model}}� is used to describe a procedure for
transforming the dependent variable so that the �right hand side� of the model
equation can be interpreted as a \textbf{�\textit{linear combination}}� of the explanatory variables. 	In logistic regression, the logit may be computed in a manner similar to linear regression:
\[ \eta_i = \beta_0 + \beta_1x_1 + \beta_2x_2 + \ldots  \]

In situations where the dependent (y) variable is continuous and can be
reasonably assumed to have a normal distribution we do not transform the y
variable at all and we can simply run a multiple linear regression analysis.

Otherwise some sort of transformation is applied.


%---------------------------------------------------------------%
\subsection{Binomial Logistic Regression} 
A binomial logistic regression (often referred to simply as logistic regression), predicts the probability that an observation falls into one of two categories of a \textbf{dichotomous} dependent variable based on one or more independent variables that can be either continuous or categorical.

% \section{Binomial Logistic Regression}
Binomial logistic regression estimates the probability of an event (as an example, having heart disease) occurring. 
\begin{itemize}
	\item If the estimated probability of the event occurring is greater than or equal to 0.5 (better than even chance), the procedure classifies the event as occurring (e.g., heart disease being present). \item If the probability is less than 0.5, Logistic regression classifies the event as not occurring (e.g., no heart disease). 
\end{itemize}

\subsection{Examples of Logistic Regression}

\begin{description}
	\item[Example 1:]  Suppose that we are interested in the factors that influence whether a political candidate wins an election.  The outcome (response) variable is binary (0/1); \textit{ win or lose}.  The predictor variables of interest are the amount of money spent on the campaign, the amount of time spent campaigning negatively and whether or not the candidate is an incumbent.
	
	\item[Example 2:]  A researcher is interested in how variables, such as GRE (Graduate Record Exam scores), GPA (grade point average) and prestige of the undergraduate institution, effect admission into graduate school. The response variable, \textit{admit/don't admit}, is a binary variable.
\end{description}


\subsection{Assumptions of logistic regression}


\begin{description}
	\item[Assumption 1:] Your dependent variable should be measured on a \textbf{dichotomous scale}. Examples of dichotomous variables include gender (two groups: "males" and "females"), presence of heart disease (two groups: "yes" and "no"), personality type (two groups: "introversion" or "extroversion"), body composition (two groups: "obese" or "not obese"), and so forth. \\
	\newline
	
	
	\item[Assumption 2:] You have one or more independent variables, which can be either continuous (i.e., an interval or ratio variable) or categorical (i.e., an ordinal or nominal variable). 
	
	
	
	\item[Assumption 3:] You should have independence of observations and the dependent variable should have mutually exclusive and exhaustive categories.
	
	\item[Assumption 4:] There needs to be a linear relationship between any continuous independent variables and the logit transformation of the dependent variable. 
\end{description}
\begin{itemize}
	\item Logistic regression does not assume a linear relationship between the dependent and
	independent variables.
	\item The dependent variable must be a dichotomy (2 categories).
	\textit{(Remark: Dichotomous refers to two outcomes. Multichotomous refers to more than two outcomes)}.
	\item The independent variables need not be interval, nor normally distributed, nor linearly
	related, nor of equal variance within each group.
	\item The categories (groups) must be mutually exclusive and exhaustive; a case can only be
	in one group and every case must be a member of one of the groups.
	\item Larger samples are needed than for linear regression because maximum likelihood
	coefficients are large sample estimates. A minimum of 50 cases per predictor is
	recommended.
\end{itemize}	
%----------------------------------------------------------------------------------------%

\begin{framed}
	\subsection*{Types of Variables (Revision)}
	\begin{itemize}
		\item Examples of \textbf{continuous variables} include revision time (measured in hours), intelligence (measured using IQ score), exam performance (measured from 0 to 100), weight (measured in kg), and so forth. 
		
		\item Examples of \textbf{ordinal variables} include \textit{Likert} items (e.g., a 7-point scale from "strongly agree" through to "strongly disagree"), amongst other ways of ranking categories (e.g., a 3-point scale explaining how much a customer liked a product, ranging from "Not very much" to "Yes, a lot"). 
		\item Examples of \textbf{nominal variables} include gender (e.g., 2 groups: male and female), ethnicity (e.g., 3 groups: Caucasian, African American and Hispanic), profession (e.g., 5 groups: surgeon, doctor, nurse, dentist, therapist), and so forth.
	\end{itemize}
\end{framed}






\subsection*{Logistic function} 

The logistic function, with $\beta_0 + \beta_1 x$ on the horizontal axis and $\pi(x)$ on the vertical axis
An explanation of logistic regression begins with an explanation of the logistic function, which always takes on values between zero and one:
\[
F(t) = \frac{e^t}{e^t+1} = \frac{1}{1+e^{-t}},
\]
and viewing $t$ as a linear function of an explanatory variable x (or of a linear combination of explanatory variables), the logistic function can be written as:
\[\pi(x) = \frac{e^{(\beta_0 + \beta_1 x)}} {e^{(\beta_0 + \beta_1 x)} + 1} = \frac {1} {1+e^{-(\beta_0 + \beta_1 x)}}.
\]
This will be interpreted as the probability of the dependent variable equalling a ``success" or ``case" rather than a failure or non-case. We also define the inverse of the logistic function, the logit:
\[g(x) = \log \frac {\pi(x)} {1 - \pi(x)} = \beta_0 + \beta_1 x ,
\]and equivalently:
\[\frac{\pi(x)} {1 - \pi(x)} = e^{(\beta_0 + \beta_1 x)}.
\]
%======================================================%


%---------------------------%


%------------------------------------------------------------------------------------%
\section{Logistic Regression}
Logistic regression, also called a logit model, is used to model \textbf{dichotomous outcome} variables. 
In the logit model the \textbf{log odds} of the outcome is modeled as a linear combination of the predictor variables.

In logistic regression theory, the predicted dependent variable is a function of the probability that a particular subject will be in one of the categories (for example, the probability that a patient has the disease, given his or her set of scores on the predictor variables).

\subsection{Dummy variables}
When an explanatory variable is categorical we can use \textbf{dummy variables} to contrast
the different categories. For each variable we choose a baseline category and then
contrast all remaining categories with the base line. If an explanatory variable
has k categories, we need k-1 dummy variables to investigate all the differences in
the categories with respect to the dependent variable.

For example suppose the explanatory variable was \textbf{\textit{housing}} coded like this:
\begin{itemize}
	\item[1:] Owner occupier
	\item[2:] renting from a private landlord
	\item[3:] renting from the local authority
\end{itemize}

We would therefore need to choose a baseline category and create two dummy
variables. For example if we chose owner occupier as the baseline category we
would code the dummy variables (House1 and House2) like this

%Tenure: &House1 &House2\\
%Owner occupier &0& 0\\
%Rented private &1 &0\\
%Rented local authority &0 &1\\

\subsection{Log Likelihood}
A ``likelihood" is a probability, specifically the probability that the observed values of the dependent may be predicted from the observed values of the independents. 

Like any probability, the likelihood varies from 0 to 1. The log likelihood (LL) is its log and varies from 0 to minus infinity (it is negative because the log of any number less than 1 is negative). LL is calculated through iteration, using maximum likelihood estimation (MLE).


\subsection{Maximum Likelihood Estimation}
\begin{itemize}
	\item Maximum likelihood estimation, MLE, is the method used to calculate the logit coefficients. This contrasts to the use of ordinary least squares (OLS) estimation of coefficients in regression. OLS seeks to minimize the sum of squared distances of the data points to the regression line. 
	\item MLE seeks to maximize the log likelihood, LL, which reflects how likely it is (the odds) that the observed values of the dependent may be predicted from the observed values of the independents. (Equivalently MLE seeks to minimize the -2LL value.)
	
	\item MLE is an iterative algorithm which starts with an initial arbitrary ``guesstimate" of what the logit coefficients should be, the MLE algorithm determines the direction and size change in the logit coefficients which will increase LL. 
	\item After this initial function is estimated, the residuals are tested and a re-estimate is made with an improved function, and the process is repeated (usually about a half-dozen times) until convergence is reached (that is, until LL does not change significantly). There are several alternative convergence criteria.
\end{itemize}




\subsection*{Wald Test}
\begin{itemize}
	\item 	The Wald test is a way of testing the signi�cance of particular explanatory variables in a statistical model. In logistic regression we have a binary outcome variable and one or more explanatory variables. For each explanatory variable in the model there will be an associated parameter.
	\item 	The Wald test, described by Polit (1996) and Agresti (1990), is one of a number of ways of testing whether the parameters associated with a group of
	explanatory variables are zero.
	
	\item 	If for a particular explanatory variable, or group of explanatory variables, the Wald test is significant, then we would conclude that the parameters associated with these variables are not zero, so that the variables should be included in the model. If the Wald test is not significant then these explanatory variables can be omitted from the model. When considering a single explanatory variable, Altman (1991) uses a t-test to check whether the parameter is significant. 
	
	\item 		For a single parameter the Wald statistic is just the square of the t-statistic and so will give exactly equivalent results.
	An alternative and widely used approach to testing the significance of a number of explanatory variables is to use the likelihood ratio test. This is
	appropriate for a variety of types of statistical models. Agresti (1990) argues that the likelihood ratio test is better, particularly if the sample size is small or the parameters are large.
\end{itemize}


The Wald Test is a statistical test used to determine whether an effect exists or not,

It tests whether an independent variable has a statistically significant relationship with a dependent variable.

It is used in a great variety of different models including models for dichotomous variables and model for continuous
variables.

\begin{itemize}
	\item $\hat{\theta}$ Maximum likelihood estimate of the parameter of interest $\theta$
	\item $\theta_o$ Proposed value.
	This is an assumption of the fact that the differences between $\hat{\theta}$ and $\theta_o$ is normal.
	
	Univariate case
	\[ \frac{(\hat{\theta} - \theta_o )^2}{\mbox{var}(\hat{\theta})} \sim \chi^2 \]
	\[ \frac{(\hat{\theta} - \theta_o )^2}{\mbox{s.e.}(\hat{\theta})} \sim \mbox{Normal} \]
\end{itemize}	
The likelihood ratio test is also used to determine whether an effect exists.
]

%============================================================================================%


	\section{Wald statistic}
\begin{itemize}
	\item Alternatively, when assessing the contribution of individual predictors in a given model, one may examine the significance of the Wald statistic. The Wald statistic, analogous to the t-test in linear regression, is used to assess the significance of coefficients. 
	\item Alternatively, when assessing the contribution of individual predictors in a given model, one may examine the significance of the Wald statistic. The Wald statistic, analogous to the t-test in linear regression, is used to assess the significance of coefficients. 
	
	\item The Wald statistic is commonly used to test the significance of individual logistic regression coefficients for each independent variable (that is, to test the null hypothesis in logistic regression that a particular logit (effect) coefficient is zero). 
	
	\item The Wald Statistic is the ratio of the unstandardized logit coefficient to its standard error. The Wald statistic and its corresponding p probability level is part of SPSS output in the section \textbf{\textit{Variables in the Equation.}} This corresponds to significance testing of b coefficients in OLS regression. The researcher may well want to drop independents from the model when their effect is not significant by the Wald statistic.
	
	\item The Wald statistic is the ratio of the square of the regression coefficient to the square of the standard error of the coefficient and is asymptotically distributed as a chi-square distribution.
	
	\[W_j = \frac{B^2_j} {SE^2_{B_j}}\]
	
	\item Although several statistical packages (e.g., SPSS, SAS) report the Wald statistic to assess the contribution of individual predictors, the Wald statistic has limitations. 
	\item When the regression coefficient is large, the standard error of the regression coefficient also tends to be large increasing the probability of Type-II error. 
	
	\item The Wald statistic is the ratio of the square of the regression coefficient to the square of the standard error of the coefficient and is asymptotically distributed as a chi-square distribution.
	
	\item The Wald statistic also tends to be biased when data are sparse.
\end{itemize}
% Wald Test?
%--------------------------------------------------------%
\begin{figure}[h!]
	\centering
	\includegraphics[width=0.7\linewidth]{images/waldtest}
	\caption{}
	\label{fig:waldtest}
\end{figure}

\subsection{The Wald Test}
\begin{itemize}
	\item The Wald test is a way of testing the significance of particular predictor variables in a statistical model. 
	
	\item In logistic regression we have a binary outcome variable and one or more explanatory variables. For each predictor variable in the model there will be an associated parameter. The Wald test is one of a number of ways of testing whether the parameters associated with a group of explanatory variables are zero.
	%, described by Polit (1996) and Agresti (1990), 
	
	\item If for a particular explanatory variable, or group of explanatory variables, the Wald test is significant, then we would conclude that the parameters associated with these variables are not zero, so that the variables should be included in the model. If the Wald test is not significant then these explanatory variables can be omitted from the model. 
	
	\item When considering a single explanatory variable, Altman (1991) uses a t-test to check whether the parameter is significant. For a single parameter the Wald statistic is just the square of the t-statistic and so will give exactly equivalent results.
	
	\item An alternative and widely used approach to testing the significance of a number of explanatory variables is to use the likelihood ratio test. This is appropriate for a variety of types of statistical models. 
	
	\item Agresti (1990) argues that the likelihood ratio test is better, particularly if the sample size is small or the parameters are large.
\end{itemize}



%-------------------------------------------%
\newpage
\subsection{Variable Selection}
Like ordinary regression, logistic regression provides a coefficient \textbf{b} estimates, which measures
each IV�s partial contribution to variations in the response variables. The goal is to correctly predict
the category of outcome for individual cases using the most parsimonious model.

To accomplish this goal, a model (i.e. an equation) is created that includes all predictor variables that are useful in predicting the response variable. Variables can, if necessary, be entered into the model in the order specified by the researcher in a stepwise fashion like regression.

There are two main uses of logistic regression:
\begin{itemize}
	\item The first is the prediction of group membership. Since logistic regression calculates the
	probability of success over the probability of failure, the results of the analysis are in
	the form of an \textbf{odds ratio}.
	\item Logistic regression also provides knowledge of the relationships and strengths among
	the variables (e.g. playing golf with the boss puts you at a higher probability for job
	promotion than undertaking five hours unpaid overtime each week).
\end{itemize}



\subsection{Exercise Data Set}
The exercise data set comes from a survey of home owners
conducted by an electricity company about an offer of roof solar panels with a 50\% subsidy
from the state government as part of the state�s environmental policy. The variables involve
household income measured in units of a thousand dollars, age, monthly mortgage, size of
family household, and as the dependent variable, whether the householder would take or decline the offer.
The purpose of the exercise is to conduct a logistic regression to determine whether family
size and monthly mortgage will predict taking or declining the offer.

For the first demonstration, we will use `family size� and
`mortgage� only. For the options, select Classification Plots, Hosmer-Lemeshow Goodness
Of Fit, Casewise Listing Of Residuals and select Outliers Outside 2sd. Retain default
entries for probability of stepwise, classification cutoff and maximum iterations.

\begin{figure}[h!]
	\begin{center}
		% Requires \usepackage{graphicx}
		\includegraphics[scale=0.8]{images/Logistic10}\\
		\caption{Selected Options for Exercises}
	\end{center}
\end{figure}

We are not using any categorical variables this time. If there are categorical variables, use the \textbf{\textit{categorical}} option. For most situations, choose the �indicator� coding scheme (it is the
default).
\subsection{SPSS Outout  - Block 0: Beginning Block.}
Block 0 presents the results with only the constant included
before any coefficients (i.e. those relating to family size and mortgage) are entered into
the equation. Logistic regression compares this model with a model including all the
predictors (family size and mortgage) to determine whether the latter model is more
appropriate. The table suggests that if we knew nothing about our variables and guessed
that a person would not take the offer we would be correct 53.3\% of the time.
\begin{figure}[h!]
	\begin{center}
		% Requires \usepackage{graphicx}
		\includegraphics[scale=0.6]{images/Logistic3}\\
		\caption{Classification table}
	\end{center}
\end{figure}
The variables not in the equation table tells us whether each IV improves the model. The answer is yes for both variables, with family size slightly better than mortgage size, as both are significant and if included would add to the predictive power of the model. If they had not been significant and able to contribute to the prediction,
then termination of the analysis would obviously occur at this point

\begin{figure}
	\begin{center}
		% Requires \usepackage{graphicx}
		\includegraphics[scale=0.6]{images/Logistic4}\\
		\caption{Variables in / not in the equation}
	\end{center}
\end{figure}
This presents the results when the predictors �family size� and
�mortgage� are included. Later SPSS prints a classification table which shows how the
classification error rate has changed from the original 53.3%. By adding the variables
we can now predict with 90\% accuracy (see Classification Table later). The
model appears good, but we need to evaluate model fit and significance as well. SPSS will
offer you a variety of statistical tests for model fit and whether each of the independent
variables included make a significant contribution to the model.
\begin{figure}
	\begin{center}
		% Requires \usepackage{graphicx}
		\includegraphics[scale=0.6]{images/Logistic5}\\
		\caption{Test Outcomes}
	\end{center}
\end{figure}


%The likelihood function can be thought of as a measure of how well a candidate model fits the data (although that is a very simplistic definition). The AIC criterion is based on the Likelihood function.
%The likelihood function of a fitted model is commonly re-expressed as -2LL (i.e. The log of the likelihood times minus 2).

%The difference between �2LL for the best-fitting model and �2LL for the null hypothesis model (in which all the b values are set to zero in block 0) is distributed like
%chi squared, with degrees of freedom equal to the number of predictors; this difference
%is the Model chi square that SPSS refers to. Very conveniently, the difference between �2LL values for models with successive terms added also has a chi squared distribution,
%so when we use a stepwise procedure, we can use chi-squared tests to find out if adding
%one or more extra predictors significantly improves the fit of our model.



\subsection{Logistic Regression: Decision Rule}
Our decision rule will take the following form: If the probability of the event is greater than or equal to some threshold, we shall predict that the event will take place. By default, SPSS sets this threshold to .5. While that seems reasonable, in many cases we may want to set it higher or lower than .5.

%---------------------------------------------------------- %

\subsection{SPSS Output}
\begin{itemize}
	\item The variable Vote2005 is a binary variable describing turnout at a general election. The predictor variables are gender and age.
	\begin{center}
		\begin{figure}[h!]
			% Requires \usepackage{graphicx}
			\includegraphics[scale=0.6]{LogWeek10B}\\
			\caption{General Election 2005}
		\end{figure}
	\end{center}
	
	% Image LogWeek10-B
	
	\[\mbox{logit(vote2005)} = -.779 + .077\mbox{gender(1)}+.037\mbox{age}\]
	
	\item The age coefficient is statistically significant. Exp(B) for age is 1.038, which
	means for each year different in age, the person is 1.038 times more likely to turn
	out to vote, having allowed for gender in the model. Eg. a 21 year old is 1.038
	times as likely to turn out to vote than a 20 year old. \item This might not seem much
	of a difference but a 20 year difference leads to a person being $1.038^20 = 2.11$
	times more likely to turn out to vote. E.g. a 40 year old is 2.11 times more likely to
	turn out to vote than a 20 year old, having allowed for gender in the model.
	
	
	\item The gender coefficient is not statistically significant.
	
	
\end{itemize}

%-------------------------------------------%


\subsection{Hosmer-Lemeshow Prostate Example}
We will now consider a real life example to demonstrate Logistic Regression. This example is taken from a Prostate Cancer Study from Hosmer and Lemeshow (2000). The goal of the analysis is to determine if variables
measured at baseline can predict whether a tumour has penetrated the prostatic capsule. The variables are as follows:
\begin{center}
	\begin{figure}[h!]
		% Requires \usepackage{graphicx}
		\includegraphics[scale=0.6]{LogWeek10C}\\
		\caption{Variables}
	\end{figure}
\end{center}
\subsection{Kasser and Bruce Infarction Data Example}
We use a set of coronary data (Kasser and Bruce, 1969;
Kronmal and Tarter, 1974) to see if age, history of angina pectoris (ANGINA:
yes, no), history of high blood pressure (HIGHBP: yes, no), and functional class
(FUNCTION: none, minimal, moderate, and more than moderate) can be used to
predict the probability of past myocardial infarction (INFARCT: yes, no).



\subsection{The Likelihood Ratio Test}
The likelihood ratio test to test this hypothesis is based on the likelihood
function. We can formally test to see whether inclusion of an explanatory variable in a model tells us
more about the outcome variable than a model that does not include that variable. Suppose
we have to evaluate two models. 
\begin{center}
	\begin{figure}[h!]
		% Requires \usepackage{graphicx}
		\includegraphics[scale=0.75]{LogWeek10D}\\
		\caption{Variables}
	\end{figure}
\end{center}
Here, Model 1 is said to be nested within Model 2 � all the explanatory variables in Model 1
(X1) are included in Model 2. We are interested in whether the additional explanatory
variable in Model 2 ($X_2$) is required, i.e. does the simpler model (Model 1) fit the data just as
well as the fuller model (Model 2). In other words, we test the null hypothesis that $\beta_2 = 0$
against the alternative hypothesis that $\beta_2 \neq 0$. 



	\section{Wald statistic}
	\begin{itemize}
		\item Alternatively, when assessing the contribution of individual predictors in a given model, one may examine the significance of the Wald statistic. The Wald statistic, analogous to the t-test in linear regression, is used to assess the significance of coefficients. 
				\item Alternatively, when assessing the contribution of individual predictors in a given model, one may examine the significance of the Wald statistic. The Wald statistic, analogous to the t-test in linear regression, is used to assess the significance of coefficients. 
				
				\item The Wald statistic is commonly used to test the significance of individual logistic regression coefficients for each independent variable (that is, to test the null hypothesis in logistic regression that a particular logit (effect) coefficient is zero). 
				
				\item The Wald Statistic is the ratio of the unstandardized logit coefficient to its standard error. The Wald statistic and its corresponding p probability level is part of SPSS output in the section \textbf{\textit{Variables in the Equation.}} This corresponds to significance testing of b coefficients in OLS regression. The researcher may well want to drop independents from the model when their effect is not significant by the Wald statistic.
				
				\item The Wald statistic is the ratio of the square of the regression coefficient to the square of the standard error of the coefficient and is asymptotically distributed as a chi-square distribution.
		
		\[W_j = \frac{B^2_j} {SE^2_{B_j}}\]
		
		\item Although several statistical packages (e.g., SPSS, SAS) report the Wald statistic to assess the contribution of individual predictors, the Wald statistic has limitations. 
		\item When the regression coefficient is large, the standard error of the regression coefficient also tends to be large increasing the probability of Type-II error. 
				
		\item The Wald statistic is the ratio of the square of the regression coefficient to the square of the standard error of the coefficient and is asymptotically distributed as a chi-square distribution.
		
		\item The Wald statistic also tends to be biased when data are sparse.
	\end{itemize}
	% Wald Test?
		%--------------------------------------------------------%
	\begin{figure}[h!]
		\centering
		\includegraphics[width=0.7\linewidth]{images/waldtest}
		\caption{}
		\label{fig:waldtest}
	\end{figure}
	
	\subsection{The Wald Test}
	\begin{itemize}
		\item The Wald test is a way of testing the significance of particular predictor variables in a statistical model. 
		
		\item In logistic regression we have a binary outcome variable and one or more explanatory variables. For each predictor variable in the model there will be an associated parameter. The Wald test is one of a number of ways of testing whether the parameters associated with a group of explanatory variables are zero.
		%, described by Polit (1996) and Agresti (1990), 
		
		\item If for a particular explanatory variable, or group of explanatory variables, the Wald test is significant, then we would conclude that the parameters associated with these variables are not zero, so that the variables should be included in the model. If the Wald test is not significant then these explanatory variables can be omitted from the model. 
		
		\item When considering a single explanatory variable, Altman (1991) uses a t-test to check whether the parameter is significant. For a single parameter the Wald statistic is just the square of the t-statistic and so will give exactly equivalent results.
		
		\item An alternative and widely used approach to testing the significance of a number of explanatory variables is to use the likelihood ratio test. This is appropriate for a variety of types of statistical models. 
		
		\item Agresti (1990) argues that the likelihood ratio test is better, particularly if the sample size is small or the parameters are large.
	\end{itemize}
	
	
	
	%-------------------------------------------%
	\newpage
\subsection{Variable Selection}
Like ordinary regression, logistic regression provides a coefficient \textbf{b} estimates, which measures
each IV�s partial contribution to variations in the response variables. The goal is to correctly predict
the category of outcome for individual cases using the most parsimonious model.

To accomplish this goal, a model (i.e. an equation) is created that includes all predictor variables that are useful in predicting the response variable. Variables can, if necessary, be entered into the model in the order specified by the researcher in a stepwise fashion like regression.

There are two main uses of logistic regression:
\begin{itemize}
\item The first is the prediction of group membership. Since logistic regression calculates the
probability of success over the probability of failure, the results of the analysis are in
the form of an \textbf{odds ratio}.
\item Logistic regression also provides knowledge of the relationships and strengths among
the variables (e.g. playing golf with the boss puts you at a higher probability for job
promotion than undertaking five hours unpaid overtime each week).
\end{itemize}



\subsection{Exercise Data Set}
The exercise data set comes from a survey of home owners
conducted by an electricity company about an offer of roof solar panels with a 50\% subsidy
from the state government as part of the state�s environmental policy. The variables involve
household income measured in units of a thousand dollars, age, monthly mortgage, size of
family household, and as the dependent variable, whether the householder would take or decline the offer.
The purpose of the exercise is to conduct a logistic regression to determine whether family
size and monthly mortgage will predict taking or declining the offer.

For the first demonstration, we will use `family size� and
`mortgage� only. For the options, select Classification Plots, Hosmer-Lemeshow Goodness
Of Fit, Casewise Listing Of Residuals and select Outliers Outside 2sd. Retain default
entries for probability of stepwise, classification cutoff and maximum iterations.

\begin{figure}[h!]
\begin{center}
  % Requires \usepackage{graphicx}
  \includegraphics[scale=0.8]{images/Logistic10}\\
  \caption{Selected Options for Exercises}
\end{center}
\end{figure}

We are not using any categorical variables this time. If there are categorical variables, use the \textbf{\textit{categorical}} option. For most situations, choose the �indicator� coding scheme (it is the
default).
\subsection{SPSS Outout  - Block 0: Beginning Block.}
Block 0 presents the results with only the constant included
before any coefficients (i.e. those relating to family size and mortgage) are entered into
the equation. Logistic regression compares this model with a model including all the
predictors (family size and mortgage) to determine whether the latter model is more
appropriate. The table suggests that if we knew nothing about our variables and guessed
that a person would not take the offer we would be correct 53.3\% of the time.
\begin{figure}[h!]
\begin{center}
  % Requires \usepackage{graphicx}
  \includegraphics[scale=0.6]{images/Logistic3}\\
  \caption{Classification table}
\end{center}
\end{figure}
The variables not in the equation table tells us whether each IV improves the model. The answer is yes for both variables, with family size slightly better than mortgage size, as both are significant and if included would add to the predictive power of the model. If they had not been significant and able to contribute to the prediction,
then termination of the analysis would obviously occur at this point

\begin{figure}
\begin{center}
  % Requires \usepackage{graphicx}
  \includegraphics[scale=0.6]{images/Logistic4}\\
  \caption{Variables in / not in the equation}
\end{center}
\end{figure}
This presents the results when the predictors �family size� and
�mortgage� are included. Later SPSS prints a classification table which shows how the
classification error rate has changed from the original 53.3%. By adding the variables
we can now predict with 90\% accuracy (see Classification Table later). The
model appears good, but we need to evaluate model fit and significance as well. SPSS will
offer you a variety of statistical tests for model fit and whether each of the independent
variables included make a significant contribution to the model.
\begin{figure}
\begin{center}
  % Requires \usepackage{graphicx}
  \includegraphics[scale=0.6]{images/Logistic5}\\
  \caption{Test Outcomes}
\end{center}
\end{figure}


%The likelihood function can be thought of as a measure of how well a candidate model fits the data (although that is a very simplistic definition). The AIC criterion is based on the Likelihood function.
%The likelihood function of a fitted model is commonly re-expressed as -2LL (i.e. The log of the likelihood times minus 2).

%The difference between �2LL for the best-fitting model and �2LL for the null hypothesis model (in which all the b values are set to zero in block 0) is distributed like
%chi squared, with degrees of freedom equal to the number of predictors; this difference
%is the Model chi square that SPSS refers to. Very conveniently, the difference between �2LL values for models with successive terms added also has a chi squared distribution,
%so when we use a stepwise procedure, we can use chi-squared tests to find out if adding
%one or more extra predictors significantly improves the fit of our model.



\subsection{Logistic Regression: Decision Rule}
Our decision rule will take the following form: If the probability of the event is greater than or equal to some threshold, we shall predict that the event will take place. By default, SPSS sets this threshold to .5. While that seems reasonable, in many cases we may want to set it higher or lower than .5.

%---------------------------------------------------------- %

\subsection{SPSS Output}
\begin{itemize}
	\item The variable Vote2005 is a binary variable describing turnout at a general election. The predictor variables are gender and age.
	\begin{center}
		\begin{figure}[h!]
			% Requires \usepackage{graphicx}
			\includegraphics[scale=0.6]{LogWeek10B}\\
			\caption{General Election 2005}
		\end{figure}
	\end{center}
	
	% Image LogWeek10-B
	
	\[\mbox{logit(vote2005)} = -.779 + .077\mbox{gender(1)}+.037\mbox{age}\]
	
	\item The age coefficient is statistically significant. Exp(B) for age is 1.038, which
	means for each year different in age, the person is 1.038 times more likely to turn
	out to vote, having allowed for gender in the model. Eg. a 21 year old is 1.038
	times as likely to turn out to vote than a 20 year old. \item This might not seem much
	of a difference but a 20 year difference leads to a person being $1.038^20 = 2.11$
	times more likely to turn out to vote. E.g. a 40 year old is 2.11 times more likely to
	turn out to vote than a 20 year old, having allowed for gender in the model.
	
	
	\item The gender coefficient is not statistically significant.
	
	
\end{itemize}

%-------------------------------------------%


\subsection{Hosmer-Lemeshow Prostate Example}
We will now consider a real life example to demonstrate Logistic Regression. This example is taken from a Prostate Cancer Study from Hosmer and Lemeshow (2000). The goal of the analysis is to determine if variables
measured at baseline can predict whether a tumour has penetrated the prostatic capsule. The variables are as follows:
\begin{center}
	\begin{figure}[h!]
		% Requires \usepackage{graphicx}
		\includegraphics[scale=0.6]{LogWeek10C}\\
		\caption{Variables}
	\end{figure}
\end{center}
\subsection{Kasser and Bruce Infarction Data Example}
We use a set of coronary data (Kasser and Bruce, 1969;
Kronmal and Tarter, 1974) to see if age, history of angina pectoris (ANGINA:
yes, no), history of high blood pressure (HIGHBP: yes, no), and functional class
(FUNCTION: none, minimal, moderate, and more than moderate) can be used to
predict the probability of past myocardial infarction (INFARCT: yes, no).



\subsection{The Likelihood Ratio Test}
The likelihood ratio test to test this hypothesis is based on the likelihood
function. We can formally test to see whether inclusion of an explanatory variable in a model tells us
more about the outcome variable than a model that does not include that variable. Suppose
we have to evaluate two models. 
\begin{center}
	\begin{figure}[h!]
		% Requires \usepackage{graphicx}
		\includegraphics[scale=0.75]{LogWeek10D}\\
		\caption{Variables}
	\end{figure}
\end{center}
Here, Model 1 is said to be nested within Model 2 � all the explanatory variables in Model 1
(X1) are included in Model 2. We are interested in whether the additional explanatory
variable in Model 2 ($X_2$) is required, i.e. does the simpler model (Model 1) fit the data just as
well as the fuller model (Model 2). In other words, we test the null hypothesis that $\beta_2 = 0$
against the alternative hypothesis that $\beta_2 \neq 0$. 



\section{Summary of Logistic Regression}
logistic regression or logit regression is a type of regression analysis used for predicting
the outcome of a categorical dependent variable (a dependent variable that can take on a limited number of values,
whose magnitudes are not meaningful but whose ordering of magnitudes may or may not be meaningful)
based on one or more predictor variables.



\begin{itemize}
	\item[(1.)] Logistic regression is intended for the modeling
	of dichotomous categorical outcomes (e.g., characterized by binary responses: buy vs Don't buy, dead vs. alive, cancer vs. none,�).
	
	
	\item[(2.)] We want to predict the probability of a particular response  (0 to 1 scale).
	
	\item[(3.)] For binary responses, linear regression should not be used for several reasons
	but the most common-sense reason is that linear regression can provide predictions NOT on a 0 to 1 scale.
	but rather a predicted response of some numeric value (e.g 2.4 or -800.3).
	
	\item[(4.)] We need a way to link the probabilistic response variable to the continuous and/or categorical predictors and
	keep things on this 0 to 1 scale.
	
	\item[(5.)] Logistic regression winds up transforming the probabilities to odds and then taking the natural logarithm of these odds, called logits.
	
	
	\item[(6.)] Suppose a response variable is passing a test (by convention, 0=no and 1=yes).
	You have 1 predictor - number of days present in class over the past 30 days.
	Suppose the regression coefficient (often just called beta) in the output is 0.14.
	You would then say that, on average, as class presence increases by 1 day, the natural logarithm of the
	odds of passing the test increases by 0.14.
	
	\item[7.)] For the interpretation, you can just talk about the odds.
	Most computer output will give you this number.
	Suppose the answer in odds is 1.24. Then, you just say that,on average, as class presence increases by 1 day,
	the odds of passing the test are multiplied by 1.24.
	In other words, for each additional day present, the odds of passing are 24% greater than that of not passing.
	
	\item[8.)] To validate our findings, normally, we test whether the regression coefficient is equal to zero in the population.
	In logistic regression, the corresponding value for the odds is one (not zero). We got an odds of 1.24.
	Can we trust this? Or should we go with one (which would mean that the odds are the same for both passing and not passing,
	and hence class presence makes no difference at all)?  Look at the p-value (significance). If it less than .05 (by convention), you have enough evidence to reject
	the notion that the odds are really one. You go ahead and support the 1.24 result.
\end{itemize}


%-------------------------------------------%





%---------------------------------------------------------- %
\subsection{Variables in the Equation}
The Variables in the Equation table has several important elements. The Wald statistic and associated probabilities provide an index of the significance of each predictor in the equation.
The simplest way to assess Wald is to take the significance values and if less
than 0.05 reject the null hypothesis as the variable does make a significant contribution.
In this case, we note that family size contributed significantly to the prediction
(p = .013) but mortgage did not (p = .075). The researcher may well want to drop
independents from the model when their effect is not significant by the Wald statistic
(in this case mortgage).

\begin{figure}
	\begin{center}
		% Requires \usepackage{graphicx}
		\includegraphics[scale=0.6]{images/Logistic8}\\
		\caption{Variables in the Equation}
	\end{center}
\end{figure}

The \textbf{\textit{Exp(B)}} column in the table presents the extent to which raising the corresponding measure by one unit influences the odds ratio. We can interpret \textbf{\textit{Exp(B)}}) in
terms of the change in odds. If the value exceeds 1 then the odds of an outcome occurring increase; if the figure is less than 1, any increase in the predictor leads to a drop in
the odds of the outcome occurring. For example, the \textbf{\textit{Exp(B)}} value associated with
family size is 11.007. Hence when family size is raised by one unit (one person) the
odds ratio is 11 times as large and therefore householders are 11 more times likely to
belong to the take offer group.

The \textbf{\textit{B}} values are the logistic coefficients that can be used to create a predictive
equation (similar to the b values in linear regression) formula seen previously.
\begin{figure}
	\begin{center}
		% Requires \usepackage{graphicx}
		\includegraphics[scale=0.75]{images/Logistic11}\\
		\caption{Logistic Regression Equation}
	\end{center}
\end{figure}

Here is an example of the use of the predictive equation for a new case. Imagine a
householder whose household size including themselves was seven and paying
a monthly mortgage of $2,500$ euros. Would they take up the offer, i.e. belong to category 1?
Substituting in we get:
\begin{figure}
	\begin{center}
		% Requires \usepackage{graphicx}
		\includegraphics[scale=0.75]{images/Logistic12}\\
		\caption{Logistic Regression Equation : Example}
	\end{center}
\end{figure}

Therefore, the probability that a householder with seven in the household and a mortgage of 2,500 p.m. will take up the offer is 99\%, or 99\% of such individuals will be
expected to take up the offer.
Note that, given the non-significance of the mortgage variable, you could be justified
in leaving it out of the equation. As you can imagine, multiplying a mortgage value by
B adds a negligible amount to the prediction as its B value is so small (.005).

\newpage
\section{Summary of Logistic Regression}
logistic regression or logit regression is a type of regression analysis used for predicting
the outcome of a categorical dependent variable (a dependent variable that can take on a limited number of values,
whose magnitudes are not meaningful but whose ordering of magnitudes may or may not be meaningful)
based on one or more predictor variables.



\begin{itemize}
	\item[(1.)] Logistic regression is intended for the modeling
	of dichotomous categorical outcomes (e.g., characterized by binary responses: buy vs Don't buy, dead vs. alive, cancer vs. none,�).
	
	
	\item[(2.)] We want to predict the probability of a particular response  (0 to 1 scale).
	
	\item[(3.)] For binary responses, linear regression should not be used for several reasons
	but the most common-sense reason is that linear regression can provide predictions NOT on a 0 to 1 scale.
	but rather a predicted response of some numeric value (e.g 2.4 or -800.3).
	
	\item[(4.)] We need a way to link the probabilistic response variable to the continuous and/or categorical predictors and
	keep things on this 0 to 1 scale.
	
	\item[(5.)] Logistic regression winds up transforming the probabilities to odds and then taking the natural logarithm of these odds, called logits.
	
	
	\item[(6.)] Suppose a response variable is passing a test (by convention, 0=no and 1=yes).
	You have 1 predictor - number of days present in class over the past 30 days.
	Suppose the regression coefficient (often just called beta) in the output is 0.14.
	You would then say that, on average, as class presence increases by 1 day, the natural logarithm of the
	odds of passing the test increases by 0.14.
	
	\item[7.)] For the interpretation, you can just talk about the odds.
	Most computer output will give you this number.
	Suppose the answer in odds is 1.24. Then, you just say that,on average, as class presence increases by 1 day,
	the odds of passing the test are multiplied by 1.24.
	In other words, for each additional day present, the odds of passing are 24% greater than that of not passing.
	
	\item[8.)] To validate our findings, normally, we test whether the regression coefficient is equal to zero in the population.
	In logistic regression, the corresponding value for the odds is one (not zero). We got an odds of 1.24.
	Can we trust this? Or should we go with one (which would mean that the odds are the same for both passing and not passing,
	and hence class presence makes no difference at all)?  Look at the p-value (significance). If it less than .05 (by convention), you have enough evidence to reject
	the notion that the odds are really one. You go ahead and support the 1.24 result.
\end{itemize}


%-------------------------------------------%





%---------------------------------------------------------- %
\subsection{Variables in the Equation}
The Variables in the Equation table has several important elements. The Wald statistic and associated probabilities provide an index of the significance of each predictor in the equation.
The simplest way to assess Wald is to take the significance values and if less
than 0.05 reject the null hypothesis as the variable does make a significant contribution.
In this case, we note that family size contributed significantly to the prediction
(p = .013) but mortgage did not (p = .075). The researcher may well want to drop
independents from the model when their effect is not significant by the Wald statistic
(in this case mortgage).

\begin{figure}
\begin{center}
  % Requires \usepackage{graphicx}
  \includegraphics[scale=0.6]{images/Logistic8}\\
  \caption{Variables in the Equation}
\end{center}
\end{figure}

The \textbf{\textit{Exp(B)}} column in the table presents the extent to which raising the corresponding measure by one unit influences the odds ratio. We can interpret \textbf{\textit{Exp(B)}}) in
terms of the change in odds. If the value exceeds 1 then the odds of an outcome occurring increase; if the figure is less than 1, any increase in the predictor leads to a drop in
the odds of the outcome occurring. For example, the \textbf{\textit{Exp(B)}} value associated with
family size is 11.007. Hence when family size is raised by one unit (one person) the
odds ratio is 11 times as large and therefore householders are 11 more times likely to
belong to the take offer group.

The \textbf{\textit{B}} values are the logistic coefficients that can be used to create a predictive
equation (similar to the b values in linear regression) formula seen previously.
\begin{figure}
\begin{center}
  % Requires \usepackage{graphicx}
  \includegraphics[scale=0.75]{images/Logistic11}\\
  \caption{Logistic Regression Equation}
\end{center}
\end{figure}

Here is an example of the use of the predictive equation for a new case. Imagine a
householder whose household size including themselves was seven and paying
a monthly mortgage of $2,500$ euros. Would they take up the offer, i.e. belong to category 1?
Substituting in we get:
\begin{figure}
\begin{center}
  % Requires \usepackage{graphicx}
  \includegraphics[scale=0.75]{images/Logistic12}\\
  \caption{Logistic Regression Equation : Example}
\end{center}
\end{figure}

Therefore, the probability that a householder with seven in the household and a mortgage of 2,500 p.m. will take up the offer is 99\%, or 99\% of such individuals will be
expected to take up the offer.
Note that, given the non-significance of the mortgage variable, you could be justified
in leaving it out of the equation. As you can imagine, multiplying a mortgage value by
B adds a negligible amount to the prediction as its B value is so small (.005).



\end{document}

\subsection{HSB2 Example}
The hsb2 dataset is taken from a national survey of high school seniors. Two hundred observation were randomly sampled from the High School and Beyond survey. Descriptive statistics and exploratory data analysis are shown below.
Because we do not have a suitable dichotomous variable to use as our dependent variable, we will create one (which we will call honcomp, for honors composition) based on the continuous variable write.  We do not advocate making dichotomous variables out of continuous variables; rather, we do this here only for purposes of this illustration.


Here is the list of variables in the file.
\begin{verbatim}

obs:           200    highschool and beyond (200 cases)
vars:            12    28 Feb 2005 09:25
-----------------------------------------------------------------------------
variable
variable name   type   about the variable
-----------------------------------------------------------------------------
id              scale  student id
female        nominal  (0/1)
race          nominal  ethnicity (1=hispanic 2=asian 3=african-amer 4=white)
ses           ordinal  (1=low 2=middle 3=high)
schtyp        nominal  type of school (1=public 2=private)
prog          nominal  type of program (1=general 2=academic 3=vocational)
read            scale  standardized reading score
write           scale  standardized writing score
math            scale  standardized math score
science         scale  standardized science score
socst           scale  standardized social studies score
hon           nominal  honors english (0/1)
\end{verbatim}
\newpage
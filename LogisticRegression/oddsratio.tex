\documentclass[]{report}

\voffset=-1.5cm
\oddsidemargin=0.0cm
\textwidth = 480pt

\usepackage{framed}
\usepackage{subfiles}
\usepackage{enumerate}
\usepackage{graphics}
\usepackage{newlfont}
\usepackage{eurosym}
\usepackage{amsmath,amsthm,amsfonts}
\usepackage{amsmath}
\usepackage{color}
\usepackage{amssymb}
\usepackage{multicol}
\usepackage[dvipsnames]{xcolor}
\usepackage{graphicx}
\begin{document}

\section*{Odds Ratio - Example}

Suppose that in a sample of 100 men, 90 drank wine in the previous week, while in a sample of 100 women only 20 drank wine in the same period. The odds of a man drinking wine are 90 to 10, or 9:1, while the odds of a woman drinking wine are only 20 to 80, or 1:4 = 0.25:1. The odds ratio is thus 9/0.25, or 36, showing that men are much more likely to drink wine than women. The detailed calculation is:
\[{ 0.9/0.1 \over 0.2/0.8}=\frac{\;0.9\times 0.8\;}{\;0.1\times 0.2\;} ={0.72 \over 0.02} = 36.\]
This example also shows how odds ratios are sometimes sensitive in stating relative positions: in this sample men are 90/20 = 4.5 times more likely to have drunk wine than women, but have 36 times the odds. The logarithm of the odds ratio, the difference of the logits of the probabilities, tempers this effect, and also makes the measure symmetric with respect to the ordering of groups. For example, using natural logarithms, an odds ratio of 36/1 maps to 3.584, and an odds ratio of 1/36 maps to −3.584.

\section*{What is an odds ratio?}


An odds ratio (OR) is a measure of association between an exposure and an outcome. The OR represents the odds that an outcome will occur given a particular exposure, compared to the odds of the outcome occurring in the absence of that exposure. Odds ratios are most commonly used in case-control studies, however they can also be used in cross-sectional and cohort study designs as well (with some modifications and/or assumptions).

\subsection*{Odds ratios and logistic regression}
When a logistic regression is calculated, the regression coefficient (b1) is the estimated increase in the log odds of the outcome per unit increase in the value of the exposure. In other words, the exponential function of the regression coefficient (eb1) is the odds ratio associated with a one-unit increase in the exposure.

%---------------------------------------------------------------%
\subsection{When is it used?}
Odds ratios are used to compare the relative odds of the occurrence of the outcome of interest (e.g. disease or disorder), given exposure to the variable of interest (e.g. health characteristic, aspect of medical history). The odds ratio can also be used to determine whether a particular exposure is a risk factor for a particular outcome, and to compare the magnitude of various risk factors for that outcome.

\begin{itemize}
\item OR$=1$ Exposure does not affect odds of outcome
\item OR$>1$ Exposure associated with higher odds of outcome
\item OR$<1$ Exposure associated with lower odds of outcome
\end{itemize}
%---------------------------------------------------------------%

\subsection{What about confidence intervals?}
The 95\% confidence interval (CI) is used to estimate the precision of the OR. A large CI indicates a low level of precision of the OR, whereas a small CI indicates a higher precision of the OR. It is important to note however, that unlike the p value, the 95\% CI does not report a measure’s statistical significance. In practice, the 95\% CI is often used as a proxy for the presence of statistical significance if it does not overlap the null value (e.g. OR=1). Nevertheless, it would be inappropriate to interpret an OR with 95\% CI that spans the null value as indicating evidence for lack of association between the exposure and outcome.

\end{document}


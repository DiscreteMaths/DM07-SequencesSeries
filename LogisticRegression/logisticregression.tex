From Linear Regression
to Logistic Regression
In Chapter 2, Linear Regression, we discussed simple linear regression, multiple
linear regression, and polynomial regression. These models are special cases of the
generalized linear model, a flexible framework that requires fewer assumptions
than ordinary linear regression. In this chapter, we will discuss some of these
assumptions as they relate to another special case of the generalized linear model
called logistic regression.
Unlike the models we discussed previously, logistic regression is used for classification
tasks. Recall that the goal in classification tasks is to find a function that maps an
observation to its associated class or label. A learning algorithm must use pairs of
feature vectors and their corresponding labels to induce the values of the mapping
function's parameters that produce the best classifier, as measured by a particular
performance metric. In binary classification, the classifier must assign instances to one
of the two classes. Examples of binary classification include predicting whether or not
a patient has a particular disease, whether or not an audio sample contains human
speech, or whether or not the Duke men's basketball team will lose in the first round
of the NCAA tournament. In multiclass classification, the classifier must assign one
of several labels to each instance. In multilabel classification, the classifier must assign
a subset of the labels to each instance. In this chapter, we will work through several
classification problems using logistic regression, discuss performance measures for the
classification task, and apply some of the feature extraction techniques you learned in
the previous chapter.
%==================================%
\begin{frame}\frametitle{Logistic Regression}
% [ 72 ]
Binary classification with logistic
regression
Ordinary linear regression assumes that the response variable is normally distributed.
The normal distribution, also known as the Gaussian distribution or bell curve, is a
function that describes the probability that an observation will have a value between
any two real numbers. Normally distributed data is symmetrical. That is, half of the
values are greater than the mean and the other half of the values are less than the
mean. The mean, median, and mode of normally distributed data are also equal.
Many natural phenomena approximately follow normal distributions. For instance,
the height of people is normally distributed; most people are of average height, a few
are tall, and a few are short.
In some problems the response variable is not normally distributed. For instance,
a coin toss can result in two outcomes: heads or tails. The Bernoulli distribution
describes the probability distribution of a random variable that can take the positive
case with probability P or the negative case with probability 1-P. If the response
variable represents a probability, it must be constrained to the range {0,1}. Linear
regression assumes that a constant change in the value of an explanatory variable
results in a constant change in the value of the response variable, an assumption
that does not hold if the value of the response variable represents a probability.
Generalized linear models remove this assumption by relating a linear combination
of the explanatory variables to the response variable using a link function. In fact,
we already used a link function in Chapter 2, Linear Regression; ordinary linear
regression is a special case of the generalized linear model that relates a linear
combination of the explanatory variables to a normally distributed response variable
using the identity link function. We can use a different link function to relate a
linear combination of the explanatory variables to the response variable that is not
normally distributed.
In logistic regression, the response variable describes the probability that the outcome
is the positive case. If the response variable is equal to or exceeds a discrimination
threshold, the positive class is predicted; otherwise, the negative class is predicted. The
response variable is modeled as a function of a linear combination of the explanatory
variables using the logistic function. Given by the following equation, the logistic
function always returns a value between zero and one:
( ) 1
1 t F t
e- =
+
%==================================%
Chapter 4
% [ 73 ]
The following is a plot of the value of the logistic function for the range {-6,6}:
For logistic regression, t is equal to a linear combination of explanatory variables,
as follows:
( ) ( 0 )
1
1 x
F t
e -  + =
+
The logit function is the inverse of the logistic function. It links F (x) back to a linear
combination of the explanatory variables:
( ) ( )
1 ( ) 0 x
F x
g x ln
F x
= =  +
-
Now that we have defined the model for logistic regression, let's apply it to a binary
classification task.
Spam filtering
Our first problem is a modern version of the canonical binary classification problem:
spam classification. In our version, however, we will classify spam and ham SMS
messages rather than e-mail. We will extract TF-IDF features from the messages using
techniques you learned in Chapter 3, Feature Extraction and Preprocessing, and classify
the messages using logistic regression.
%==================================%
\begin{frame}\frametitle{Logistic Regression}
% [ 74 ]
We will use the SMS Spam Classification Data Set from the UCI Machine Learning
Repository. The dataset can be downloaded from http://archive.ics.uci.edu/
ml/datasets/SMS+Spam+Collection. First, let's explore the data set and calculate
some basic summary statistics using pandas:
>>> import pandas as pd
>>> df = pd.read_csv('data/SMSSpamCollection', delimiter='\t',
header=None)
>>> print df.head()
0 1
0 ham Go until jurong point, crazy.. Available only ...
1 ham Ok lar... Joking wif u oni...
2 spam Free entry in 2 a wkly comp to win FA Cup fina...
3 ham U dun say so early hor... U c already then say...
4 ham Nah I don't think he goes to usf, he lives aro...
[5 rows x 2 columns]
>>> print 'Number of spam messages:', df[df[0] == 'spam'][0].count()
>>> print 'Number of ham messages:', df[df[0] == 'ham'][0].count()
Number of spam messages: 747
Number of ham messages: 4825
A binary label and a text message comprise each row. The data set contains 5,574
instances; 4,827 messages are ham and the remaining 747 messages are spam. The
ham messages are labeled with zero, and the spam messages are labeled with one.
While the noteworthy, or case, outcome is often assigned the label one and the
non-case outcome is often assigned zero, these assignments are arbitrary. Inspecting
the data may reveal other attributes that should be captured in the model. The
following selection of messages characterizes both of the classes:
Spam: Free entry in 2 a wkly comp to win FA Cup final tkts 21st May
2005. Text FA to 87121 to receive entry question(std txt rate)T&C's
apply 08452810075over18's
Spam: WINNER!! As a valued network customer you have been selected
to receivea £900 prize reward! To claim call 09061701461. Claim code
KL341. Valid 12 hours only.
Ham: Sorry my roommates took forever, it ok if I come by now?
Ham: Finished class where are you.
%==================================%
Chapter 4
% [ 75 ]
Let's make some predictions using scikit-learn's LogisticRegression class:
>>> import numpy as np
>>> import pandas as pd
>>> from sklearn.feature_extraction.text import TfidfVectorizer
>>> from sklearn.linear_model.logistic import LogisticRegression
>>> from sklearn.cross_validation import train_test_split, cross_val_
score
First, we load the .csv file using pandas and split the data set into training and
test sets. By default, train_test_split() assigns 75 percent of the samples to the
training set and allocates the remaining 25 percent of the samples to the test set:
>>> df = pd.read_csv('data/SMSSpamCollection', delimiter='\t',
header=None)
>>> X_train_raw, X_test_raw, y_train, y_test = train_test_split(df[1],
df[0])
Next, we create a TfidfVectorizer. Recall from Chapter 3, Feature Extraction
and Preprocessing, that TfidfVectorizer combines CountVectorizer and
TfidfTransformer. We fit it with the training messages, and transform both the
training and test messages:
>>> vectorizer = TfidfVectorizer()
>>> X_train = vectorizer.fit_transform(X_train_raw)
>>> X_test = vectorizer.transform(X_test_raw)
Finally, we create an instance of LogisticRegression and train our model. Like
LinearRegression, LogisticRegression implements the fit() and predict()
methods. As a sanity check, we printed a few predictions for manual inspection:
>>> classifier = LogisticRegression()
>>> classifier.fit(X_train, y_train)
>>> predictions = classifier.predict(X_test)
>>> for i, prediction in enumerate(predictions[:5]):
>>> print 'Prediction: %s. Message: %s' % (prediction, X_test_
raw[i])
The following is the output of the script:
Prediction: ham. Message: If you don't respond imma assume you're
still asleep and imma start calling n shit
Prediction: spam. Message: HOT LIVE FANTASIES call now 08707500020
Just 20p per min NTT Ltd, PO Box 1327 Croydon CR9 5WB 0870 is a
national rate call
%==================================%
\begin{frame}\frametitle{Logistic Regression}
% [ 76 ]
Prediction: ham. Message: Yup... I havent been there before... You
want to go for the yoga? I can call up to book
Prediction: ham. Message: Hi, can i please get a &lt;#&gt; dollar
loan from you. I.ll pay you back by mid february. Pls.
Prediction: ham. Message: Where do you need to go to get it?
How well does our classifier perform? The performance metrics we used for linear
regression are inappropriate for this task. We are only interested in whether the
predicted class was correct, not how far it was from the decision boundary. In the
next section, we will discuss some performance metrics that can be used to evaluate
binary classifiers.
\end{frame}
%============================================================%
\begin{frame}
Binary classification performance metrics
A variety of metrics exist to evaluate the performance of binary classifiers against
trusted labels. The most common metrics are accuracy, precision, recall, F1 measure,
and ROC AUC score. All of these measures depend on the concepts of true positives,
true negatives, false positives, and false negatives. Positive and negative refer to the
classes. True and false denote whether the predicted class is the same as the true class.
For our SMS spam classifier, a true positive prediction is when the classifier correctly
predicts that a message is spam. A true negative prediction is when the classifier
correctly predicts that a message is ham. A prediction that a ham message is spam
is a false positive prediction, and a spam message incorrectly classified as ham is a
false negative prediction. 
A confusion matrix, or contingency table, can be used to
visualize true and false positives and negatives. The rows of the matrix are the true
classes of the instances, and the columns are the predicted classes of the instances:
>>> from sklearn.metrics import confusion_matrix
>>> import matplotlib.pyplot as plt
>>> y_test = [0, 0, 0, 0, 0, 1, 1, 1, 1, 1]
>>> y_pred = [0, 1, 0, 0, 0, 0, 0, 1, 1, 1]
>>> confusion_matrix = confusion_matrix(y_test, y_pred)
>>> print(confusion_matrix)
>>> plt.matshow(confusion_matrix)
>>> plt.title('Confusion matrix')
>>> plt.colorbar()
>>> plt.ylabel('True label')
>>> plt.xlabel('Predicted label')
>>> plt.show()
[[4 1]
[2 3]]
%==================================%
\begin{frame}
Chapter 4
% [ 77 ]
The confusion matrix indicates that there were four true negative predictions, three
true positive predictions, two false negative predictions, and one false positive
prediction. Confusion matrices become more useful in multi-class problems, in
which it can be difficult to determine the most frequent types of errors.
%==================================%
\begin{frame}
Accuracy
Accuracy measures a fraction of the classifier's predictions that are correct.
scikit-learn provides a function to calculate the accuracy of a set of predictions
given the correct labels:
>>> from sklearn.metrics import accuracy_score
>>> y_pred, y_true = [0, 1, 1, 0], [1, 1, 1, 1]
>>> print 'Accuracy:', accuracy_score(y_true, y_pred)
Accuracy: 0.5
%==================================%
\begin{frame}\frametitle{Logistic Regression}
% [ 78 ]
LogisticRegression.score() predicts and scores labels for a test set using
accuracy. Let's evaluate our classifier's accuracy:
>>> import numpy as np
>>> import pandas as pd
>>> from sklearn.feature_extraction.text import TfidfVectorizer
>>> from sklearn.linear_model.logistic import LogisticRegression
>>> from sklearn.cross_validation import train_test_split, cross_val_
score
>>> df = pd.read_csv('data/sms.csv')
>>> X_train_raw, X_test_raw, y_train, y_test = train_test_
split(df['message'], df['label'])
>>> vectorizer = TfidfVectorizer()
>>> X_train = vectorizer.fit_transform(X_train_raw)
>>> X_test = vectorizer.transform(X_test_raw)
>>> classifier = LogisticRegression()
>>> classifier.fit(X_train, y_train)
>>> scores = cross_val_score(classifier, X_train, y_train, cv=5)
>>> print np.mean(scores), scores
Accuracy 0.956217208018 % [ 0.96057348 0.95334928 0.96411483
0.95454545 0.94850299]
Note that your accuracy may differ as the training and test sets are assigned
randomly. While accuracy measures the overall correctness of the classifier, it
does not distinguish between false positive errors and false negative errors. Some
applications may be more sensitive to false negatives than false positives, or vice
versa. 
%==================================%
\begin{frame}
Furthermore, accuracy is not an informative metric if the proportions of
the classes are skewed in the population. For example, a classifier that predicts
whether or not credit card transactions are fraudulent may be more sensitive to
false negatives than to false positives. To promote customer satisfaction, the credit
card company may prefer to risk verifying legitimate transactions than risk ignoring
a fraudulent transaction. Because most transactions are legitimate, accuracy is
not an appropriate metric for this problem. A classifier that always predicts that
transactions are legitimate could have a high accuracy score, but would not be
useful. For these reasons, classifiers are often evaluated using two additional
measures called precision and recall.
%==================================%
Chapter 4
% [ 79 ]
Precision and recall
Recall from Chapter 1, The Fundamentals of Machine Learning, that precision is the
fraction of positive predictions that are correct. For instance, in our SMS spam
classifier, precision is the fraction of messages classified as spam that are actually
spam. Precision is given by the following ratio:
P TP
TP FP
=
+
Sometimes called sensitivity in medical domains, recall is the fraction of the truly
positive instances that the classifier recognizes. A recall score of one indicates
that the classifier did not make any false negative predictions. For our SMS spam
classifier, recall is the fraction of spam messages that were truly classified as spam.
Recall is calculated with the following ratio:
R TP
TP FN
=
+
Individually, precision and recall are seldom informative; they are both incomplete
views of a classifier's performance. Both precision and recall can fail to distinguish
classifiers that perform well from certain types of classifiers that perform poorly. A
trivial classifier could easily achieve a perfect recall score by predicting positive for
every instance. For example, assume that a test set contains ten positive examples
and ten negative examples. A classifier that predicts positive for every example will
achieve a recall of one, as follows:
10 1
10 0
R = =
+
A classifier that predicts negative for every example, or that makes only false positive
and true negative predictions, will achieve a recall score of zero. Similarly, a classifier
that predicts that only a single instance is positive and happens to be correct will
achieve perfect precision.
%==================================%
\begin{frame}\frametitle{Logistic Regression}
% [ 80 ]
scikit-learn provides a function to calculate the precision and recall for a classifier
from a set of predictions and the corresponding set of trusted labels. Let's calculate
our SMS classifier's precision and recall:
>>> import numpy as np
>>> import pandas as pd
>>> from sklearn.feature_extraction.text import TfidfVectorizer
>>> from sklearn.linear_model.logistic import LogisticRegression
>>> from sklearn.cross_validation import train_test_split, cross_val_
score
>>> df = pd.read_csv('data/sms.csv')
>>> X_train_raw, X_test_raw, y_train, y_test = train_test_
split(df['message'], df['label'])
>>> vectorizer = TfidfVectorizer()
>>> X_train = vectorizer.fit_transform(X_train_raw)
>>> X_test = vectorizer.transform(X_test_raw)
>>> classifier = LogisticRegression()
>>> classifier.fit(X_train, y_train)
>>> precisions = cross_val_score(classifier, X_train, y_train, cv=5,
scoring='precision')
>>> print 'Precision', np.mean(precisions), precisions
>>> recalls = cross_val_score(classifier, X_train, y_train, cv=5,
scoring='recall')
>>> print 'Recalls', np.mean(recalls), recalls
Precision 0.992137651822 % [ 0.98717949 0.98666667 1.
0.98684211 1. ]
Recall 0.677114261885 % [ 0.7 0.67272727 0.6 0.68807339
0.72477064]
Our classifier's precision is 0.992; almost all of the messages that it predicted as
spam were actually spam. Its recall is lower, indicating that it incorrectly classified
approximately 22 percent of the spam messages as ham. Your precision and recall
may vary since the training and test data are randomly partitioned.
Calculating the F1 measure
The F1 measure is the harmonic mean, or weighted average, of the precision and
recall scores. Also called the f-measure or the f-score, the F1 score is calculated using
the following formula:
F1 2 PR
P R
=
+
%==================================%
Chapter 4
% [ 81 ]
The F1 measure penalizes classifiers with imbalanced precision and recall scores,
like the trivial classifier that always predicts the positive class. A model with perfect
precision and recall scores will achieve an F1 score of one. A model with a perfect
precision score and a recall score of zero will achieve an F1 score of zero. As for
precision and recall, scikit-learn provides a function to calculate the F1 score for
a set of predictions. Let's compute our classifier's F1 score. The following snippet
continues the previous code sample:
>>> f1s = cross_val_score(classifier, X_train, y_train, cv=5,
scoring='f1')
>>> print 'F1', np.mean(f1s), f1s
F1 0.80261302628 % [ 0.82539683 0.8 0.77348066 0.83157895
0.7826087 ]
The arithmetic mean of our classifier's precision and recall scores is 0.803. As the
difference between the classifier's precision and recall is small, the F1 measure's
penalty is small. Models are sometimes evaluated using the F0.5 and F2 scores,
which favor precision over recall and recall over precision, respectively.
ROC AUC
A Receiver Operating Characteristic, or ROC curve, visualizes a classifier's
performance. Unlike accuracy, the ROC curve is insensitive to data sets with
unbalanced class proportions; unlike precision and recall, the ROC curve illustrates
the classifier's performance for all values of the discrimination threshold. ROC
curves plot the classifier's recall against its fall-out. Fall-out, or the false positive
rate, is the number of false positives divided by the total number of negatives. It is
calculated using the following formula:
F FP
TN FP
=
+
%==================================%
\begin{frame}\frametitle{Logistic Regression}
% [ 82 ]
AUC is the area under the ROC curve; it reduces the ROC curve to a single value,
which represents the expected performance of the classifier. The dashed line in the
following figure is for a classifier that predicts classes randomly; it has an AUC of
0.5. The solid curve is for a classifier that outperforms random guessing:
Let's plot the ROC curve for our SMS spam classifier:
>>> import numpy as np
>>> import pandas as pd
>>> import matplotlib.pyplot as plt
>>> from sklearn.feature_extraction.text import TfidfVectorizer
>>> from sklearn.linear_model.logistic import LogisticRegression
>>> from sklearn.cross_validation import train_test_split, cross_val_
score
>>> from sklearn.metrics import roc_curve, auc
>>> df = pd.read_csv('data/sms.csv')
>>> X_train_raw, X_test_raw, y_train, y_test = train_test_
split(df['message'], df['label'])
>>> vectorizer = TfidfVectorizer()
%==================================%
Chapter 4
% [ 83 ]
>>> X_train = vectorizer.fit_transform(X_train_raw)
>>> X_test = vectorizer.transform(X_test_raw)
>>> classifier = LogisticRegression()
>>> classifier.fit(X_train, y_train)
>>> predictions = classifier.predict_proba(X_test)
>>> false_positive_rate, recall, thresholds = roc_curve(y_test,
predictions[:, 1])
>>> roc_auc = auc(false_positive_rate, recall)
>>> plt.title('Receiver Operating Characteristic')
>>> plt.plot(false_positive_rate, recall, 'b', label='AUC = %0.2f' %
roc_auc)
>>> plt.legend(loc='lower right')
>>> plt.plot([0, 1], [0, 1], 'r--')
>>> plt.xlim([0.0, 1.0])
>>> plt.ylim([0.0, 1.0])
>>> plt.ylabel('Recall')
>>> plt.xlabel('Fall-out')
>>> plt.show()
From the ROC AUC plot, it is apparent that our classifier outperforms random
guessing; most of the plot area lies under its curve:
%==================================%
\begin{frame}\frametitle{Logistic Regression}
% [ 84 ]
Tuning models with grid search
Hyperparameters are parameters of the model that are not learned. For example,
hyperparameters of our logistic regression SMS classifier include the value of
the regularization term and thresholds used to remove words that appear too
frequently or infrequently. In scikit-learn, hyperparameters are set through the
model's constructor. In the previous examples, we did not set any arguments for
LogisticRegression(); we used the default values for all of the hyperparameters.
These default values are often a good start, but they may not produce the optimal
model. Grid search is a common method to select the hyperparameter values
that produce the best model. Grid search takes a set of possible values for each
hyperparameter that should be tuned, and evaluates a model trained on each
element of the Cartesian product of the sets. That is, grid search is an exhaustive
search that trains and evaluates a model for each possible combination of the
hyperparameter values supplied by the developer. A disadvantage of grid search
is that it is computationally costly for even small sets of hyperparameter values.
Fortunately, it is an embarrassingly parallel problem; many models can easily be
trained and evaluated concurrently since no synchronization is required between
the processes. Let's use scikit-learn's GridSearchCV() function to find better
hyperparameter values:
import pandas as pd
from sklearn.feature_extraction.text import TfidfVectorizer
from sklearn.linear_model.logistic import LogisticRegression
from sklearn.grid_search import GridSearchCV
from sklearn.pipeline import Pipeline
from sklearn.cross_validation import train_test_split
from sklearn.metrics import precision_score, recall_score, accuracy_
score
pipeline = Pipeline([
('vect', TfidfVectorizer(stop_words='english')),
('clf', LogisticRegression())
])
parameters = {
'vect__max_df': (0.25, 0.5, 0.75),
'vect__stop_words': ('english', None),
'vect__max_features': (2500, 5000, 10000, None),
'vect__ngram_range': ((1, 1), (1, 2)),
'vect__use_idf': (True, False),
'vect__norm': ('l1', 'l2'),
'clf__penalty': ('l1', 'l2'),
'clf__C': (0.01, 0.1, 1, 10),
}
%==================================%
Chapter 4
% [ 85 ]
GridSearchCV() takes an estimator, a parameter space, and performance measure.
The argument n_jobs specifies the maximum number of concurrent jobs; set n_jobs
to -1 to use all CPU cores. Note that fit() must be called in a Python main block in
order to fork additional processes; this example must be executed as a script, and not
in an interactive interpreter:
if __name__ == "__main__":
grid_search = GridSearchCV(pipeline, parameters, n_jobs=-1,
verbose=1, scoring='accuracy', cv=3)
df = pd.read_csv('data/sms.csv')
X, y, = df['message'], df['label']
X_train, X_test, y_train, y_test = train_test_split(X, y)
grid_search.fit(X_train, y_train)
print 'Best score: %0.3f' % grid_search.best_score_
print 'Best parameters set:'
best_parameters = grid_search.best_estimator_.get_params()
for param_name in sorted(parameters.keys()):
print '\t%s: %r' % (param_name, best_parameters[param_name])
predictions = grid_search.predict(X_test)
print 'Accuracy:', accuracy_score(y_test, predictions)
print 'Precision:', precision_score(y_test, predictions)
print 'Recall:', recall_score(y_test, predictions)
The following is the output of the script:
Fitting 3 folds for each of 1536 candidates, totalling 4608 fits
[Parallel(n_jobs=-1)]: Done 1 jobs | elapsed: 0.2s
[Parallel(n_jobs=-1)]: Done 50 jobs | elapsed: 4.0s
[Parallel(n_jobs=-1)]: Done 200 jobs | elapsed: 16.9s
[Parallel(n_jobs=-1)]: Done 450 jobs | elapsed: 36.7s
[Parallel(n_jobs=-1)]: Done 800 jobs | elapsed: 1.1min
[Parallel(n_jobs=-1)]: Done 1250 jobs | elapsed: 1.7min
[Parallel(n_jobs=-1)]: Done 1800 jobs | elapsed: 2.5min
[Parallel(n_jobs=-1)]: Done 2450 jobs | elapsed: 3.4min
[Parallel(n_jobs=-1)]: Done 3200 jobs | elapsed: 4.4min
[Parallel(n_jobs=-1)]: Done 4050 jobs | elapsed: 7.7min
[Parallel(n_jobs=-1)]: Done 4608 out of 4608 | elapsed: 8.5min
finished
Best score: 0.983
Best parameters set:
clf__C: 10
clf__penalty: 'l2'
vect__max_df: 0.5
vect__max_features: None
%==================================%
\begin{frame}\frametitle{Logistic Regression}
% [ 86 ]
vect__ngram_range: (1, 2)
vect__norm: 'l2'
vect__stop_words: None
vect__use_idf: True
Accuracy: 0.989956958393
Precision: 0.988095238095
Recall: 0.932584269663
Optimizing the values of the hyperparameters has improved our model's recall score
on the test set.
Multi-class classification
In the previous sections you learned to use logistic regression for binary classification.
In many classification problems, however, there are more than two classes that are
of interest. We might wish to predict the genres of songs from samples of audio,
or classify images of galaxies by their types. The goal of multi-class classification
is to assign an instance to one of the set of classes. scikit-learn uses a strategy
called one-vs.-all, or one-vs.-the-rest, to support multi-class classification. Onevs.-
all classification uses one binary classifier for each of the possible classes. The
class that is predicted with the greatest confidence is assigned to the instance.
LogisticRegression supports multi-class classification using the one-versus-all
strategy out of the box. Let's use LogisticRegression for a multi-class
classification problem.
Assume that you would like to watch a movie, but you have a strong aversion
to watching bad movies. To inform your decision, you could read reviews of the
movies you are considering, but unfortunately you also have a strong aversion to
reading movie reviews. Let's use scikit-learn to find the movies with good reviews.
In this example, we will classify the sentiments of phrases taken from movie reviews
in the Rotten Tomatoes data set. Each phrase can be classified as one of the following
sentiments: negative, somewhat negative, neutral, somewhat positive, or positive.
While the classes appear to be ordered, the explanatory variables that we will use
do not always corroborate this order due to sarcasm, negation, and other linguistic
phenomena. Instead, we will approach this problem as a multi-class classification task.
%==================================%
Chapter 4
% [ 87 ]
The data can be downloaded from http://www.kaggle.com/c/sentimentanalysis-
on-movie-reviews/data. First, let's explore the data set using pandas.
Note that the import and data-loading statements in the following snippet are
required for the subsequent snippets:
>>> import pandas as pd
>>> df = pd.read_csv('movie-reviews/train.tsv', header=0,
delimiter='\t')
>>> print df.count()
PhraseId 156060
SentenceId 156060
Phrase 156060
Sentiment 156060
dtype: int64
The columns of the data set are tab delimited. The data set contains 1,56,060 instances.
>>> print df.head()
PhraseId SentenceId
Phrase \
0 1 1 A series of escapades demonstrating the adage
...
1 2 1 A series of escapades demonstrating the adage
...
2 3 1 A
series
3 4 1
A
4 5 1
series
Sentiment
0 1
1 2
2 2
3 2
4 2
[5 rows x 4 columns]
%==================================%
\begin{frame}\frametitle{Logistic Regression}
% [ 88 ]
The Sentiment column contains the response variables. The 0 label corresponds
to the sentiment negative, 1 corresponds to somewhat negative, and so on. The
Phrase column contains the raw text. Each sentence from the movie reviews has
been parsed into smaller phrases. We will not require the PhraseId and SentenceId
columns in this example. Let's print some of the phrases and examine them:
>>> print df['Phrase'].head(10)
0 A series of escapades demonstrating the adage ...
1 A series of escapades demonstrating the adage ...
2 A series
3 A
4 series
5 of escapades demonstrating the adage that what...
6 of
7 escapades demonstrating the adage that what is...
8 escapades
9 demonstrating the adage that what is good for ...
Name: Phrase, dtype: object
Now let's examine the target classes:
>>> print df['Sentiment'].describe()
count 156060.000000
mean

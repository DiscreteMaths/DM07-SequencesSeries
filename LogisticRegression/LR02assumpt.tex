\documentclass[a4paper,12pt]{article}
%%%%%%%%%%%%%%%%%%%%%%%%%%%%%%%%%%%%%%%%%%%%%%%%%%%%%%%%%%%%%%%%%%%%%%%%%%%%%%%%%%%%%%%%%%%%%%%%%%%%%%%%%%%%%%%%%%%%%%%%%%%%%%%%%%%%%%%%%%%%%%%%%%%%%%%%%%%%%%%%%%%%%%%%%%%%%%%%%%%%%%%%%%%%%%%%%%%%%%%%%%%%%%%%%%%%%%%%%%%%%%%%%%%%%%%%%%%%%%%%%%%%%%%%%%%%
\usepackage{eurosym}
\usepackage{vmargin}
\usepackage{amsmath}
\usepackage{graphics}
\usepackage{epsfig}
\usepackage{subfigure}
\usepackage{fancyhdr}
%\usepackage{listings}
\usepackage{framed}
\usepackage{graphicx}

\setcounter{MaxMatrixCols}{10}
%TCIDATA{OutputFilter=LATEX.DLL}
%TCIDATA{Version=5.00.0.2570}
%TCIDATA{<META NAME="SaveForMode" CONTENT="1">}
%TCIDATA{LastRevised=Wednesday, February 23, 2011 13:24:34}
%TCIDATA{<META NAME="GraphicsSave" CONTENT="32">}
%TCIDATA{Language=American English}

\pagestyle{fancy}
\setmarginsrb{20mm}{0mm}{20mm}{25mm}{12mm}{11mm}{0mm}{11mm}
\lhead{MA4128} \rhead{10 April 2013}
\chead{Logistic Regression}
%\input{tcilatex}


\begin{document}

\subsection{Assumptions of logistic regression}
\begin{itemize}
\item Logistic regression does not assume a linear relationship between the dependent and
independent variables.
\item The dependent variable must be a dichotomy (2 categories).
(Remark: Dichotomous refers to two outcomes. Multichotomous refers to more than two outcomes).
\item The independent variables need not be interval, nor normally distributed, nor linearly
related, nor of equal variance within each group.
\item The categories (groups) must be mutually exclusive and exhaustive; a case can only be
in one group and every case must be a member of one of the groups.
\item Larger samples are needed than for linear regression because maximum likelihood
coefficients are large sample estimates. A minimum of 50 cases per predictor is
recommended.
\end{itemize}
\subsection{Maximum Likelihood Estimation}
Maximum likelihood estimation, MLE, is the method used to calculate the logit coefficients. This contrasts to the use of ordinary least squares (OLS) estimation of coefficients in regression. OLS seeks to minimize the sum of squared distances of the data points to the regression line. MLE seeks to maximize the log likelihood, LL, which reflects how likely it is (the odds) that the observed values of the dependent may be predicted from the observed values of the independents. (Equivalently MLE seeks to minimize the -2LL value.)

MLE is an iterative algorithm which starts with an initial arbitrary ``guesstimate" of what the logit coefficients should be, the MLE algorithm determines the direction and size change in the logit coefficients which will increase LL. After this initial function is estimated, the residuals are tested and a re-estimate is made with an improved function, and the process is repeated (usually about a half-dozen times) until convergence is reached (that is, until LL does not change significantly). There are several alternative convergence criteria.
%-------------------------------------------%


\subsection{Log Likelihood}
A ``likelihood" is a probability, specifically the probability that the observed values of the dependent may be predicted from the observed values of the independents. 

Like any probability, the likelihood varies from 0 to 1. The log likelihood (LL) is its log and varies from 0 to minus infinity (it is negative because the log of any number less than 1 is negative). LL is calculated through iteration, using maximum likelihood estimation (MLE).


\end{document}
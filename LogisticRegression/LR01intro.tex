\documentclass[a4paper,12pt]{article}
%%%%%%%%%%%%%%%%%%%%%%%%%%%%%%%%%%%%%%%%%%%%%%%%%%%%%%%%%%%%%%%%%%%%%%%%%%%%%%%%%%%%%%%%%%%%%%%%%%%%%%%%%%%%%%%%%%%%%%%%%%%%%%%%%%%%%%%%%%%%%%%%%%%%%%%%%%%%%%%%%%%%%%%%%%%%%%%%%%%%%%%%%%%%%%%%%%%%%%%%%%%%%%%%%%%%%%%%%%%%%%%%%%%%%%%%%%%%%%%%%%%%%%%%%%%%
\usepackage{eurosym}
\usepackage{vmargin}
\usepackage{amsmath}
\usepackage{graphics}
\usepackage{epsfig}
\usepackage{subfigure}
\usepackage{fancyhdr}
%\usepackage{listings}
\usepackage{framed}
\usepackage{graphicx}

\setcounter{MaxMatrixCols}{10}
%TCIDATA{OutputFilter=LATEX.DLL}
%TCIDATA{Version=5.00.0.2570}
%TCIDATA{<META NAME="SaveForMode" CONTENT="1">}
%TCIDATA{LastRevised=Wednesday, February 23, 2011 13:24:34}
%TCIDATA{<META NAME="GraphicsSave" CONTENT="32">}
%TCIDATA{Language=American English}

\pagestyle{fancy}
\setmarginsrb{20mm}{0mm}{20mm}{25mm}{12mm}{11mm}{0mm}{11mm}
\lhead{Dublin \texttt{R}} \rhead{10 April 2013}
\chead{Introduction to \texttt{R} (Module A)}
%\input{tcilatex}


\begin{document}

\tableofcontents
\newpage
\subsection{Introduction to Logistic Regression
%=========================================%
Logistic regression determines the impact of multiple independent variables
presented simultaneously to predict membership of one or other of the two
dependent variable categories.


\newpage
\section{Logistic Regression}
Logistic regression determines the impact of multiple independent variables
presented simultaneously to predict membership of one or other of the two
dependent variable categories.

\subsection{The purpose of logistic regression}
The crucial limitation of linear regression is that it cannot deal with Dependent Variables’s that are \textbf{\textit{dichotomous}} and categorical. Many interesting variables in the business world are dichotomous: for
example, consumers make a decision to buy or not buy (\textit{\textbf{Buy/Don't Buy}}), a product may pass or fail quality control (\textit{\textbf{Pass/Fail}}), there are good or poor credit risks (\textit{\textbf{Good/Poor}}), an employee may be promoted or not (\textit{\textbf{Promote/Don't Promote}}).


A range of regression techniques have been developed for analysing data with categorical dependent
variables, including logistic regression and discriminant analysis (Hence referred to as DA, which is the next section of course).

Logistical regression is regularly used rather than discriminant analysis when there are only two categories
for the dependent variable. Logistic regression is also easier to use with SPSS than DA when
there is a mixture of numerical and categorical Independent Variables’s, because it includes procedures for
generating the necessary dummy variables automatically, requires fewer assumptions, and
is more statistically robust. DA strictly requires the continuous independent variables  (though dummy variables can be used as in multiple regression). Thus, in instances where
the independent variables are categorical, or a mix of continuous and categorical, and the
DV is categorical, logistic regression is necessary.

\subsection{Use of Binomial Probability Theory}
Since the dependent variable is dichotomous we cannot predict a numerical value for it
using logistic regression, so the usual regression least squares deviations criteria for best fit
approach of minimizing error around the line of best fit is inappropriate.

Instead, logistic regression employs binomial probability theory in which there are only two values to
predict: that probability (p) is 1 rather than 0, i.e. the event/person belongs to one group
rather than the other. Logistic regression forms a best fitting equation or function using the
maximum likelihood method (not part of course), which maximizes the probability of classifying the observed
data into the appropriate category given the regression coefficients.

\newpage
\section{Logistic Regression}
Logistic regression determines the impact of multiple independent variables
presented simultaneously to predict membership of one or other of the two
dependent variable categories.

\subsection{The purpose of logistic regression}
\begin{itemize}
	\item The crucial limitation of linear regression is that it cannot deal with Dependent Variables’s that are \textbf{\textit{dichotomous}} and categorical. Many interesting variables in the business world are dichotomous: for
	example, consumers make a decision to buy or not buy (\textit{\textbf{Buy/Don't Buy}}), a product may pass or fail quality control (\textit{\textbf{Pass/Fail}}), there are good or poor credit risks (\textit{\textbf{Good/Poor}}), an employee may be promoted or not (\textit{\textbf{Promote/Don't Promote}}).
	
	\item 	A range of regression techniques have been developed for analysing data with categorical dependent
	variables, including logistic regression and discriminant analysis (Hence referred to as DA, which is the next section of course).
	
	\item 	Logistical regression is regularly used rather than discriminant analysis when there are only two categories
	for the dependent variable. Logistic regression is also easier to use with SPSS than DA when
	there is a mixture of numerical and categorical Independent Variables’s, because it includes procedures for
	generating the necessary dummy variables automatically, requires fewer assumptions, and
	is more statistically robust. DA strictly requires the continuous independent variables  (though dummy variables can be used as in multiple regression). Thus, in instances where
	the independent variables are categorical, or a mix of continuous and categorical, and the
	DV is categorical, logistic regression is necessary.
\end{itemize}


\subsection{Use of Binomial Probability Theory}
\begin{itemize}
	\item Since the dependent variable is dichotomous we cannot predict a numerical value for it
	using logistic regression, so the usual regression least squares deviations criteria for best fit
	approach of minimizing error around the line of best fit is inappropriate.
	
	\item 	Instead, logistic regression employs binomial probability theory in which there are only two values to
	predict: that probability (p) is 1 rather than 0, i.e. the event/person belongs to one group
	rather than the other.
	\item Logistic regression forms a best fitting equation or function using the
	maximum likelihood method (not part of course), which maximizes the probability of classifying the observed
	data into the appropriate category given the regression coefficients.
\end{itemize}


\section{Introduction to Logistic Regression}
\begin{itemize}
	\item Logistic regression or logit regression is a type of probabilistic statistical classification model.
	
	\item It is also used to predict a binary response from a binary predictor, used for predicting the outcome of a categorical dependent variable (i.e., a class label) based on one or more predictor variables (features). 
	
	\item That is, it is used in estimating empirical values of the parameters in a qualitative response model. The probabilities describing the possible outcomes of a single trial are modeled, as a function of the explanatory (predictor) variables, using a logistic function. 
	
	\item Logistic regression, also called a logit model, is used to model \textbf{dichotomous (i.e. Binary) outcome variables}. In the logit model the log odds of the outcome is modeled as a linear combination of the predictor variables.
	
	\item 
	Binary Logistic regression is used to determine the impact of multiple independent variables
	presented simultaneously to predict membership of one or other of the two
	dependent variable categories.
	\item Logistic regression determines the impact of multiple independent variables
	presented simultaneously to predict membership of one or other of the two
	dependent variable categories.
	\item However, if your dependent variable was not measured on a dichotomous scale, but a continuous scale instead, you will need to carry out \textbf{multiple regression}, whereas if your dependent variable was measured on an ordinal scale, \textbf{ordinal regression} would be a more appropriate starting point.
\end{itemize}



\section*{Introduction to Logistic Regression}

The term ‘\textbf{\textit{generalized linear model}}’ is used to describe a procedure for
transforming the dependent variable so that the ‘right hand side’ of the model
equation can be interpreted as a \textbf{‘\textit{linear combination}}’ of the explanatory variables. 	In logistic regression, the logit may be computed in a manner similar to linear regression:
\[ \eta_i = \beta_0 + \beta_1x_1 + \beta_2x_2 + \ldots  \]

In situations where the dependent (y) variable is continuous and can be
reasonably assumed to have a normal distribution we do not transform the y
variable at all and we can simply run a multiple linear regression analysis.

Otherwise some sort of transformation is applied.


%---------------------------------------------------------------%
\subsection{Binomial Logistic Regression} 
A binomial logistic regression (often referred to simply as logistic regression), predicts the probability that an observation falls into one of two categories of a \textbf{dichotomous} dependent variable based on one or more independent variables that can be either continuous or categorical.

% \section{Binomial Logistic Regression}
Binomial logistic regression estimates the probability of an event (as an example, having heart disease) occurring. 
\begin{itemize}
	\item If the estimated probability of the event occurring is greater than or equal to 0.5 (better than even chance), the procedure classifies the event as occurring (e.g., heart disease being present). \item If the probability is less than 0.5, Logistic regression classifies the event as not occurring (e.g., no heart disease). 
\end{itemize}

\subsection{Examples of Logistic Regression}

\begin{description}
	\item[Example 1:]  Suppose that we are interested in the factors that influence whether a political candidate wins an election.  The outcome (response) variable is binary (0/1); \textit{ win or lose}.  The predictor variables of interest are the amount of money spent on the campaign, the amount of time spent campaigning negatively and whether or not the candidate is an incumbent.
	
	\item[Example 2:]  A researcher is interested in how variables, such as GRE (Graduate Record Exam scores), GPA (grade point average) and prestige of the undergraduate institution, effect admission into graduate school. The response variable, \textit{admit/don't admit}, is a binary variable.
\end{description}


\section{Review of Logistic Regression}
% http://www.nesug.org/proceedings/nesug06/an/da26.pdf
% http://www.ccsr.ac.uk/publications/teaching/blr.pdf
% http://www.southampton.ac.uk/ghp3/docs/unicef/presentation7.1a.pdf
% ftp://public.dhe.ibm.com/software/analytics/spss/documentation/statistics/20.0/en/client/Manuals/IBM_SPSS_Regression.pdf
% http://www.umass.edu/statdata/statdata/data/

%---------------------------------------------------------------%
\subsection{Generalized Linear Model}

The term ‘generalized linear model’ is used to describe a procedure for
transforming the dependent variable so that the ‘right hand side’ of the model
equation can be interpreted as a ‘linear combination’ of the explanatory variables:

\[ \operatorname{F(y)} = b_0 + b_1x_1 + b_2x_2 + b_3x_3 \ldots b_kx_k + e \]

In situations where the dependent (y) variable is continuous and can be
reasonably assumed to have a normal distribution we do not transform the y
variable at all and we can simply run a multiple linear regression analysis.

Otherwise some sort of transformation is applied.
%---------------------------------------------------------------%
\subsection{Logistic Regression}
logistic regression or logit regression is a type of regression analysis used for predicting
the outcome of a categorical dependent variable (a dependent variable that can take on a limited number of values,
whose magnitudes are not meaningful but whose ordering of magnitudes may or may not be meaningful)
based on one or more predictor variables.



\begin{itemize}
	\item[1.)] Logistic regression is intended for the modeling
	of dichotomous categorical outcomes (e.g., characterized by binary responses: buy vs Don't buy, dead vs. alive, cancer vs. none,…).
	
	
	\item[2.)] We want to predict the probability of a particular response  (0 to 1 scale).
	
	\item[3.)] For binary responses, linear regression should not be used for several reasons
	but the most common-sense reason is that linear regression can provide predictions NOT on a 0 to 1 scale.
	but rather a predicted response of some numeric value (e.g 2.4 or -800.3).
	
	\item[4.)] We need a way to link the probabilistic response variable to the continuous and/or categorical predictors and
	keep things on this 0 to 1 scale.
	
	\item[5.)] Logistic regression winds up transforming the probabilities to odds and then taking the natural logarithm of these odds, called logits.
	
	
	\item[6.)] Suppose a response variable is passing a test (by convention, 0=no and 1=yes).
	You have 1 predictor - number of days present in class over the past 30 days.
	Suppose the regression coefficient (often just called beta) in the output is .14.
	You would then say that, on average, as class presence increases by 1 day, the natural logarithm of the
	odds of passing the test increases by .14.
	
	\item[7.)] For the interpretation, you can just talk about the odds.
	Most computer output will give you this number.
	Suppose the answer in odds is 1.24. Then, you just say that,on average, as class presence increases by 1 day,
	the odds of passing the test are multiplied by 1.24.
	In other words, for each additional day present, the odds of passing are 24% greater than that of not passing.
	
	\item[8.)] To validate our findings, normally, we test whether the regression coefficient is equal to zero in the population.
	In logistic regression, the corresponding value for the odds is one (not zero). We got an odds of 1.24.
	Can we trust this? Or should we go with one (which would mean that the odds are the same for both passing and not passing,
	and hence class presence makes no difference at all)?  Look at the p-value (significance). If it less than .05 (by convention), you have enough evidence to reject
	the notion that the odds are really one. You go ahead and support the 1.24 result.
\end{itemize}

\end{document}
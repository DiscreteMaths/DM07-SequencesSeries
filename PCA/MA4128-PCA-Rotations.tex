
\documentclass[a4paper,12pt]{article}
%%%%%%%%%%%%%%%%%%%%%%%%%%%%%%%%%%%%%%%%%%%%%%%%%%%%%%%%%%%%%%%%%%%%%%%%%%%%%%%%%%%%%%%%%%%%%%%%%%%%%%%%%%%%%%%%%%%%%%%%%%%%%%%%%%%%%%%%%%%%%%%%%%%%%%%%%%%%%%%%%%%%%%%%%%%%%%%%%%%%%%%%%%%%%%%%%%%%%%%%%%%%%%%%%%%%%%%%%%%%%%%%%%%%%%%%%%%%%%%%%%%%%%%%%%%%
\usepackage{eurosym}
\usepackage{vmargin}
\usepackage{amsmath}
\usepackage{graphics}
\usepackage{epsfig}
\usepackage{subfigure}
\usepackage{fancyhdr}
\usepackage{listings}
\usepackage{framed}
\usepackage{graphicx}

\setcounter{MaxMatrixCols}{10}
%TCIDATA{OutputFilter=LATEX.DLL}
%TCIDATA{Version=5.00.0.2570}
%TCIDATA{<META NAME="SaveForMode" CONTENT="1">}
%TCIDATA{LastRevised=Wednesday, February 23, 2011 13:24:34}
%TCIDATA{<META NAME="GraphicsSave" CONTENT="32">}
%TCIDATA{Language=American English}

\pagestyle{fancy}
\setmarginsrb{20mm}{0mm}{20mm}{25mm}{12mm}{11mm}{0mm}{11mm}
\lhead{MA4128} \rhead{Mr. Kevin O'Brien}
\chead{Principal Component Analysis}
%\input{tcilatex}

\begin{document}
	

\subsection{What is a Rotation}
\begin{itemize}
	\item Ideally, you would like to review the correlations between the variables and the
	components and use this information to interpret the components; that is, to determine what
	construct seems to be measured by component 1, what construct seems to be measured by
	component 2, and so forth. 
	\item Unfortunately, when more than one component has been retained in
	an analysis, the interpretation of an unrotated factor pattern is usually quite difficult. To make
	interpretation easier, you will normally perform an operation called a rotation. 
	\item A rotation is a
	linear transformation that is performed on the factor solution for the purpose of making the
	solution easier to interpret.
\end{itemize}

%-------------------------------------------------------------------------------------------------------%

\subsection{Varimax Rotation}
A varimax rotation is an orthogonal rotation, meaning that
it results in uncorrelated components. Compared to some other types of rotations, a varimax
rotation tends to maximize the variance of a column of the factor pattern matrix (as opposed to a
row of the matrix). This rotation is probably the most commonly used orthogonal rotation in the
social sciences.

\subsection{Interpreting the Rotated Solution}
\begin{itemize}
\item Interpreting a rotated solution means determining just what is measured by each of the retained
components. Briefly, this involves identifying the variables that demonstrate high loadings for a
given component, and determining what these variables have in common. Usually, a brief name
is assigned to each retained component that describes its content.
\item The first decision to be made at this stage is to decide how large a factor loading must be to be
considered ``large."
\item 
Guidelines are provided in statistical literature for testing the statistical significance of factor loadings. Given that this is an introductory treatment of principal component analysis, however, simply consider a loading
to be large if its absolute value exceeds 0.40.
\end{itemize}

\end{document}
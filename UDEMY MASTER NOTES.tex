%\documentclass[11pt, a4paper,dalthesis]{report}    % final 
%\documentclass[11pt,a4paper,dalthesis]{report}
%\documentclass[11pt,a4paper,dalthesis]{book}

\documentclass[11pt,a4paper,titlepage,oneside,openany]{article}

\pagestyle{plain}
%\renewcommand{\baselinestretch}{1.7}

\usepackage{setspace}
%\singlespacing
\onehalfspacing
%\doublespacing
%\setstretch{1.1}

\usepackage{amsmath}
\usepackage{amssymb}
\usepackage{amsthm}

\usepackage[margin=3cm]{geometry}
\usepackage{graphicx,psfrag}%\usepackage{hyperref}


\usepackage{framed}
\include{zcust_preamble}
\include{zcust_commands}

\begin{document}

\section*{Sequences and limits}

Heron's method and the reciprocal algorithm both generated \emph{sequences} of numbers which hopefully converged to a \emph{limit}. We need to study and understand these objects and concepts.

A sequence is an ordered list of numbers given by a definite rule. A sequence of numbers is written in the form
\begin{equation*}
 a_0,a_1,a_2,a_3,a_4,a_5,a_6,\ldots
\end{equation*}
The term $a_n$ denotes the $n^{th}$ \emph{term} in the sequence, \eg $a_{105}$ denotes the $105^{th}$ term. Note that $n$ must be a positive integer ($n \in \mathbb{N}_0$). It cannot be negative or a fraction, \etc. Some sequences start at $n=0$, other can start at $n=1$. A sequence can be \emph{finite}, with the terms ending after a large enough value of $n$ is reach. Sequences can also be \emph{infinite}, with the terms continuing forever without ending.

The \emph{rule} which defines the sequence may take many forms but it must define all terms in the sequence unambiguously.

\subsection*{Simple Rules}

A simple rule for a sequence is a formula which has the same behavior/form for all inputs $n$. For example
\begin{align*}
%  a_n=&n\ , \quad &a_0=0,a_1=1,a_2=2,a_3=3,\ldots \\
  a_n=&2n^2+3n\\%\ , \quad &a_0=0,a_1=5,a_2=14,a_3=27,\ldots\\
  b_n=&\frac{1}{n+1}\\%\ , \quad &a_0=1,a_1=\frac{1}{2},a_2=\frac{1}{3},a_3=\frac{1}{4},\ldots\\
  c_n=&2^n+3^n%\ , \quad &a_0=2,a_1=5,a_2=13,a_3=35,\ldots
\end{align*}
The rule is of the form $a_n=f(n)$ for some simple function $f$. Different letter can be used to distinguish terms in different sequences. Some terms in the sequences above are listed in the tables below.

\begin{center}

\begin{tabular}{|c|c|c|c|c|c|}
\hline
$a_0$ & $a_1$ & $a_2$ & $a_3$ & $a_4$ & $a_5$ \\
\hline
  0 & 5 & 14 & 27 & 44 & 65  \\
\hline
\end{tabular}
\end{center}

\begin{center}
\begin{tabular}{|c|c|c|c|c|c|}
\hline
$b_0$ & $b_1$ & $b_2$ & $b_3$ & $b_4$ & $b_5$ \\
\hline
  1 & $\frac{1}{2}$ & $\frac{1}{3}$  & $\frac{1}{4}$  & $\frac{1}{5}$  & $\frac{1}{6}$   \\
\hline
\end{tabular}
\end{center}

\begin{center}
\begin{tabular}{|c|c|c|c|c|c|}
\hline
$c_0$ & $c_1$ & $c_2$ & $c_3$ & $c_4$ & $c_5$ \\
\hline
  2 & 5 & 13 & 35 & 97 & 275  \\
\hline
\end{tabular}
\end{center}

\subsection*{Piecewise Rules}

A piecewise rule for a sequence is one which changes its formula depending on the input $n$. The input must first be checked and matched to a particular formula before evaluating the term. 

\noindent {\bf Example:}
\begin{align*}
  a_n&= \begin{cases}
    n^2 &,\qquad n < 3\\
    -2n &,\qquad n \ge 3
  \end{cases}
\end{align*}
The sequence $a_n$ is given by the formula $n^2$ when $n<10$ and by the formula $-2n$ when $n \ge 10$. Before evaluation, the input $n$ much be checked to see which of the conditions it satisfies.

For example to evaluate the term $a_2$, the input $n=2$ is seen to satisfy the condition $n<3$. Therefore the formula $n^2$ is used, and so $a_2=2^2=4$. To evaluate the term $a_5$, the input $n=5$ is seen to satisfy the condition $n \ge3$. Therefore the formula $-2n$ is used, and so $a_5=-2(5) = -10$. In this way, the terms of the sequence can be evaluated as long as the inputs $n$ are matched to their proper conditions. Further terms in this sequence are listed in the table below.
\begin{center}
\begin{tabular}{|c|c|c|c|c|c|}
\hline
$a_0$ & $a_1$ & $a_2$ & $a_3$ & $a_4$ & $a_5$ \\
\hline
  0 & 1 & 4 & -6 & -8 & -10  \\
\hline
\end{tabular}
\end{center}

The sequences $b_n$ and $c_n$ below are also defined using piecewise rules. Some terms in the sequences $b_n$ and $c_n$ are listed in the tables below.

The conditions for the sequence $b_n$ check whether the input $n$ is even$(0,2,4,6,\ldots)$ or odd$(1,3,5,7,\ldots)$. The sequence $c_n$ has $3$ formulas, and $3$ corresponding conditions on the input $n$. The input must be checked against all listed conditions until a suitable formula is found.
\begin{align*}
  b_n&= \begin{cases}
    2n^2+3n&,\qquad n \text{ even}\\
    \frac{1}{n} &,\qquad n \text{ odd}
  \end{cases}\\
  c_n&= \begin{cases}
    0 &,\qquad n<2\\
    n^2 &,\qquad n=2\\
    n^3 &,\qquad n>2
  \end{cases}
\end{align*}


\begin{center}
\begin{tabular}{|c|c|c|c|c|c|}
\hline
$b_0$ & $b_1$ & $b_2$ & $b_3$ & $b_4$ & $b_5$ \\
\hline
  0 & $\frac{1}{1}$ & 14 & $\frac{1}{3}$ & 44 & $\frac{1}{5}$ \\
\hline
\end{tabular}
\end{center}

\begin{center}
\begin{tabular}{|c|c|c|c|c|c|}
\hline
$c_0$ & $c_1$ & $c_2$ & $c_3$ & $c_4$ & $c_5$ \\
\hline
  0 & 0 & 4 & 27 & 64 & 125  \\
\hline
\end{tabular}
\end{center}

To represent piecewise rules on a computer, some kind of if/else statement or other control flow statements are needed. For multiple conditions, a switch statement can be used.

Piecewise and simple rules are convenient as they allow any term in a sequence to be immediately evaluated without needed to evaluate any previous terms. For example $b_{100}=2(100)^2+3(100)=20300$, and $a_{1000000} = -2000000$.

Piecewise rules can also be used to define more general function. For example, the modulus or absolute value function $|x|$, or $\text{abs}(x)$, can be defined using the rule.
\begin{equation*}
  |x| =
  \begin{cases}
    x, \qquad x \ge 0\\
    -x, \qquad x<0
  \end{cases}
\end{equation*}
This function strips off the negative sign of any number, but leaves positive numbers unchanged. It gives the ``size'' of a number, or its distance from $0$. The function $|x|$ is plotted in Figure \vref{fig:abs_x}.

\begin{figure}
  \centering
  \includegraphics[width=0.4\textwidth]{images/abs_x.eps}
  \caption{The absolute value function $|x|$, also written $\text{abs}(x)$}
  \label{fig:abs_x}
\end{figure}


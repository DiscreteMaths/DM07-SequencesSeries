openssl

curl

xml2

httr


\subsection{Multi-collinearity}
When choosing a predictor variable you should select one that might be correlated with the criterion variable, but that is not strongly correlated with the other predictor variables. However, correlations amongst the predictor variables are not unusual. The term multi-collinearity  is used to describe the situation
when a high correlation is detected between two or more predictor variables.
Such high correlations cause problems when trying to draw inferences about the relative contribution of each predictor variable to the success of the model. 

\subsection{Variance Inflation Factor (VIF)}



The Variance Inflation Factor (VIF) measures the impact of multi-collinearity among the variables in a regression model. 
%Variance Inflation Factor (VIF) is 1/Tolerance, it is always greater than or equal to 1.
     
There is no formal VIF valu-e for determining presence of multi-collinearity. Values of VIF that exceed 10 are often regarded as indicating multicollinearity, but in weaker models values above 2.5 may be a cause for concern. In many statistics programs, the results are shown both as an individual $R^2$ value (distinct from the overall $R^2$ of the model) and a Variance Inflation Factor (VIF). When those $R^2$ and VIF values are high for any of the variables in your model, multi-collinearity is probably an issue. 

%When VIF is high there is high multi-collinearity and instability of the regression estimates. It is often difficult to sort this out.
     
%The variance inflation factor (or ``VIF") provides us with a measure of how much the variance for a given regression coefficient is increased compared to if all predictors were uncorrelated. To understand what the variance inflation factor is, and what it measures, we need to examine the computation of the standard error of a regression coefficient.

\subsection{Tolerance}

Tolerance is simply the reciprocal of VIF, and is computed as
\[ \mbox{Tolerance} = \frac{1}{VIF}\]
Whereas large values of VIF were unwanted and undesirable, since tolerance is the reciprocal of VIF, larger than not values of tolerance are indicative of a lesser problem with collinearity. In other words, we want large tolerances.




\subsection{Boosting}

Boosting is a machine learning ensemble meta-algorithm for primarily reducing bias, and also variance in supervised learning, and a family of machine learning algorithms which convert weak learners to strong ones.[2] Boosting is based on the question posed by Kearns and Valiant (1988, 1989):[3][4] Can a set of weak learners create a single strong learner? A weak learner is defined to be a classifier which is only slightly correlated with the true classification (it can label examples better than random guessing). In contrast, a strong learner is a classifier that is arbitrarily well-correlated with the true classification.

Robert Schapire's affirmative answer in a 1990 paper to the question of Kearns and Valiant has had significant ramifications in machine learning and statistics, most notably leading to the development of boosting.[6]


%==========================================================%
\subsection{AdaBoost}
\begin{itemize}
\item AdaBoost, short for \textbf{"Adaptive Boosting"}, is a machine learning meta-algorithm formulated by Yoav Freund and Robert Schapire who won the Gödel Prize in 2003 for their work. It can be used in conjunction with many other types of learning algorithms to improve their performance. The output of the other learning algorithms ('weak learners') is combined into a weighted sum that represents the final output of the boosted classifier. AdaBoost is adaptive in the sense that subsequent weak learners are tweaked in favor of those instances misclassified by previous classifiers. AdaBoost is sensitive to noisy data and outliers. In some problems it can be less susceptible to the overfitting problem than other learning algorithms. The individual learners can be weak, but as long as the performance of each one is slightly better than random guessing (e.g., their error rate is smaller than 0.5 for binary classification), the final model can be proven to converge to a strong learner.
\item 
While every learning algorithm will tend to suit some problem types better than others, and will typically have many different parameters and configurations to be adjusted before achieving optimal performance on a dataset, AdaBoost (with decision trees as the weak learners) is often referred to as the best out-of-the-box classifier.[1][2] When used with decision tree learning, information gathered at each stage of the AdaBoost algorithm about the relative 'hardness' of each training sample is fed into the tree growing algorithm such that later trees tend to focus on harder-to-classify examples.
\end{itemize}

\end{document}


\documentclass[]{report}

\voffset=-1.5cm
\oddsidemargin=0.0cm
\textwidth = 480pt

%\usepackage{framed}
%\usepackage{subfiles}
%\usepackage{enumerate}
%\usepackage{graphics}
%\usepackage{newlfont}
%\usepackage{eurosym}
%\usepackage{amsmath,amsthm,amsfonts}
%\usepackage{amsmath}
%\usepackage{color}
%\usepackage{amssymb}
%\usepackage{multicol}
%\usepackage[dvipsnames]{xcolor}
%\usepackage{graphicx}
\begin{document}


\section*{Multicollinearity}
\begin{itemize}

\item  Multi-collinearity: Multicollinearity occurs when two or more predictors in the model are correlated
and provide redundant information about the response.
\item  Examples of pairs of multicollinear predictors are years of education and income, height and weight of a
person, and assessed value and square footage of a house.
\item  Consequences of high multicollinearity:
Multicollinearity leads to decreased reliability and predictive power of statistical models, and hence, very
often, confusing and misleading results.
\item Consequences of high multicollinearity:
\begin{enumerate}
\item Increased standard error of estimates of the regression coefficients (i.e. decreased reliability of fitted
model).
\item Often confusing and misleading results.
\end{enumerate}
\item  Multicollinearity will be dealt with in a future component of this course: Variable Selection Procedures.
\item  This issue is not a serious one with respect to the usefulness of the overall model, but it does affect any attempt to interpret the meaning of the partial regression coefficients in the model.
\item  When choosing a predictor variable you should select one that might be correlated with the criterion
variable, but that is not strongly correlated with the other predictor variables. However, correlations
amongst the predictor variables are not unusual.
\item  The term multicollinearity is used to describe the situation when a high correlation is detected between
two or more predictor variables.
\item  Such high correlations cause problems when trying to draw inferences about the relative contribution of
each predictor variable to the success of the model.
\end{itemize}
%======================================================%
\subsection{Variance Inflation Factor (VIF)}
\begin{itemize}
\item  The Variance In
ation Factor (VIF) measures the impact of multicollinearity among the variables in a
regression model.
\item  There is no formal VIF value for determining presence of multicollinearity. Values of VIF that exceed 10
are often regarded as indicating multicollinearity, but in weaker models values above 2.5 may be a cause
for concern.
\item  In many statistics programs, the results are shown both as an individual R2 value (distinct from the
overall R2 of the model) and a Variance Inflation Factor (VIF).
\item  When those R2 and VIF values are high for any of the variables in your model, multi-collinearity is
probably an issue.

\item \textbf{Multi-collinearity}: Multicollinearity occurs when two or more predictors in the model are
correlated and provide redundant information about the response. Examples of pairs of multicollinear predictors are years of education and income, height and weight of a person, and assessed value and square footage
of a house.

\item \textbf{Consequences of high multicollinearity}:
Multicollinearity leads to decreased reliability and predictive power of statistical models, and hence, very often, confusing and misleading results.
\item Multicollinearity will be dealt with in a future component of this course: Variable Selection Procedures.
\item This issue is not a serious one with respect to the
usefulness of the overall model, but it does affect any attempt to interpret the meaning of the partial regression
coefficients in the model.


\end{itemize}
%======================================================%
\section*{Variance Infiation Factor}
\begin{itemize}
\item  The variance inflation factor (VIF) is used to detect whether one predictor has a strong linear association
with the remaining predictors (the presence of multicollinearity among the predictors).
\item  VIF measures how much the variance of an estimated regression coefficient increases if your predictors
are correlated (multicollinear). VIF = 1 indicates no relation; VIF > 1, otherwise.

\item  The variance inflation factor (VIF) quantifies the severity of multicollinearity in a regression analysis.
\item  The VIF provides an index that measures how much the variance (the square of the estimate's standard
deviation) of an estimated regression coefficient is increased because of collinearity.

\item  You should consider the options to break up the multicollinearity: collecting additional data, deleting predictors, using different predictors, or 
an alternative to least square regression.
\end{itemize}




\subsection*{Multicollinearity}
\begin{itemize}
\item When choosing a predictor variable you should select one that might be correlated with the
criterion variable, but that is not strongly correlated with the other predictor variables. How-
ever, correlations amongst the predictor variables are not unusual.
\item  The term multi-collinearity
is used to describe the situation when a high correlation is detected between two or more pre-
dictor variables. 
\item Such high correlations cause problems when trying to draw inferences about
the relative contribution of each predictor variable to the success of the model.
\end{itemize}


%===============================================%
\section*{Variance Inflation Factor (VIF)}
\begin{itemize}
\item The Variance Inflation Factor (VIF) measures the impact of multicollinearity among the variables in a regression model.
\item There is no formal VIF value for determining presence of multicollinearity. Values of VIF
that exceed 10 are often regarded as indicating multicollinearity, but in weaker models values
above 2.5 may be a cause for concern. 
\item In many statistics programs, the results are shown both
as an individual $R^2$ value (distinct from the overall $R^2$ of the model) and a Variance Inflation
Factor (VIF). 
\item When those $R^2$ and VIF values are high for any of the variables in your model,
multi-collinearity is probably an issue.

\end{itemize}




%===============================================%
\section*{Variance Inflation Factor}
\begin{itemize}
\item The variance inflation factor (VIF) is used to detect whether one predictor has a strong linear
association with the remaining predictors (the presence of multicollinearity among the
predictors). 
\item VIF measures how much the variance of an estimated regression coefficient
increases if your predictors are correlated (multicollinear).
VIF = 1 indicates no relation; VIF > 1, otherwise.
\item The variance in
ation factor (VIF) quanties the severity of multicollinearity in a regression
analysis.
\item The VIF provides an index that measures how much the variance (the square of the es-
timate's standard deviation) of an estimated regression coefficient is increased because of
collinearity.

\end{itemize}


%===============================================%

\section{Interpreting VIF Multicollinearity}

\begin{itemize}
\item  A common rule of thumb is that if the VIF is greater than 5 then multicollinearity is high. Also a VIF
level of 10 has been proposed as a cut off value.
\item  The largest VIF among all predictors is often used as an indicator of severe multicollinearity.
\item  Montgomery and Peck [21] suggest that when VIF is greater than 5-10, then the regression coefficients are poorly estimated.
\item A common rule of thumb is that if the VIF is greater than 5 then multicollinearity is high.
Also a VIF level of 10 has been proposed as a cut off value.
\item The largest VIF among all predictors is often used as an indicator of severe multicollinearity. 
\item Montgomery and Peck [21] suggest that when VIF is greater than 5-10, then the regression coefficients are poorly estimated. \item You should
consider the options to break up the multicollinearity: collecting additional data, deleting
predictors, using different predictors, or an alternative to least square regression.
\end{itemize}

\subsection*{Tolerance}
Tolerance is simply the reciprocal of VIF, and is computed as


\[ Tolerance = \frac{1}{VIF}\]

Whereas large values of VIF were unwanted and undesirable, since tolerance is the reciprocal
of VIF, larger than not values of tolerance are indicative of a lesser problem with collinearity.
In other words, we want large tolerances.


\newpage
%%-- Section 10.4 
\end{document}
%--------------------------------------------------------------------------------------------------Advanced R Coding%
\newpage
\chapter{Advanced R code}
\section{Data frame}
A Data frame is
\subsection{Merging Data frames}

\section{The Apply family}

Sometimes want to apply a function to each element of a
vector/data frame/list/array.
\\
Four members: lapply, sapply, tapply, apply
\\
lapply: takes any structure and gives a list of results (hence
the `l')
\\
sapply: like lapply, but tries to simplify the result to a
vector or matrix if possible (hence the `s')
\\
apply: only used for arrays/matrices
\\
tapply: allows you to create tables (hence the `t') of values
from subgroups defined by one or more factors.

\section{Writing Functions}

A simple function can be constructed as follows:

\begin{verbatim}
function_name <- function(arg1, arg2, ...){
commands
output
}
\end{verbatim}

You decide on the name of the function. The function command shows R that you are writing a function. Inside the parenthesis you outline the input objects required and decide what to call them. The commands occur inside the { }.

The name of whatever output you want goes at the end of the function. Comments lines (usually a description of what the function does is placed at the beginning) are denoted by "\#".

\begin{verbatim}sf1 <- function(x){
x^2
}
\end{verbatim}

This function is called sf1. It has one argument, called x.
Whatever value is inputted for x will be squared and the result outputted to the screen. This function must be loaded into \texttt{R} and can then be called. We can call the function using:
\begin{verbatim}
sf1(x = 3)
#sf1(3)
[1] 9
To store the result into a variable x.sq
x.sq <- sf1(x = 3)
x.sq <- sf1(3)
> x.sq
[1] 9
\end{verbatim}
Example
\begin{verbatim}
sf2 <- function(a1, a2, a3){
x <- sqrt(a1^2 + a2^2 + a3^2)
return(x)
}
\end{verbatim}

This function is called sf2 with 3 arguments. The values inputted for a1, a2, a3 will be squared, summed and the square root of the sum calculated and stored in x. (There will be no output to the screen as in the last example.)
The return command specifies what the function returns, here the value of x. We will not be able to view the result of the function unless we store it.
\begin{verbatim}sf2(a1=2, a2=3, a3=4)
sf2(2, 3, 4) # Can't see result.
res <- sf2(a1=2, a2=3, a3=4)
res <- sf2(2, 3, 4) # Need to use this.
res
[1] 5.385165
\end{verbatim}
We can also give some/all arguments default values.
\begin{verbatim}mypower <- function(x, pow=2){
x^pow
}
\end{verbatim}
If a value for the argument pow is not specified in the function call,
a value of 2 is used.
\begin{verbatim}mypower(4)
[1] 16
\end{verbatim}
If a value for "pow" is specified, that value is used.
\begin{verbatim}
mypower(4, 3)
[1] 64
mypower(pow=5, x=2)
[1] 32
\end{verbatim}


\section{Functions}
Syntax to define functions

\begin{myindentpar}{1cm}
\begin{verbatim}
        myfct <- function(arg1, arg2, ...) { function_body }
\end{verbatim}
\end{myindentpar}
The value returned by a function is the value of the function body, which is usually an unassigned final expression, e.g.: return()

Syntax to call functions
\begin{myindentpar}{1cm}
\begin{verbatim}
        myfct(arg1=..., arg2=...)
\end{verbatim}
\end{myindentpar}


\section{Time and Date}
It is useful . The length of time a program takes is interesting.


\begin{myindentpar}{1cm}
\begin{verbatim}
date() # returns the current system date and time
\end{verbatim}
\end{myindentpar}



\newpage

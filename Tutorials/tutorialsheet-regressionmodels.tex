\documentclass[a4paper,12pt]{article}
%%%%%%%%%%%%%%%%%%%%%%%%%%%%%%%%%%%%%%%%%%%%%%%%%%%%%%%%%%%%%%%%%%%%%%%%%%%%%%%%%%%%%%%%%%%%%%%%%%%%%%%%%%%%%%%%%%%%%%%%%%%%%%%%%%%%%%%%%%%%%%%%%%%%%%%%%%%%%%%%%%%%%%%%%%%%%%%%%%%%%%%%%%%%%%%%%%%%%%%%%%%%%%%%%%%%%%%%%%%%%%%%%%%%%%%%%%%%%%%%%%%%%%%%%%%%
\usepackage{eurosym}
\usepackage{vmargin}
\usepackage{amsmath}
\usepackage{graphics}
\usepackage{epsfig}
\usepackage{subfigure}
\usepackage{fancyhdr}

\setcounter{MaxMatrixCols}{10}
%TCIDATA{OutputFilter=LATEX.DLL}
%TCIDATA{Version=5.00.0.2570}
%TCIDATA{<META NAME="SaveForMode" CONTENT="1">}
%TCIDATA{LastRevised=Wednesday, February 23, 2011 13:24:34}
%TCIDATA{<META NAME="GraphicsSave" CONTENT="32">}
%TCIDATA{Language=American English}

\pagestyle{fancy}
\setmarginsrb{20mm}{0mm}{20mm}{25mm}{12mm}{11mm}{0mm}{11mm}
\lhead{MA4128} \rhead{Mr. Kevin O'Brien}
\chead{Advanced Data Modelling}
%\input{tcilatex}

\begin{document}
	\section*{Regression Models: Tutorial Sheet 3B}
\begin{enumerate}

\item Describe how to use to the Akaike Information Criterion for model selection.
\item Compare and contrast three types of variable selection procedure.
\item Explain what variable selection procedures are used for.
	
%============================%
\item Model Selection Question
$x_1$, $x_2$,$x_3$ and $x_4$.

Suppose we have 5 predictor variables.
Use \textbf{Forward Selection} and \textbf{Backward Selection} to choose the optimal set of predictor variables, based on the AIC measure.

{
	\large
	\begin{center}
		\begin{tabular}{||c|c||c|c||}
			\hline
			Variables & AIC & Variables & AIC \\ \hline \hline
			$\emptyset$	&	200	&	x1, x2, x3	&	74	\\ \hline
			\phantom{makemakespace}
			&	\phantom{makespace}
			&	x1, x2, x4	&	75	\\ \hline
			x1	&	150	&	x1, x2, x5	&	79	\\ \hline
			x2	&	145	&	x1, x3, x4	&	72	\\ \hline
			x3	&	135	&	x1, x3, x5	&	85	\\ \hline
			x4	&	136	&	x1, x4, x5	&	95	\\ \hline
			x5	&	139	&	x2, x3, x4	&	83	\\ \hline
			&		&	x2, x3, x5	&	82	\\ \hline
			x1, x2	&	97	&	x2, x4, x5	&	78	\\ \hline
			x1, x3	&	81	&	x3, x4, x5	&	85	\\ \hline
			x1, x4	&	94	&	\phantom{makemakespace}
			&	\phantom{makespace}
			\\ \hline
			x1, x5	&	88	&	x1, x2, x3, x4	&	93	\\ \hline
			x2, x3	&	87	&	x1, x2, x3, x5	&	120	\\ \hline
			x2, x4	&	108	&	x1, x2, x4, x5	&	104	\\ \hline
			x2, x5	&	87	&	x1, x3, x4, x5	&	101	\\ \hline
			x3, x4	&	105	&	x2, x3, x4, x5	&	89	\\ \hline
			x3, x5	&	82	&		&		\\ \hline
			x4, x5	&	86	&	x1, x2, x3, x4, x5	&	100	\\ \hline
		\end{tabular} 
	\end{center}
}


\end{enumerate}

\end{document}
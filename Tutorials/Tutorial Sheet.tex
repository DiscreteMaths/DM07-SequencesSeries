\documentclass[a4paper,12pt]{article}
%%%%%%%%%%%%%%%%%%%%%%%%%%%%%%%%%%%%%%%%%%%%%%%%%%%%%%%%%%%%%%%%%%%%%%%%%%%%%%%%%%%%%%%%%%%%%%%%%%%%%%%%%%%%%%%%%%%%%%%%%%%%%%%%%%%%%%%%%%%%%%%%%%%%%%%%%%%%%%%%%%%%%%%%%%%%%%%%%%%%%%%%%%%%%%%%%%%%%%%%%%%%%%%%%%%%%%%%%%%%%%%%%%%%%%%%%%%%%%%%%%%%%%%%%%%%
\usepackage{eurosym}
\usepackage{vmargin}
\usepackage{amsmath}
\usepackage{graphics}
\usepackage{epsfig}
\usepackage{subfigure}
\usepackage{fancyhdr}

\setcounter{MaxMatrixCols}{10}
%TCIDATA{OutputFilter=LATEX.DLL}
%TCIDATA{Version=5.00.0.2570}
%TCIDATA{<META NAME="SaveForMode" CONTENT="1">}
%TCIDATA{LastRevised=Wednesday, February 23, 2011 13:24:34}
%TCIDATA{<META NAME="GraphicsSave" CONTENT="32">}
%TCIDATA{Language=American English}

\pagestyle{fancy}
\setmarginsrb{20mm}{0mm}{20mm}{25mm}{12mm}{11mm}{0mm}{11mm}
\lhead{MA4128} \rhead{Mr. Kevin O'Brien}
\chead{Advanced Data Modelling}
%\input{tcilatex}

\begin{document}
Version: May 10th 2013




Explain how the Akaike Information Criterion would be used in the context of model selection.




Under what sort of modelling problem would you use Binary Logistic Regression?

Discuss some of the traditional technique for dealing with Missing Data, making references to the limitations of each.
\section*{Principal Component Analysis and Factor Analysis}
\begin{itemize}
\item[1.a] What is Dimensionality Reduction
\item[1.b] What is the KMO statistic? Describe how to interpret the KMO statistic.
\item[1.c] What is the Bartlett Test of Sphericity used for?
\item[1.d] varimax, quartimax and equamax are the commonly used methods in a certain procedure. What is this procedure? What is the purpose of the procedure.
Which method is the most commonly used?
\item[1.e] Describe how to use a Scree plot in the context of dimensionality reduction techniques.
\item[1.f] What problems occur if a principal component analysis is done on a data matrix where the columns contain measurements on very difference scales?  What can be done to overcome this problem?
\item[1.g] Principal Component Analysis is a data reduction technique. Explain what this term
means.
\item[1.h] The KMO is used to measure what characteristic of the data. Explain how the KMO
measure should be interpreted.
\item[1.i] Briefly describe the Bartlett Test for Sphericity, with reference to the null and alternative
hypotheses, and how those statements relate to the purpose of the test.
\item[1.j] Discuss three techniques for determining the appropriate number of principal components.
\item[1.k] In the context of principal components what is meant by orthogonality.

\item[1.l] What is the purpose of a principal component analysis?
\item[1.m] Explain the difference between PCA and factor analysis
\item[1.n] Explain what is meant by the "true" dimension of the data? How does an analyst determine the appropriate number of factors to retain. Make reference to three difference techniques
\end{itemize}


\section*{Missing Data}

\begin{itemize}
\item[5.a] Describe three types of missing data.
\item[5.b] what is meant by multiple imputation?
\item[5.c] Compare and contrast the following types of missing data: Missing At Random, Missing
Not At Random, Missing Completely at Random.
\item[5.d]Briefly describe the technique of Multiple Imputation.
\item[5.e] Discuss some of the traditional techniques for dealing with Missing Data. For each technique discuss the limitations of that technique.
\item[5.f] What is meant by missing data? Discuss the implications of Missing data in the context of a statistical analysis.
\end{itemize}



\section*{MANOVA and Discriminant Analysis}
\begin{itemize}
\item[6.a] The MANOVA procedure is sensitive to Multivariate Outliers. What is a multivariate outlier? Describe a method for detecting multivariate outliers.

\item[6.b] Pillai's Trace and Wilk's Lambda are two test procedures used in MANOVA, each fulfiling the same purpose.
Describe the purpose of these tests.

\item[6.c] What is the purpose of a discriminant analysis? How does discriminant analysis differ from MANOVA?

\item[6.d] Explain the following terms: confusion matrix, prior probabilities cost of misclassification, and apparent error rate.

\item[6.e] Distinguish between the True Error Rate and the Apparent Error Rate

\item[6.f]  What is the confusion matrix? Explain how it is interpreted.

\item[6.g] Explain why multinomial logistic regression may be used in preference to Discriminant Analysis.

\item[6.h] The apparent error rate calculated when all observations are used to construct
the discriminant rules is known to underestimate the true error rate. What can be done
to overcome this problem?

\item[6.i] Compare and contrast univariate and bivariate outliers. Describe how Mahalanobis Distance is used to detect bivariate outliers. Support your answer with an illustration.

\end{itemize}



\end{document}


%------------------------------------------------------------------------% 
\documentclass[a4paper,12pt]{article}
%%%%%%%%%%%%%%%%%%%%%%%%%%%%%%%%%%%%%%%%%%%%%%%%%%%%%%%%%%%%%%%%%%%%%%%%%%%%%%%%%%%%%%%%%%%%%%%%%%%%%%%%%%%%%%%%%%%%%%%%%%%%%%%%%%%%%%%%%%%%%%%%%%%%%%%%%%%%%%%%%%%%%%%%%%%%%%%%%%%%%%%%%%%%%%%%%%%%%%%%%%%%%%%%%%%%%%%%%%%%%%%%%%%%%%%%%%%%%%%%%%%%%%%%%%%%
\usepackage{eurosym}
\usepackage{vmargin}
\usepackage{amsmath}
\usepackage{graphics}
\usepackage{epsfig}
\usepackage{subfigure}
\usepackage{fancyhdr}

\setcounter{MaxMatrixCols}{10}
%TCIDATA{OutputFilter=LATEX.DLL}
%TCIDATA{Version=5.00.0.2570}
%TCIDATA{<META NAME="SaveForMode" CONTENT="1">}
%TCIDATA{LastRevised=Wednesday, February 23, 2011 13:24:34}
%TCIDATA{<META NAME="GraphicsSave" CONTENT="32">}
%TCIDATA{Language=American English}

\pagestyle{fancy}
\setmarginsrb{20mm}{0mm}{20mm}{25mm}{12mm}{11mm}{0mm}{11mm}
\lhead{MA4128} \rhead{Mr. Kevin O'Brien}
\chead{Advanced Data Modelling}
%\input{tcilatex}

\begin{document}
	\section*{Principal Component Analysis : Tutorial Sheet}
	\begin{enumerate}
		\item What is Dimensionality Reduction
		\item What is the KMO statistic? Describe how to interpret the KMO statistic.
		\item What is the Bartlett Test of Sphericity used for?
		\item varimax, quartimax and equamax are the commonly used methods in a certain procedure. What is this procedure? What is the purpose of the procedure.
		Which method is the most commonly used?
		\item Describe how to use a Scree plot in the context of dimensionality reduction techniques.
		\item What problems occur if a principal component analysis is done on a data matrix where the columns contain measurements on very difference scales?  What can be done to overcome this problem?
		\item Principal Component Analysis is a data reduction technique. Explain what this term
		means.
		\item The KMO is used to measure what characteristic of the data. Explain how the KMO
		measure should be interpreted.
		\item Briefly describe the Bartlett Test for Sphericity, with reference to the null and alternative
		hypotheses, and how those statements relate to the purpose of the test.
		\item Discuss three techniques for determining the appropriate number of principal components.
		\item In the context of principal components what is meant by orthogonality.
		
		\item What is the purpose of a principal component analysis?
		\item Explain the difference between PCA and factor analysis
		\item Explain what is meant by the "true" dimension of the data? How does an analyst determine the appropriate number of factors to retain. Make reference to three difference techniques
	\end{enumerate}
	\end{document}
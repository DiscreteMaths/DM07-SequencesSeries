\documentclass[]{report}

\voffset=-1.5cm
\oddsidemargin=0.0cm
\textwidth = 480pt

\usepackage{framed}
\usepackage{subfiles}
\usepackage{enumerate}
\usepackage{graphics}
\usepackage{newlfont}
\usepackage{eurosym}
\usepackage{amsmath,amsthm,amsfonts}
\usepackage{amsmath}
\usepackage{color}
\usepackage{amssymb}
\usepackage{multicol}
\usepackage[dvipsnames]{xcolor}
\usepackage{graphicx}
\begin{document}

\section*{Root-Mean-Square Error (RMSE) }
\begin{itemize}
\item The root-mean-square error (RMSE) is a frequently used measure of the difference between values (sample and population values) predicted by a model or an estimator and the values actually observed. 
\item The RMSE represents the sample standard deviation of the differences between predicted values and observed values. 

\item These individual differences are called residuals when the calculations are performed over the data sample that was used for estimation, and are called prediction errors when computed out-of-sample. The RMSE serves to aggregate the magnitudes of the errors in predictions for various times into a single measure of predictive power. 
\item RMSE is a measure of accuracy, to compare forecasting errors of different models for a particular data and not between datasets, as it is scale-dependent.
\item (Outiers)
RMSE is the square root of the average of squared errors. The effect of each error on RMSE is proportional to the size of the squared error; thus larger errors have a disproportionately large effect on RMSE. Consequently, RMSE is sensitive to outliers.
\end{itemize}
%==========================================%
\subsection*{Mean Absolute Error (MAE)}
In statistics, mean absolute error (MAE) is a measure of difference between two continuous variables. Assume X and Y are variables of paired observations that express the same phenomenon. Examples of Y versus X include comparisons of predicted versus observed, subsequent time versus initial time, and one technique of measurement versus an alternative technique of measurement. 


Consider a scatter plot of n points, where point $i$ has coordinates $(x_i, y_i)...$ Mean Absolute Error (MAE) is the average vertical distance between each point and the Y=X line, which is also known as the One-to-One line. MAE is also the average horizontal distance between each point and the Y=X line.

The Mean Absolute Error is given by:

\[{\displaystyle \mathrm {MAE} ={\frac {\sum _{i=1}^{n}\left|y_{i}-x_{i}\right|}{n}}={\frac {\sum _{i=1}^{n}\left|e_{i}\right|}{n}}.}\]

It is possible to express MAE as the sum of two components: Quantity Disagreement and Allocation Disagreement. Quantity Disagreement is the absolute value of the Mean Error. Allocation Disagreement is MAE minus Quantity Disagreement. The Mean Error is given by:

\[{\displaystyle \mathrm {ME} ={\frac {\sum _{i=1}^{n}y_{i}-x_{i}}{n}}.} \]

It is also possible to identify the types of difference by looking at an ${\displaystyle (x,y)}$ (x,y) plot. Allocation difference exists if and only if points reside on both sides of the Y=X line. Quantity difference exists when the average of the X values does not equal the average of the Y values.[2][3]
%==========================================%
\subsection*{Related measures}
Some researchers have recommended the use of Mean Absolute Error (MAE) instead of Root Mean Square Deviation. MAE possesses advantages in interpretability over RMSD. MAE is the average absolute difference between two variables designated X and Y. MAE is fundamentally easier to understand than the square root of the average of squared errors. Furthermore, each error influences MAE in direct proportion to the absolute value of the error, which is not the case for RMSD.



\end{document}



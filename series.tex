
\section*{Series}



\noindent \textbf{Series}: A series is the sum of a sequence. For a given sequence $a_0,a_1,a_2,a_3,\ldots,a_n,\ldots$ the terms in the corresponding \emph{series} $s_n$ are given by
\begin{align*}
  s_0&=a_0\\
  s_1&=a_0+a_1\\
  s_2&=a_0+a_1+a_2\\
  s_3&=a_0+a_1+a_2+a_3,\quad \text{etc}
\end{align*}
The general term $s_n$, which is the $n^{th}$ term in the series, is given by
\begin{equation*}
  s_n=a_0+a_1+a_2+\cdots+a_{n-1}+a_n
\end{equation*}
Note that
\begin{itemize}
\item Every series is itself another sequence. Therefore a series can have all the properties of sequences; boundedness, limits, etc.
\item Every sequence $a_n$ has a corresponding series $s_n$.
\item Every series $s_n$ has a corresponding sequence $a_n=s_n-s_{n-1}$.
\end{itemize}
\noindent  \textbf{Example}
Let $a_n=n^2$, So that the terms in the sequence are $0,1,4,9,16,25,\ldots$
\begin{align*}
  s_0&=0\\
  s_1&=0+1=1\\
  s_2&=0+1+4=5\\
  s_3&=0+1+4+9=14,\quad \text{etc}
\end{align*}

\subsection*{Sigma Notation}
Instead of writing out a long sum for every term in a series, the greek letter Sigma is used to denote such sums. For a sum over a range or terms $a_p$ to $a_q$, with $p<q$ let
\begin{equation*}
  \sum_{k=p}^{q}a_k = a_p+a_{p+1}+a_{p+2}+a_{p+3}+\cdots+a_{q-1}+a_q
\end{equation*}
The sum is over the variable $k$ which ranges from the lower bound $p$, increasing in steps of $1$, until it reaches the upper bound $q$. The Sigma denotes that all the terms are to be added together. With this, the general term $s_n$ in the sequence is written
\begin{multicols}{2}
$\qquad \qquad s_n=\displaystyle \sum_{k=0}^n a_k$
\columnbreak

or $\ s_n=\displaystyle \sum_{k=1}^n a_k\quad $ if $a_0$ is not defined.
\end{multicols}
\pagebreak

\noindent \textbf{Example:} Sum of first $n$ integers
\begin{equation*}
  s_n=\sum_{k=0}^n k = 0 + 1 + 2 + 3 + \cdots + (n-1) + n 
\end{equation*}
\begin{center}
\begin{tabular}{c|c|c|c|c|c|c|c|c|c|c|c}
$n$ & $0$ & $1$ & $2$ & $3$ & $4$ & $5$ & $6$ & $7$ & $8$ & $9$ & $10$ \\ \hline
$a_n$ & $0$ & $1$ & $2$ & $3$ & $4$ & $5$ & $6$ & $7$ & $8$ & $9$ & $10$\\ \hline
$s_n$ & $0$ & $1$ & $3$ & $6$ & $10$ & $15$ & $21$ & $28$ & $36$ & $45$ & $55$
\end{tabular}
\end{center}
Instead of having to perform the summation for each $s_n$, a formula for the sum of the first $n$ integers can be derived.
\begin{thm*}
  \begin{equation*}
    \sum_{k=0}^n k = \frac{n(n+1)}{2}
  \end{equation*}
\begin{proof}
  Let $s_n=\displaystyle \sum_{k=0}^n k$. Proceed by writing out this sum, then writing out the reversed sum, and then adding the two
\\
\\
  \begin{tabular}{ccccccccccccccc}
    \setlength{\arraycolsep}{0.01cm}
    $s_n$&$=$&$0$&$+$&$1$&$+$&$2$&$+$&$\cdots$&$+$&$(n-2)$&$+$&$(n-1)$&$+$&$n$\\
    $s_n$&$=$&$n$&$+$&$(n-1)$&$+$&$(n-2)$&$+$&$\cdots$&$+$&$2$&$+$&$1$&$+$&$0$\\\hline
    $2s_n$&$=$&$n$&$+$&$n$&$+$&$n$&$+$&$\cdots$&$+$&$n$&$+$&$n$&$+$&$n$\\
  \end{tabular}
\\
\\
\\
  There are $n+1$ terms in the sum, therefore
  \begin{equation*}
    2s_n = (n+1)n, \qquad \Rightarrow s_n=\frac{n(n+1)}{2}, \qquad \text{and so } \sum_{k=0}^n k=\frac{n(n+1)}{2}
  \end{equation*}
\end{proof}
\end{thm*}

\noindent \textbf{Other Series Theorems}\\
If $a_n$ and $b_n$ are sequences, and $c$ is a constant then,
\[\sum_{k=0}^{n} \left(a_n+b_n\right) = \sum_{k=0}^{n} a_n + \sum_{k=0}^{n}b_n \]
\[\sum_{k=0}^{n} c a_n = c \sum_{k=0}^{n} a_n \]
\[\sum_{k=0}^{n} c = (n+1) c\]
\pagebreak

\subsection*{Arithmetic Series}
And \emph{arithmetic sequence} $a_n$ is a sequences whose terms increase by a constant difference $d$ at each step. And a \emph{arithmetic series} $s_n$ is the sum of an arithmetic sequence. Formally
\begin{equation*}
  a_n=a_0+nd, \qquad s_n=\sum_{k=0}^n \left( a_0 + nd \right)
\end{equation*}
\noindent \textbf{Example:}\\
Let $a_n=10+3n$.
\begin{center}
\begin{tabular}{c|c|c|c|c|c|c|c|c}
$n$ & $0$ & $1$ & $2$ & $3$ & $4$ & $5$ & $6$ & $7$  \\ \hline
$a_n$ & $10$ & $13$ & $16$ & $19$ & $22$ & $25$ & $28$ & $31$ \\ \hline
$s_n$ & $10$ & $23$ & $39$ & $58$ & $80$ & $105$ & $133$ & $164$
\end{tabular}
\end{center}
Instead of having to sum up every term in an arithmetic series, a formula for the $n^{th}$ term in the series can be derived.
\begin{thm*}
  For an arithmetic series
  \begin{equation*}
    s_n = (n+1)\left(a_0 + \frac{n}{2} d\right)
  \end{equation*}
\begin{proof}
This can be proved by using the theorems on the previous page, in particular using the formula for the sum of the first $n$ integers.
  \begin{align*}
    s_n=\sum_{k=0}^n a_k = \sum_{k=0}^n a_0+kd &= \sum_{k=0}^n a_0+\sum_{k=0}^n kd\\
    &= (n+1)a_0+d\sum_{k=0}^n k\\
    &= (n+1)a_0+d\frac{n(n+1)}{2}= \frac{(n+1)}{2} \left( 2a_0+nd\right)
  \end{align*}
\end{proof}
\end{thm*}

\noindent \textbf{Example:} For $a_n=10+3n$, the formula gives
\begin{equation*}
  s_7=(7+1) \left( 10+\dfrac{7}{2} \cdot 3 \right) = 8 \left(  \dfrac{41}{2} \right) = 164
\end{equation*}
Which matches what was calculated for $s_7$ in the table above. Using the formula, terms even further out can now be calculated without direct summation. For example
\begin{align*}
  s_{100}&=(100+1) \left( 10+\dfrac{100}{2} \cdot 3 \right) = 101 \left(  \dfrac{320}{2} \right) = 16160\\
  s_{1000}&= 1001 \left( 10+  \dfrac{1000}{2} \cdot 3 \right) =  1511510\\
\end{align*}


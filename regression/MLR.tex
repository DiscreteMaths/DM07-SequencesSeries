\section{Multiple Linear Regression}
\documentclass[a4paper,12pt]{article}
%%%%%%%%%%%%%%%%%%%%%%%%%%%%%%%%%%%%%%%%%%%%%%%%%%%%%%%%%%%%%%%%%%%%%%%%%%%%%%%%%%%%%%%%%%%%%%%%%%%%%%%%%%%%%%%%%%%%%%%%%%%%%%%%%%%%%%%%%%%%%%%%%%%%%%%%%%%%%%%%%%%%%%%%%%%%%%%%%%%%%%%%%%%%%%%%%%%%%%%%%%%%%%%%%%%%%%%%%%%%%%%%%%%%%%%%%%%%%%%%%%%%%%%%%%%%
\usepackage{eurosym}
\usepackage{vmargin}
\usepackage{amsmath}
\usepackage{graphics}
\usepackage{epsfig}
\usepackage{framed}
\usepackage{subfigure}
\usepackage{fancyhdr}

\setcounter{MaxMatrixCols}{10}
%TCIDATA{OutputFilter=LATEX.DLL}
%TCIDATA{Version=5.00.0.2570}
%TCIDATA{<META NAME="SaveForMode"CONTENT="1">}
%TCIDATA{LastRevised=Wednesday, February 23, 201113:24:34}
%TCIDATA{<META NAME="GraphicsSave" CONTENT="32">}
%TCIDATA{Language=American English}

\pagestyle{fancy}
\setmarginsrb{20mm}{0mm}{20mm}{25mm}{12mm}{11mm}{0mm}{11mm}
\lhead{MA4128} \rhead{Kevin O'Brien} \chead{Linear Models} %\input{tcilatex}

%===========================================================%

\begin{document}

\subsection{What is Multiple Linear Regression}

Multiple regression is a statistical technique that allows us to predict a numeric value on the response variable on the basis of the observed values on several other independent variables.


\[\hat{y} = b_0 + b_1x_1 + b_2x_2 + \ldots \]

\begin{itemize}
\item $\hat{y}$ is the \textbf{\textit{fitted value}} for the dependent variable \textbf{$Y$}, given a linear combination of values for the independent valriables.

\item $x_i$ is the value for independent variable \textbf{$X_i$}. (For Example, $x_1$ is the value for independent variable \textbf{$X_1$}.)
\item $b_o$ is the constant regression estimate ( commonly known as the \textbf{Intercept Estimate} in the case of simple linear regression).
\item $b_i$ is the regression estimate for Independent Variable \textbf{$X_1$} ( commonly known as the \textbf{Slope Estimate} in the case of simple linear regression).
\end{itemize}

\subsubsection{Simple Example}
Suppose we were interested in predicting how much an individual enjoys their job. Independent Variables such as salary, extent of academic qualifications, age, sex, number of years in full-time employment and socioeconomic status might all contribute towards \textbf{\textit{job satisfaction}}.

If we collected data on all of these variables, perhaps by surveying a few hundred members of the public, we would be able to see how many and which of these variables gave rise to the most accurate prediction of job satisfaction. We might find that job satisfaction is most accurately predicted by type of occupation, salary and years in full-time employment, with the other variables not helping us to predict job satisfaction.


%-----------------------------------------------------------------------------------------%
\section{Multiple Linear Regression}
Multiple regression: To quantify the relationship between several independent (predictor) variables and a dependent (response) variable. The coefficients ($a, b_{1} to b_{i}$) are estimated by the least squares method, which is equivalent to maximum likelihood estimation. A multiple regression model is built upon three major assumptions:

\begin{enumerate}
	\item The response variable is normally distributed,
	\item The residual variance does not vary for small and large fitted values (constant variance),
	\item The observations (explanatory variables) are independent.
\end{enumerate}


\subsection{Dummy Variables}
A dummy variable is a numerical variable used in regression analysis to represent subgroups of the sample in your study. In research design, a dummy variable is often used to distinguish different treatment groups. In the simplest case, we would use a 0,1 dummy variable where a person is given a value of 0 if they are in the control group or a 1 if they are in the treated group. Dummy variables are useful because they enable us to use a single regression equation to represent multiple groups. This means that we don't need to write out separate equation models for each subgroup.


\end{document}
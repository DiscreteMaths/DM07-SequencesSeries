
\documentclass[12pt]{article}
%\usepackage[final]{pdfpages}

\usepackage{graphicx}
\graphicspath{{/Users/kevinhayes/Documents/teaching/images/}}

\usepackage{tikz}
\usetikzlibrary{arrows}

\newcommand{\bbr}{\Bbb{R}}
\newcommand{\zn}{\Bbb{Z}^n}

%\usepackage{epsfig}
%\usepackage{subfigure}
\usepackage{amscd}
\usepackage{amssymb}
\usepackage{amsbsy}
\usepackage{amsthm}
\usepackage{natbib}
\usepackage{amsbsy}
\usepackage{enumerate}
\usepackage{amsmath}
\usepackage{eurosym}
%\usepackage{beamerarticle}
\usepackage{txfonts}
\usepackage{fancyvrb}
\usepackage{fancyhdr}
\usepackage{natbib}
\bibliographystyle{chicago}

\usepackage{vmargin}
% left top textwidth textheight headheight
% headsep footheight footskip
\setmargins{2.0cm}{2.5cm}{16 cm}{22cm}{0.5cm}{0cm}{1cm}{1cm}
\renewcommand{\baselinestretch}{1.3}


\pagenumbering{arabic}

\begin{document}
\section{Proof by Induction}


\textbf{Proof by Induction :  three steps}
\begin{itemize}
\item Step 1 : Base Case
\begin{itemize}
\item[$\bullet$]  Demonstrate for $n=1$.
\item[$\bullet$]  Normally a simple calculation.
\end{itemize}
\item Step 2 : Induction Hypothesis 

\begin{itemize}
\item[$\bullet$] State assumption for $n=k$
\item[$\bullet$]  Can extend assumption of statement for $n=k-1$ in the case of multi-phase recurrence rules 
\item[$\bullet$] For example, $u_{n+1}$ is evaluated by $u_{n}$ and $u_{n-1}$, the two previous terms.
\end{itemize}

\item Step 3 : Induction Step.
\begin{itemize}
\item[$\bullet$]  Demonstrate for $n=k+1$.
\item[$\bullet$] 
This state is main computational component of Proof by Induction method
\end{itemize}
\end{itemize}




\section{Proof By Induction}
In Step 1, you are trying to show it is true for specific values. You are free to do this test with just one value or fifty values of your choice or more.

However, showing it is true for one million values or more still does not prove it will be true for all values. This is a very important observation!

In Step 2, since you have already shown that it is true for one or more values, it is logical to suppose or assume it is true for n =k or generally speaking.

We usually use the asumption that we make here to complete or prove Step  3

In Step 3, you finally show it is true for any values. Notice that step 2 did not show it is true for any values
\newpage


Let the summation $s_n$ be defined as follows:
\[s_n = 1 + 3 + 5 + \ldots + (2n − 1) \] for $n \in \mathbb{Z}^{+}$.

Use the method of induction to prove that $s_n = n^2$ for all $n \geq 1$.

\end{document}

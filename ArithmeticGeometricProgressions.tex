% - http://www.robots.ox.ac.uk/~dwm/Courses/1DF/1DF-N3.pdf
%========================================================================================%

3.1 Arithmetic progression
An arithmetic series or progression is a sequence of n numbers successive members of
which differ by the same amount d. That is, fa1; (a1+d); (a1+2d); \ldots ; (a1+(n􀀀1)d)g.
The i-th term is ai = a1 + (i - 1)d.
Suppose we wanted the sum to the n-th term, Sn =
Pn
i=1 ai . Write the sequence in
the forward and reverse directions, then add:
Sn= a1 + (a1 + d) + (a1 + 2d) +\ldots+(a1 + (n -  1)d)
=(a1 + (n -  1)d)+(a1 + (n -  2)d)+(a1 + (n -  3)d)+\ldots+ a1
(3.1)

The sum of corresponding terms is always 2a1 + (n -  1)d and there are n of them so
the sum is
Sn =
1
2
n(2a1 + (n -  1)d) =
1
2
n(a1 + an) : (3.2)


Convergence.
 The sum of this series always diverges as more terms are added. The
only series that remains finite is a string of zeros!
1

%========================================================================================%
\subsection*{3.2 Geometric progression}
A geometric series or progression is a sequence of n numbers successive members of
which are multiplied by a common factor r . That is, fa1; ra1; r 2a1; \ldots ; r n􀀀1a1)g. The
i-th term is ai = r i􀀀1a1. The sum to n terms is
\[Sn = a1 + ra1 + r 2a1 + \ldots + r n􀀀1a1)\]
\[rSn = ra1 + r 2a1 + \ldots + r n􀀀1a1 + r na1)\]
(3.3)
So
\[ Sn(1  -r ) = a1(1  -r n) )Sn =
a1(1  -r n)
1  -r
: (3.4)\]

Convergence. 

The sum of an infinite number of terms will diverge to 1 unless
jr j < 1. If this condition is satisfied, r n ! 0 and
S1 =
a1
1  -r
: (3.5)

%========================================================================================%
\subsection*{3.3 Convergence vs Divergence}
The set S1; S2; :::Sn::: is known as the set of partial sums. The definition of convergence
is
Definition of convergence
lim
n!1
Sn ! 


Finite value (3.6)
If a series does not converge, it diverges. Series can diverge in different ways. Some
have partials sums that tend either to +1 or 􀀀1, in others successive partial sums
oscillate between 1.

%========================================================================================%

3.4 Absolute vs Conditional Convergence
Suppose a series infinite sum
S1 = a1 + a2 + a3 + \ldots (3.7)
is convergent. But suppose that the series contains positive and negative terms. Clearly
the positive and negative terms will cancel somewhat, helping to constrain the growth
of the sum. We could ask a tougher question: does the following converge?
S0
1 = ja1j + ja2j + ja3j + \ldots (3.8)
If it does, the original series is absolutely convergent. Furthermore, if we do not know
whether S converges, but can prove S0 converges then this is sufficient to prove that
S does converge.
%========================================================================================%

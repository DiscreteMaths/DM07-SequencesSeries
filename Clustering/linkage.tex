
\documentclass[a4paper,12pt]{article}
%%%%%%%%%%%%%%%%%%%%%%%%%%%%%%%%%%%%%%%%%%%%%%%%%%%%%%%%%%%%%%%%%%%%%%%%%%%%%%%%%%%%%%%%%%%%%%%%%%%%%%%%%%%%%%%%%%%%%%%%%%%%%%%%%%%%%%%%%%%%%%%%%%%%%%%%%%%%%%%%%%%%%%%%%%%%%%%%%%%%%%%%%%%%%%%%%%%%%%%%%%%%%%%%%%%%%%%%%%%%%%%%%%%%%%%%%%%%%%%%%%%%%%%%%%%%
\usepackage{eurosym}
\usepackage{vmargin}
\usepackage{amsmath}
\usepackage{graphics}
\usepackage{epsfig}
\usepackage{subfigure}
\usepackage{fancyhdr}
%\usepackage{listings}
\usepackage{framed}
\usepackage{graphicx}

\setcounter{MaxMatrixCols}{10}
%TCIDATA{OutputFilter=LATEX.DLL}
%TCIDATA{Version=5.00.0.2570}
%TCIDATA{<META NAME="SaveForMode" CONTENT="1">}
%TCIDATA{LastRevised=Wednesday, February 23, 2011 13:24:34}
%TCIDATA{<META NAME="GraphicsSave" CONTENT="32">}
%TCIDATA{Language=American English}

\pagestyle{fancy}
\setmarginsrb{20mm}{0mm}{20mm}{25mm}{12mm}{11mm}{0mm}{11mm}
\lhead{MA4128} \rhead{Mr. Kevin O'Brien}
\chead{Advanced Data Modelling}
%\input{tcilatex}


% http://www.norusis.com/pdf/SPC_v13.pdf
\begin{document}
	
	
	%SESSION 1: Hierarchical Clustering
	% Hierarchical clustering - dendrograms
	% Divisive vs. agglomerative methods
	% Different linkage methods
	
	%SESSION 2: K-means Clustering
	
	\tableofcontents
	\newpage




The hierarchical clustering procedure attempts to identify relatively homogeneous groups of cases (or variables) based on selected
characteristics. For example: cluster television shows into homogeneous groups based on viewer
characteristics. In hierarchical clustering, an algorithm is used that starts with each case (or variable) in a
separate cluster and combines clusters until only one is left.



To cluster cases you need to identify variables you wish to be considered in creating clusters for the cases.
The variables to be used for cluster formation are here: picture quality (5 measures), reception quality (3
measures), audio quality (3 measures), ease of programming (1 measure), number of events (1 measure),
number of days for future programming (1 measure), remote control (3 measures), and extras (3 measures).
Pass these in the Variable(s) box.

Cluster Method: Choose the procedure for combining clusters. The default procedure is called the
between-group linkage. SPSS computes the smallest average distance between all group pairs and
combines the two groups that are closest. The procedure begins with as many clusters as there are cases
(here: 21). At step one, the two cases with the smallest distance between them are clustered. Then SPSS
computes distances once more and combines the two that are next closest. After the second step you will
have either 18 individual cases and one cluster of 3 cases, or 17 individual cases and two clusters of two
cases each. The process continues until all cases are grouped into one large cluster.
Measure: Indicate what method is used for distance measuring, the default is Squared Euclidean distance. 

\subsection{Linkage methods}
\begin{itemize}
	\item  Single linkage (minimum distance)
	\item  Complete linkage (maximum distance)
	\item  Average linkage
\end{itemize}

%http://www.rdg.ac.uk/~aes02mm/supermarket.sav

\subsubsection{Ward's method}
\begin{itemize}
	\item  Compute sum of squared distances within clusters
	\item  Aggregate clusters with the minimum increase in the
	overall sum of squares
\end{itemize}
\subsubsection{Centroid method}
The distance between two clusters is defined as the
difference between the centroids (cluster averages)

\end{document}
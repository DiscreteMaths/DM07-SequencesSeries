\documentclass[a4paper,12pt]{article}
%%%%%%%%%%%%%%%%%%%%%%%%%%%%%%%%%%%%%%%%%%%%%%%%%%%%%%%%%%%%%%%%%%%%%%%%%%%%%%%%%%%%%%%%%%%%%%%%%%%%%%%%%%%%%%%%%%%%%%%%%%%%%%%%%%%%%%%%%%%%%%%%%%%%%%%%%%%%%%%%%%%%%%%%%%%%%%%%%%%%%%%%%%%%%%%%%%%%%%%%%%%%%%%%%%%%%%%%%%%%%%%%%%%%%%%%%%%%%%%%%%%%%%%%%%%%
\usepackage{eurosym}
\usepackage{vmargin}
\usepackage{amsmath}
\usepackage{graphics}
\usepackage{epsfig}
\usepackage{subfigure}
\usepackage{fancyhdr}

\setcounter{MaxMatrixCols}{10}
%TCIDATA{OutputFilter=LATEX.DLL}
%TCIDATA{Version=5.00.0.2570}
%TCIDATA{<META NAME="SaveForMode"CONTENT="1">}
%TCIDATA{LastRevised=Wednesday, February 23, 201113:24:34}
%TCIDATA{<META NAME="GraphicsSave" CONTENT="32">}
%TCIDATA{Language=American English}

\pagestyle{fancy}
\setmarginsrb{20mm}{0mm}{20mm}{25mm}{12mm}{11mm}{0mm}{11mm}
\lhead{MA4128} \rhead{Kevin O'Brien} \chead{Week 6} %\input{tcilatex}

\begin{document}

\tableofcontents
\newpage

% http://www.norusis.com/pdf/SPC_v13.pdf
\section{Review of Last Class}
\begin{itemize}
\item Cluster analysis is a convenient method for identifying homogenous groups of
objects called clusters. Objects (or cases, observations) in a specific cluster share
many characteristics, but are very dissimilar to objects not belonging to that cluster.
\item  There are three cluster analysis approaches: hierarchical methods,
partitioning methods (more precisely, k-means), and two-step clustering,
which is largely a combination of the first two methods. In the last class we looked as hierarchical clustering analysis.
\item Each of these procedures
follows a different approach to grouping the most similar objects into a cluster and
to determining each object�s cluster membership.
\item Some approaches � most notably hierarchical methods � require us to specify how similar or different objects
    are in order to identify different clusters. Most software packages, such as SPSS, calculate a measure
of (dis)similarity by estimating the distance between pairs of objects. Objects with
smaller distances between one another are more similar, whereas objects with larger
distances are more dissimilar.
\item An important problem in the application of cluster analysis is the decision
regarding how many clusters should be derived from the data. This question is
explored in the next step of the analysis. Sometimes, however,
number of segments that have to be derived from the data will be known in advance.
\item
By choosing a specific clustering procedure, we determine how clusters are to be
formed. (This always involves optimizing some kind of criterion, such as minimizing
the within-cluster variance (i.e., the clustering variables� overall variance of
objects in a specific cluster), or maximizing the distance between the objects or
clusters). The procedure could also address the question of how to determine the
(dis)similarity between objects in a newly formed cluster and the remaining objects
in the dataset.
\item
Hierarchical clustering procedures are characterized by the tree-like structure
established in the course of the analysis. Most hierarchical techniques fall into a
category called agglomerative clustering. In this category, clusters are consecutively
formed from objects. Initially, this type of procedure starts with each object
representing an individual cluster. These clusters are then sequentially merged
according to their similarity. First, the two most similar clusters (i.e., those with
the smallest distance between them) are merged to form a new cluster at the bottom
of the hierarchy. In the next step, another pair of clusters is merged and linked to a
higher level of the hierarchy, and so on. This allows a hierarchy of clusters to be
established from the bottom up.
\item A cluster hierarchy can also be generated top-down. In this divisive clustering,
all objects are initially merged into a single cluster, which is then gradually split up. Divisive procedures are quite rarely used in practice. We therefore
concentrate on the agglomerative clustering procedures.
\item This means that if an object is assigned
to a certain cluster, there is no possibility of reassigning this object to another
cluster. This is an important distinction between these types of clustering and
partitioning methods such as \textbf{\textit{k-means}}.
\newpage


\end{itemize}

\end{document} 
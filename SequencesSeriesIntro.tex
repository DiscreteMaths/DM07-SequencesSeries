\documentclass{beamer}
\usepackage{amsmath}
\usepackage{amssymb}

\begin{document}
	%======================================================%
	
	\section{Learning Outcomes}
	\begin{description}
		\item[3a] Sum arithmetic, geometric and telescoping series; 
		\item[3b] test series for convergence; 
		\item[3c] find the Maclaurin series of a function; 
		\item[3d] manipulate power series; 
		\item[3e] use l'Hopital's rule. 
		\item[3f] Integrate standard functions using substitution and parts; 
		\item[3g] Apply to calculation of areas and volumes. 
	\end{description}
	
	
	%===========================================================================%
	
	\section{Limit Properties}
	
	\textbf{Sequences and Series}\\
	\begin{itemize}
		\item In this chapter we’ll be taking a look at sequences and (infinite) series. 
		\item Actually, this chapter will deal almost exclusively with series. 
		\item However, we also need to understand some of the basics of sequences in order to properly deal with series.  
		\item We will therefore, spend a little time on sequences as well.
	\end{itemize}
	
	%===========================================================================%
	
	\section{Limit Properties}
	
	\begin{itemize}
		\item  Series is one of those topics that many students don’t find all that useful. \item To be honest, many students will never see series outside of their calculus class. \item However, series do play an important role in the field of ordinary differential equations and without series large portions of the field of partial differential equations would not be possible.
	\end{itemize}
	
	%===========================================================================%
	
	\section{Limit Properties}
	
	In other words, series is an important topic even if you won’t ever see any of the applications.  Most of the applications are beyond the scope of most Calculus courses and tend to occur in classes that many students don’t take.  So, as you go through this material keep in mind that these do have applications even if we won’t really be covering many of them in this class.
	
	
	%===========================================================================%
	
	\section{Limit Properties}
	
	Here is a list of topics in this section
	
	\begin{description}
		\item[Sequences ] We will start the chapter off with a brief discussion of sequences.  This section will focus on the basic terminology and convergence of sequences
		
		\item[More on Sequences]  Here we will take a quick look about monotonic and bounded sequences.
		
		\item[Series]  The Basics  In this section we will discuss some of the basics of infinite series.
	\end{description}
	
	%===========================================================================%
	
	\section{Limit Properties}
	
	\begin{description}
		\item[Series Convergence/Divergence]  Most of this chapter will be about the convergence/divergence of a series so we will give the basic ideas and definitions in this section.
		
		\item[Special Series]  We will look at the Geometric Series, Telescoping Series, and Harmonic Series in this section.
		
		\item[Integral Test]  Using the Integral Test to determine if a series converges or diverges.
	\end{description}
	
	%===========================================================================%
	
	\section{Limit Properties}
	
	\begin{description}
		\item[Comparison Test/Limit Comparison Test]  Using the Comparison Test and Limit Comparison Tests to determine if a series converges or diverges.
		
		\item[Alternating Series Test]  Using the Alternating Series Test to determine if a series converges or diverges.
		
		\item[Absolute Convergence]  A brief discussion on absolute convergence and how it differs from convergence.
		
		\item[Ratio Test]  Using the Ratio Test to determine if a series converges or diverges.
	\end{description}
	
	%===========================================================================%
	
	\section{Limit Properties}
	
	\begin{description}
		\item[Root Test]  Using the Root Test to determine if a series converges or diverges.
		
		\item[Strategy for Series]  A set of general guidelines to use when deciding which test to use.
		
		\item[Estimating the Value of a Series]  Here we will look at estimating the value of an infinite series.
		
		\item[Power Series]  An introduction to power series and some of the basic concepts.
		
	\end{description}
	
	%===========================================================================%
	
	\section{Limit Properties}
	
	\begin{description}
		\item[Power Series and Functions]  In this section we will start looking at how to find a power series representation of a function.
		
		\item[Taylor Series]  Here we will discuss how to find the Taylor/Maclaurin Series for a function.
		
		\item[Applications of Series]  In this section we will take a quick look at a couple of applications of series.
		
		\item[Binomial Series]  A brief look at binomial series.
	\end{description}
	
	%===========================================================================%
	
\end{document}